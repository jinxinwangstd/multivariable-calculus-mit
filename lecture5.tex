\documentclass{article}
\usepackage[utf8]{inputenc}
\usepackage{amsmath}
\usepackage{amssymb}
\usepackage{amsthm}
\usepackage{tikz}
\setlength{\parindent}{0pt}

\newtheorem*{theorem}{Theorem}
\newtheorem*{definition}{Definition}
\newtheorem*{lemma}{Lemma}
\newtheorem*{corollary}{Corollary}
\newtheorem{example}{Example}

\title{Lecture 5: Parametric Equations}
\author{}
\date{}

\begin{document}
    
\maketitle

\section{Equations of Lines in Space}

From previous lecture, we see that a line can be represented as the intersection 
of two planes. However, as equations it is not so easy to use because we need to 
solve it first.

Another representation is to treat a line as the trajectory of a moving point, 
and include a parameter in the equation to describe different positions in the 
line. Such representation is called \textbf{parametric equations}.

\begin{example}
  Find the parametric equation of the line through $Q_0 = (-1, 2, 2)$ and 
  $Q_1 = (1, 3, -1)$.

  Given that the line can be the trajectory of a moving point, the parameter in 
  the equation could be the time $t$. Suppose that at $t = 0$, the moving point 
  is in $Q_0$, and at $t = 1$, the moving point is in $Q_1$, and suppose the 
  point moves at a constant speed.

  The point $Q = (x, y, z)$ in the line satisfies the following equation:
  \[
    \vec{Q_0Q} = t \cdot \vec{Q_0Q_1}
  \]
  \[
    <x + 1, y - 2, z - 2> = t \cdot <2, 1, -3>
  \]
\end{example}

\end{document}