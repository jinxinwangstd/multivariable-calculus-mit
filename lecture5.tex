\documentclass{article}
\usepackage[utf8]{inputenc}
\usepackage{amsmath}
\usepackage{amssymb}
\usepackage{amsthm}
\usepackage{tikz}
\setlength{\parindent}{0pt}

\newtheorem*{theorem}{Theorem}
\newtheorem*{definition}{Definition}
\newtheorem*{lemma}{Lemma}
\newtheorem*{corollary}{Corollary}
\newtheorem{example}{Example}

\title{Lecture 5: Parametric Equations}
\author{}
\date{}

\begin{document}
    
\maketitle

\section{Equations of Lines in Space}

From previous lecture, we see that a line can be represented as the intersection 
of two planes. However, as equations it is not so easy to use because we need to 
solve it first.

Another representation is to treat a line as the trajectory of a moving point, 
and include a parameter in the equation to describe different positions in the 
line. Such representation is called \textbf{parametric equations}.

\begin{example}
  Find the parametric equation of the line through $Q_0 = (-1, 2, 2)$ and 
  $Q_1 = (1, 3, -1)$.

  Given that the line can be the trajectory of a moving point, the parameter in 
  the equation could be the time $t$. Suppose that at $t = 0$, the moving point 
  is in $Q_0$, and at $t = 1$, the moving point is in $Q_1$, and suppose the 
  point moves at a constant speed.

  With parametric equations, the coordinates of the point $Q$ in the line are 
  functions of the parameter. Therefore, $Q = (x(t), y(t), z(t))$. The point $Q$ 
  in the line also satisfies the following equation:
  \[
    \vec{Q_0Q} = t \cdot \vec{Q_0Q_1}
  \]
  \[
    <x + 1, y - 2, z - 2> = t \cdot <2, 1, -3>
  \]
  \[
    \left\{ \begin{array}{ll}
    x(t) = 2t - 1 \\
    y(t) = t + 2 \\
    z(t) = -3t + 2 \\
    \end{array} \right.
  \]
  That is the parametric equation of the described line.
\end{example}

Notice that the coefficients of the parameter are components of a vector with 
the same direction as the line (either one of the two directions a line has). 
The constant terms in the parametric equation are coordinates of the point when 
the parameter is equal to 0.

A line can have infinite different parametric equations. With different time 
assumption and speed, the coefficients can be different, but they all express 
the same line.

The reason why the parametric equation of a line only has one parameter is that 
the points on a line only have one dimension (degree of freedom). It doesn't 
matter whether it is a line or a curve.

\begin{example}
  Consider the plane $x + 2y + 4z = 7$, and two points $Q_0 = (-1, 2, 2)$ and 
  $Q_1 = (1, 3, -1)$.

  1) Whether $Q_0$ and $Q_1$ are \underline{B}. \\
  A. on the same side. \\
  B. on the opposite sides. \\
  C. on the plane. \\
  D. cannot be decided.

  Reasons: \\
  For $Q_0 = (-1, 2, 2)$, $x + 2y + 4z = 11$. Therefore, $Q_0$ is in the plane 
  $P_0$ whose equation is $x + 2y + 4z = 11$. \\
  For $Q_1 = (1, 3, -1)$, $x + 2y + 4z = 3$. Therefore, $Q_1$ is in the plane 
  $P_1$ whose equation is $x + 2y + 4z = 3$. \\
  Assume the plane with the equation $x + 2y + 4z = 7$ is called $P$. \\
  The intersection point of these three planes with the $z$ axis are 
  $(0, 0, \frac{11}{4})$, $(0, 0, \frac{3}{4})$, and $(0, 0, \frac{7}{4})$ 
  respectively, among which the intersection point of the plane $P$ are in 
  between the ones of other two planes. \\
  Since the planes $P$, $P_0$, and $P_1$ are all parallel to each other, $P$ is 
  in between other two planes. Hence, $P$ is in between $Q_0$ and $Q_1$. 
  Therefore, $Q_0$ and $Q_1$ are on the opposite sides of the plane 
  $x + 2y + 4z = 7$.

  \bigskip

  2) Does the line through $Q_0$ and $Q_1$ pass through the plane? If so, what 
  is the intersection point?

  According to the previous example, we know that the parametric equation of the 
  line is:
  \[
    \left\{ \begin{array}{ll}
    x(t) = 2t - 1 \\
    y(t) = t + 2 \\
    z(t) = -3t + 2 \\
    \end{array} \right.
  \]
  For the intersection point, the following equation holds:
  \[
    x(t) + 2y(t) + 4z(t) = 7
  \]
  \[
    (2t - 1) + 2(t + 2) + 4(-3t + 2) = 7
  \]
  \[
    -8t + 11 = 7
  \]
  \[
    t = \frac{1}{2}
  \]
  Therefore, the intersection point $Q = (0, \frac{5}{2}, \frac{1}{2})$.

  Notice that the coefficient of the parameter for the intersection point is 
  the dot product of the direction of the line and the normal vector of the 
  plane. If the coefficient is 0, from geometric point of view it means the 
  direction of the line is perpendicular to the normal vector of the plane, 
  which means the line is parallel to the plane; from algebraic point of view, 
  the equation of the intersection point either always holds or has no 
  solution, which also means the line is parallel to the plane.

\end{example}

\section{General Usage of Parametric Equations}

More generally, we can use parametric equations for arbitrary trajectory in 
the plane or space.

\begin{example}
  Find the parametric equation of a cycloid, which is the curve traced by a 
  fixed point on a circle as it rolls along a straight line without slipping.

  Solution:

  In this case, there is a natural parameter which is the angle $\theta$ the 
  circle has rolled, because the position of the fixed point is solely decided 
  by the angle. Suppose the start point of the circle is when the intersection 
  between the circle and the $x$ axis is the origin $O$, which is also the 
  tracing point. After rolling an angle $\theta$, the fix point is at position 
  $B$, and the current intersection point is $A$. The center of the circle is 
  $C$, and its radius is $a$.

  To find the parametric equation of the cycloid, we need to find the 
  parametric equation of $\vec{OB}$.
  \[
    \vec{OB} = \vec{OA} + \vec{AC} + \vec{CB}
  \]
  Since the circle rolls along the $x$ axis without slipping,
  \[
    \vec{OA} = <a\theta, 0>
  \]
  Since $\vec{AC}$ is always perpendicular to the $x$ axis and the magnitude 
  is the radius of the circle,
  \[
    \vec{AC} = <0, a>
  \]
  We can decompose $\vec{CB}$ into horizontal and vertical directions,
  \[
    \vec{CB} = <-a\sin\theta, -a\cos\theta>
  \]
  Notice that we can verify that the equations of the three vectors hold for 
  any $\theta$, not only in $[0, 2\pi]$. Therefore,
  \[
    \begin{split}
    \vec{OB} &= \vec{OA} + \vec{AC} + \vec{CB} \\
             &= <a\theta - a\sin\theta, a - a\cos\theta> \\
    \end{split}
  \]
  Therefore, the parametric equation of a cycloid is
  \[
    \left\{ \begin{array}{ll}
    x(t) = a\theta - a\sin\theta \\
    y(t) = a - a\cos\theta \\
    \end{array} \right.
  \]

  \bigskip

  Following: What does the curve look like between two humps, i.e. around 
  $\theta = 2n\pi$?

  To figure it out, we need to look at the derivative $\frac{dy}{dx}$ around 
  $\theta = 2n\pi$.
  \[
    \begin{split}
      \frac{dy}{dx}|_{\theta = 2n\pi} &= (\frac{dy}{d\theta} / \frac{dx}{d\theta})|_{\theta = 2n\pi} \\
                                      &= (\frac{d(a - a\cos\theta)}{d\theta} / \frac{d(a\theta - a\sin\theta)}{d\theta})|_{\theta = 2n\pi} \\
                                      &= \frac{a\sin\theta}{a - a\cos\theta}|_{\theta = 2n\pi} \\
                                      &= \frac{0}{0} \\
    \end{split}
  \]
  According to L'Hospital rule,
  \[
    \begin{split}
      \frac{dy}{dx}|_{\theta = 2n\pi} &= \frac{a\sin\theta}{a - a\cos\theta}|_{\theta = 2n\pi} \\
                                      &= (\frac{d(a\sin\theta)}{d\theta} / \frac{d(a - a\cos\theta)}{d\theta})|_{\theta = 2n\pi} \\
                                      &= \frac{a\cos\theta}{a\sin\theta}|_{\theta = 2n\pi} \\
                                      &= \frac{a}{0} \\
                                      &= \infty
    \end{split}
  \]
  Therefore, the slope of a cycloid between two humps, i.e. around 
  $\theta = 2n\pi$, is infinity, which means the tangent line in the 
  intersection of two humps is vertical, perpendicular to the $x$ axis.
\end{example}
 
\end{document}