\documentclass{article}
\usepackage[utf8]{inputenc}
\usepackage{amsmath}
\usepackage{amssymb}
\usepackage{graphicx}
\setlength{\parindent}{0pt}

\title{Problem Set 1}
\author{}
\date{}

\begin{document}

\begin{center}
{\rmfamily\bfseries\Large 18.02 EXERCISES}

\vspace{25px}

{\rmfamily\bfseries\LARGE Problem Set 1: Vectors, Determinants and Planes}

{\rmfamily\bfseries\large Part I}
\end{center}

Unit 1 Vectors

1. Find the magnitude and direction of the vectors\\
a) $\vec{i} + \vec{j} + \vec{k}$
b) $2\vec{i} - \vec{j} + 2\vec{k}$
c) $3\vec{i} - 6\vec{j} - 2\vec{k}$

2. a) Let $P$ and $Q$ be two points in space, and X the midpoint of the line
segment $PQ$. Let $O$ be an arbitrary fixed point; show that as vectors, $OX =
\frac{1}{2}(OP + OQ)$.\\
b) With the notation of part (a), assume that X divides the line segment $PQ$
in the ratio $r:s$, where $r + s = 1$. Derive an expression for $OX$ in terms
of $OP$ and $OQ$.

3. What are the $\vec{i} \vec{j}$-components of a plane vector $\vec{A}$ of
length 3, if it makes an angle of $30^{\circ}$ with $\vec{i}$ and $60^{\circ}$
with $\vec{j}$. Is the second condition redundant?

4. A small plane wishes to fly due north at 200 mph (as seen from the ground),
in a wind blowing from the northeast at 50 mph. Tell with what vector velocity
in the air it should travel (given the $\vec{i} \vec{j}$-components).\\
Solution:\\
TODO

5. Let $\vec{A} = a \vec{i} + b \vec{j}$ be a plane vector; find in terms of
$a$ and $b$ the vectors $\vec{A'}$ and $\vec{A''}$ resulting from rotating
$\vec{A}$ by $90^{\circ}$ \hspace{10px} a) clockwise \hspace{10px} b)
counterclockwise.\\
c) Let $\vec{i'} = (3 \vec{i} + 4 \vec{j}) / 5$. Show that $\vec{i'}$ is a unit
vector, and use the first part of the exercise to find a vector $\vec{j'}$ such
that $\vec{i'}$, $\vec{j'}$ forms a right-handed coordinate system.

6. The direction of a space vector is in engineering practice often given by
its direction cosines. To describe these, let $\vec{A} = a \vec{i} + b \vec{j}
+ c \vec{k}$ be a space vector, represented as an origin vector, and let
$\alpha$, $\beta$, and $\gamma$ be the three angles ($\le \pi$) that $\vec{A}$
makes respectively with $\vec{i}$, $\vec{j}$, and $\vec{k}$.\\
a) Show that $dir \vec{A} = \cos\alpha \vec{i} + \cos\beta \vec{j} +
\cos\gamma \vec{k}$. (The three coefficients are called the \emph{direction
cosines} of $\vec{A}$.)\\
b) Express the direction cosines of $\vec{A}$ in terms of $a$, $b$, $c$; find
the direction cosines of the vector $-\vec{i} + 2\vec{j} + 2\vec{k}$.\\
c) Prove that three numbers $t$, $u$, $v$ are the direction cosines of a vector
in space if and only if they satisfy $t^{2} + u^{2} + v^{2} = 1$.

7. Prove using vector methods (without components) that the line segment
joining the midpoints of two sides of a triangle is parallel to the third side
and half its length. (Call the two sides $\vec{A}$ and $\vec{B}$.)

8. Prove using vector methods (without components) that the diagonals of a
parallelogram bisect each other. (One way: let $X$ and $Y$ be the midpoints of
the two diagonals; show $X$ = $Y$.)

\bigskip

Unit 2 Dot Product

1. Tell for what values of $c$ the vectors $c \vec{i} + 2 \vec{j} - \vec{k}$
and $\vec{i} - \vec{j} + 2 \vec{k}$ will\\
a) be orthogonal \hspace{10px} b) form an acute angle

2. Using vectors, find the angle between a longest diagonal $PQ$ of a cube,
and\\
a) a diagonal $PR$ of one of its faces; \hspace{10px} b) an edge $PS$ of the
cube.\\
(Choose a size and position for the cube that makes calculation easiest.)

3. Three points in space are $P:(a,1,-1)$, $Q:(0,1,1)$, $R:(a,-1,3)$. For what
value(s) of $a$ will $PQR$ be\\
a) a right angle \hspace{10px} b) an acute angle

4. Find the component of the force $\vec{F} = 2 \vec{i} - 2 \vec{j} + \vec{k}$
in\\
a) the direction $\frac{\vec{i} + \vec{j} - \vec{k}}{\sqrt{3}}$ \hspace{10px}
b) the direction of the vector $3 \vec{i} + 2 \vec{j} - 6 \vec{k}$.

5. Prove using vector methods (without components) that the diagonals of a
parallelogram have equal lengths if and only if it is a rectangle.

6. Prove using vector methods (without components) that the diagonals of a
parallelogram are perpendicular if and only if it is a rhombus, i.e., its four
sides are equal.

7. Prove using vector methods (without components) that an angle inscribed in
a semicircle is a right angle.

8. Prove the trigonometric formula: $\cos(\theta_{1} - \theta_{2}) =
\cos\theta_{1}\cos\theta_{2} + \sin\theta_{1}\sin\theta_{2}$.

\bigskip

Unit 3 Determinants

1. Calculate the value of the determinants\\
a) $\begin{vmatrix}
    1 & 4 \\
    2 & -1
\end{vmatrix}$\\
b) $\begin{vmatrix}
    3 & -4 \\
    -1 & -2
\end{vmatrix}$

2. Calculate $\begin{vmatrix}
    -1 & 0 & 4 \\
    1 & 2 & 2 \\
    3 & -2 & -1
\end{vmatrix}$ using the Laplace expansion by the cofactors of:\\
a) the first row \hspace{10px} b) the first column

3. Find the area of the plane triangle whose vertices lie at\\
a) $(0, 0), (1, 2), (1, -1)$ \hspace{10px} b) $(1, 2), (1, -1), (2, 3)$

4. a) Show that the value of a $2 \times 2$ determinants is unchanged if you add
to the second row a scalar multiple of the first row.\\
   b) Show that the value of a $2 \times 2$ determinants is unchanged if you add
to the second column a scalar multiple of the first column.

5. Use a Laplace expansion and Exercise 5a to show the value of a $3 \times 3$
determinants is unchanged if you add to the second row a scalar multiple of the
third row.

6. Let $(x_{1}, y_{1})$ and $(x_{2}, y_{2})$ both range over all unit vectors.
Find the maximum value of the function $f(x_{1}, x_{2}, y_{1}, y_{2}) =
\begin{vmatrix}
x_{1} & y_{1} \\
x_{2} & y_{2}   
\end{vmatrix}$.

\bigskip

Unit 4 Cross Product

1. Find $\vec{A} \times \vec{B}$ if\\
a) $\vec{A} = \vec{i} - 2 \vec{j} + \vec{k}$, $\vec{B} = 2 \vec{i} - \vec{j} -
\vec{k}$ \\
b) $\vec{A} = 2 \vec{i} - 3 \vec{k}$, $\vec{B} = \vec{i} + \vec{j} - \vec{k}$

2. Find the area of the triangle in space having its vertices at the points
\[ P:(2,0,1), Q:(3,1,0), R:(-1,1,-1).\]

3. Two vectors $\vec{i}'$ and $\vec{j}'$ of a right-handed coordinate system
are to have the directions respectively of the vectors $\vec{A} = 2 \vec{i} -
\vec{j}$ and $\vec{B} = \vec{i} + 2 \vec{j} + \vec{k}$. Find all three vectors
$\vec{i}'$, $\vec{j}'$, $\vec{k}'$.

4. Verify that the cross product $\times$ does not in general satisfy the
associative law, by showing that for the particular vectors $\vec{i}$,
$\vec{j}$, $\vec{k}$, we have $(\vec{i} \times \vec{j}) \times \vec{k} \neq
\vec{i} \times (\vec{j} \times \vec{k})$.

5. What can you conclude about $\vec{A}$ and $\vec{B}$\\
a) if $|\vec{A} \times \vec{B}| = |\vec{A}||\vec{B}|$;\\
b) if $|\vec{A} \times \vec{B}| = \vec{A} \cdot \vec{B}$.

6. Find the volume of the tetrahedron having vertices at the four points
\[ P:(1,0,1), Q:(-1,1,2), R:(0,0,2), S:(3,1,-1).\]

\begin{center}
{\rmfamily\bfseries\large Part II}
\end{center}

1. Find the dihedral angle between two faces of a regular tetrahedron.

2. a) Show that the 'polarization identity'
$\frac{1}{4}\left( |\vec{u} + \vec{v}|^{2} - |\vec{u} - \vec{v}|^{2} \right) =
\vec{u} \cdot \vec{v}$ holds for any two n-vectors $\vec{u}$ and $\vec{v}$. (Use
vector algebra, not components.) \\
b) Given two non-zero vectors $\vec{u}$ and $\vec{v}$, give the formula for the
unit vector which bisects the (smaller) angle between $\vec{u}$ and $\vec{v}$.
(Use the notation $\hat{\vec{u}}$ for the unit vector in the $\vec{u}$ -
direction.)

3. In this problem we examine tacking, which is the process sailboats use to
travel against the wind. Sails are a familar tool to harness the energy of the
wind for transportation over the sea. Early ships had large fixed sails which
would capture the wind blowing from behind to propel the ship forward. Even if
the wind is blowing behind at an (acute) angle the component of the wind vector
perpendicular to the sail will push on the sail and hence on the boat. However,
these early fixed sail ships had no way to go against the wind and had to rely
on oarsmen if the wind was blowing in the wrong direction.

A great advance that allowed boats to sail against the wind was the invention
of movable sails in combination with a rudder and a keel. By carefully
positioning the sail the boat can be made to sail into the wind - this process
is called \emph{tacking}.

As noted before, the component of the wind perpendicular to the sail pushes on
the sail and, through it, the boat. The keel only allows the boat to move along
its axis. (The rudder is used to turn the boat.) That is, for any force on the
boat, only the component along the boat's axis actually pushes the boat.

Described mathematically, the wind vector is first projected on the
perpendicular to the sail to get the direction of the force on the sail. This
resultant force is projected on the axis of the boat to find the direction the
boat is being pushed. By orienting the sail correctly this double projection
can result in a vector with a component pointing into the wind.

In the picture $\vec{w} = a \vec{i}$ is the wind direction. The line $l_{s}$
is perpendicular to the sail (with $0 \leq \alpha \leq \frac{\pi}{2}$). And the
line $l_{B}$ is along the boat's axis (with $0 \leq \beta \leq \frac{\pi}{2}$).

a) Let $\vec{w_{1}}$ be the projection of $\vec{w}$ onto the line $l_{s}$. Show
that $\vec{w_{1}}$ does not have a non-zero component in the direction opposite
$\vec{w}$. (It is sufficient to show the projections on the sketch.)

b) Find the projection of $\vec{w_{1}}$ onto $l_{B}$. (Give an explicit formula
in terms of $\alpha$ and $\beta$.) What is the condition on $\alpha$ and $\beta$
that this projection has a component in the $-\vec{i}$ direction? (For warmup
you might try the specific case $\alpha = \frac{\pi}{3} = \beta$.)

\end{document}