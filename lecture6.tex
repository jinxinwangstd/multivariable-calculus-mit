\documentclass{article}
\usepackage[utf8]{inputenc}
\usepackage{amsmath}
\usepackage{amssymb}
\usepackage{amsthm}
\usepackage{tikz}
\setlength{\parindent}{0pt}

\newtheorem*{theorem}{Theorem}
\newtheorem*{definition}{Definition}
\newtheorem*{lemma}{Lemma}
\newtheorem*{corollary}{Corollary}
\newtheorem{example}{Example}

\title{Lecture 6: Kepler's Second Law}
\author{}
\date{}

\begin{document}
    
\maketitle

\section{Describe Motion with Vectors}

With parametric equations, we have a way not only to describe a motion, but also 
to analyze the motion with more details.

\subsection{Position vectors}

Position vectors are used to describe the position of a point in a motion.
\[
  \vec{r}(t) = <x(t), y(t), z(t)>
\]

Take the cycloid of the wheel of radius 1 rolling at unit speed as an example:
\[
  \vec{r}(t) = <t - \sin t, 1 - \cos t>
\]

\subsection{Velocity vectors}

Velocity vectors are used to describe how fast and in what direction a point 
moves in a motion.
\begin{equation*}
\begin{split}
  \vec{v} &= \frac{d\vec{r}}{dt} \\
          &= <\frac{dx}{dt}, \frac{dy}{dt}, \frac{dz}{dt}> \\
\end{split}
\end{equation*}
Speed is a scalar, and used to describe how fast a point moves in a motion.
\begin{equation*}
  speed = |\vec{v}|
\end{equation*}

Take the cycloid of the wheel of radius 1 rolling at unit speed as an example:
\begin{gather*}
  \begin{split}
    \vec{v} &= \frac{d\vec{r}}{dt} \\
            &= <\frac{d(t - \sin t)}{dt}, \frac{d(1 - \cos t)}{dt}> \\
            &= <1 - \cos t, \sin t> \\
  \end{split} \\
  \begin{split}
    |\vec{v}| &= \sqrt{(1 - \cos t)^2 + \sin^2 t} \\
              &= \sqrt{1 - 2\cos t + \cos^2 t + \sin^2 t} \\
              &= \sqrt{2 - 2\cos t} \\
  \end{split} \\
\end{gather*}
At $t = 0$, the velocity vector $\vec{v} = <0, 0>$, so the speed at that time is
0.

\subsection{Acceleration vectors}

Acceleration vectors are used to describe how velocity changes in a motion.
\begin{equation*}
  \vec{a} = \frac{d\vec{v}}{dt}
\end{equation*}

Take the cycloid of the wheel of radius 1 rolling at unit speed as an example:
\begin{equation*}
\begin{split}
  \vec{a} &= \frac{d\vec{v}}{dt} \\
          &= <\frac{d(1 - \cos t)}{dt}, \frac{d(\sin t)}{dt}> \\
          &= <\sin t, \cos t> \\
\end{split}
\end{equation*}
At $t = 0$, the acceleration vector $\vec{a} = <0, 1>$, which means the point at 
that time has an acceleration in the positive direction of the $y$ axis. 
Combining with the previous conclusion that the velocity at this time is 0, it 
explains why the trajectory between two arches has a vertical tangent line.

\subsection{Arc length}

Arc length, usually denoted by $s$, is the distance traveled along the 
trajectory of a motion. According to this definition,
\begin{equation*}
  \frac{ds}{dt} = speed = |\vec{v}|
\end{equation*}

Take the cycloid of the wheel of radius 1 rolling at unit speed as an example. 
The length of an arch of the cycloid is
\begin{equation*}
  \int_{0}^{2\pi} \sqrt{2 - 2\cos t} dt
\end{equation*}

\subsection{Unit tangent vector}

Unit tangent vector, denoted by $\hat{T}$, is an unit vector in the direction of 
the tangent line at a position in a motion.

Since $\vec{v}$ is a tangent vector,
\begin{equation*}
  \hat{T} = \frac{\vec{v}}{|\vec{v}|}
\end{equation*}

Another definition for the unit tangent vector:
\begin{gather*}
  \begin{split}
    \vec{v} &= \frac{d\vec{r}}{dt} \\
            &= \frac{d\vec{r}}{ds} \cdot \frac{ds}{dt} \\
            &= \frac{d\vec{r}}{ds} \cdot |\vec{v}| \\
  \end{split} \\
  \begin{split}
    \frac{d\vec{r}}{ds} &= \frac{\vec{v}}{|\vec{v}|} \\
                        &= \hat{T} \\
  \end{split} \\
  \vec{v} = \hat{T} \cdot \frac{ds}{dt} \\
\end{gather*}

Therefore, we can see the velocity vector can be interpreted as:
\begin{enumerate}
  \item direction: the unit tangent vector $\hat{T}$.
  \item magnitude: the speed $\frac{ds}{dt}$.
\end{enumerate}

The geometric interpretation of $\frac{d\vec{r}}{ds} = \hat{T}$: during the time 
period of $\Delta t$, 
\begin{gather*}
  \Delta \vec{r} = \vec{r}(t + \Delta t) - \vec{r}(t) \\
  \Delta \vec{r} \approx \hat{T} \cdot \Delta s \\
  \frac{\Delta \vec{r}}{\Delta t} \approx \hat{T} \cdot \frac{\Delta s}{\Delta t} \\
\end{gather*}
With $\Delta t$ approaches 0, the above formula becomes
\begin{gather*}
  \frac{d\vec{r}}{dt} = \hat{T} \cdot \frac{ds}{dt} \\
  \frac{d\vec{r}}{ds} = \hat{T} \\
\end{gather*}

\section{Example of Describing Motion with Vectors: Kepler's Second Law}

\subsection{Kepler's Second Law}

Kepler's Second Law: Motion of planets is in a plane, and the area swept out by 
the line from sun to planet at a constant rate.

Newton later explain this law with gravitational attraction.

\subsection{Translation with Vector Notations}

\end{document}