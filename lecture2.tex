\documentclass{article}
\usepackage[utf8]{inputenc}
\usepackage{amsmath}
\usepackage{amssymb}
\usepackage{amsthm}
\usepackage{tikz}
\setlength{\parindent}{0pt}

\newtheorem*{theorem}{Theorem}
\newtheorem*{definition}{Definition}
\newtheorem*{lemma}{Lemma}
\newtheorem*{corollary}{Corollary}
\newtheorem{example}{Example}

\title{Lecture 2: Determinants, Cross Product}
\author{}
\date{}

\begin{document}
    
\maketitle

\section{Determinants}

\subsection{Determinants in 2D plane}

With two vectors, how do we calculate the area of the triangle enclosed by
these two vectors?

\subsubsection{Calculate the area of the triangle formed by two vectors with an acute angle}

\begin{tikzpicture}
  [help lines/.style={dashed}]
  \draw[->] (0, 0, 0) -- (3, 0, 0) node[anchor=south] {$\vec{A}$};
  \draw[->] (0, 0, 0) -- (2, 2, 0) node[anchor=south] {$\vec{B}$};
  \draw[help lines] (2, 2, 0) -- (2, 0, 0);
  \draw[->, help lines] (0, 0, 0) -- (0, 3, 0) node[anchor=east] {$\vec{A'}$};
  \draw (2, 0.2, 0) -- (1.8, 0.2, 0) -- (1.8, 0, 0);
  \draw (0.2, 0) arc [start angle=0, end angle=45, radius=0.2] 
    node[anchor=west] {$\theta$};
  \draw (0, 0.2) arc [start angle=90, end angle=45, radius=0.2] 
    node[anchor=south] {$\theta'$};
\end{tikzpicture}

Apparently
\[
  Area = \frac{1}{2}|\vec{A}||\vec{B}|\sin\theta
\]

But it is hard to calculate $\sin\theta$, whereas it is easy to calculate
$\cos\theta$ with dot products. Therefore, we can rotate $\vec{A}$ towards
$\vec{B}$ by $90^{\circ}$ to get $\vec{A'}$ and hence
\begin{gather*}
  \sin\theta = \cos(90^{\circ} - \theta) = \cos(\theta') \\
  \begin{split}
  Area  &= \frac{1}{2} |\vec{A}| |\vec{B}| \sin\theta \\
        &= \frac{1}{2} |\vec{A'}| |\vec{B}| \cos\theta' \\
        &= \frac{1}{2} \vec{A'} \cdot \vec{B} \\
  \end{split} \\
\end{gather*}

Suppose that $\vec{A} = <a_1, a_2>$, and $\vec{B} = <b_1, b_2>$, then
$\vec{A'} = <-a_2, a_1>$. Therefore,

\[
  \begin{split}
  \vec{A'} \cdot \vec{B}  &= <-a_2, a_1> \cdot <b_1, b_2> \\
                          &= a_1 b_2 - a_2 b_1 \\
                          &= det(\vec{A}, \vec{B}) \\
                          &=  \begin{vmatrix}
                                a_1 & a_2 \\
                                b_1 & b_2
                              \end{vmatrix}
  \end{split}
\]

Notice that it is also possible that $\vec{A'} = <a_2, -a_1>$, depending on 
the relative positions of $\vec{A}$ and $\vec{B}$, so the result could be the
negative value of the area.

\subsubsection{Calculate the area of the triangle formed by two vectors with an obtuse angle}

\begin{tikzpicture}
  [help lines/.style={dashed}]
  \draw[->] (0, 0, 0) -- (3, 0, 0) node[anchor=south] {$\vec{A}$};
  \draw[->] (0, 0, 0) -- (-2, 2, 0) node[anchor=south] {$\vec{B}$};
  \draw[help lines] (0, 0, 0) -- (-3, 0, 0);
  \draw[help lines] (-2, 2, 0) -- (-2, 0, 0);
  \draw (-2, 0.2, 0) -- (-1.8, 0.2, 0) -- (-1.8, 0, 0);
  \draw (0.2, 0) arc [start angle=0, end angle=135, radius=0.2] 
    node[anchor=south west] {$\theta$};
  \draw[->, help lines] (0, 0, 0) -- (0, 3, 0) node[anchor=east] {$\vec{A'}$};
  \draw (0, 0.4) arc [start angle=90, end angle=135, radius=0.4]
    node[anchor=south] {$\theta'$};
\end{tikzpicture}

\[
  \begin{split}
  Area  &= \frac{1}{2} |\vec{A}| |\vec{B}| \sin(\pi - \theta) \\
        &= \frac{1}{2} |\vec{A}| |\vec{B}| \sin\theta \\
        &= \frac{1}{2} |\vec{A}| |\vec{B}| \sin(\pi + \theta') \\
        &= \frac{1}{2} |\vec{A'}| |\vec{B}| \cos\theta' \\
        &= \frac{1}{2} \vec{A'} \cdot \vec{B}
  \end{split}
\]

Similarly, suppose that $\vec{A} = <a_1, a_2>$, and $\vec{B} = <b_1, b_2>$, then
$\vec{A'} = <-a_2, a_1>$. Therefore,

\[
  \begin{split}
  \vec{A'} \cdot \vec{B}  &= <-a_2, a_1> \cdot <b_1, b_2> \\
                          &= a_1 b_2 - a_2 b_1 \\
                          &= det(\vec{A}, \vec{B}) \\
                          &=  \begin{vmatrix}
                                a_1 & a_2 \\
                                b_1 & b_2
                              \end{vmatrix}
  \end{split}
\]


\subsubsection{Geometric interpretation of determinants in 2D plane}

$det(\vec{A}, \vec{B})$ measures the area of the parallelogram formed by
$\vec{A}$ and $\vec{B}$, but the value can be positive or negative of it.

\subsection{Determinants in 3D space}

Suppose that $\vec{A} = <a_1, a_2, a_3>$, $\vec{B} = <b_1, b_2, b_3>$, and 
$\vec{C} = <c_1, c_2, c_3>$, then
\[
  \begin{split}
  det(\vec{A}, \vec{B}, \vec{C}) &= \begin{vmatrix}
                                    a_1 & a_2 & a_3 \\
                                    b_1 & b_2 & b_3 \\
                                    c_1 & c_2 & c_3
                                    \end{vmatrix} \\
                                 &= a_1 \cdot \begin{vmatrix}
                                              b_2 & b_3 \\
                                              c_2 & c_3
                                              \end{vmatrix} - 
                                    a_2 \cdot \begin{vmatrix}
                                              b_1 & b_3 \\
                                              c_1 & c_3
                                              \end{vmatrix} +
                                    a_3 \cdot \begin{vmatrix}
                                              b_1 & b_2 \\
                                              c_1 & c_2
                                              \end{vmatrix}
  \end{split}
\]

We can expand the $3 \times 3$ determinants to expressions of $2 \times 2$
determinants according to the first row to calculate their values. The rules
are as follows: 

\subsubsection{Geometric interpretation of determinants in 3D space}

\begin{theorem}
Geometrically, $det(\vec{A}, \vec{B}, \vec{C}) = \pm$ volume of the
parallelepiped formed by $\vec{A}$, $\vec{B}$, and $\vec{C}$.
\end{theorem}

\section{Cross Product}

\begin{definition}
  The cross product of two vectors $\vec{A} = <a_1, a_2, a_3>$ and
  $\vec{B} = <b_1, b_2, b_3>$ is a vector defined as:
  \[
    \begin{split}
    \vec{A} \times \vec{B} &= \begin{vmatrix}
                                \vec{i} & \vec{j} & \vec{k} \\
                                a_1 & a_2 & a_3 \\
                                b_1 & b_2 & b_3
                              \end{vmatrix} \\
                           &= \begin{vmatrix}
                                a_2 & a_3 \\
                                b_2 & b_3 
                              \end{vmatrix} \vec{i} -
                              \begin{vmatrix}
                                a_1 & a_3 \\
                                b_1 & b_3
                              \end{vmatrix} \vec{j} +
                              \begin{vmatrix}
                                a_1 & a_2 \\
                                b_1 & b_2
                              \end{vmatrix} \vec{k}
    \end{split}
  \]
\end{definition}

Notice that the $3 \times 3$ determinants in the above definition is not a
legit determinant expression. It is just a symbolic notation helping memorize
the formula.

\begin{theorem}
$|\vec{A} \times \vec{B}| = $ the area of the parallelogram formed by $\vec{A}$
and $\vec{B}$ in the 3D space.
\end{theorem}

\begin{theorem}
The direction of $\vec{A} \times \vec{B}$, or $dir(\vec{A} \times \vec{B})$
is perpendicular to the parallelogram formed by $\vec{A}$ and $\vec{B}$ in
the 3D space.
\end{theorem}

There are two directions which is perpendicular to the parallelogram formed by
$\vec{A}$ and $\vec{B}$, which one is the direction of $\vec{A} \times \vec{B}$?

Right-hand rule: Use your right hand to rotate from $\vec{A}$ to $\vec{B}$ then
your right thumb is pointing to the direction of $\vec{A} \times \vec{B}$.

\begin{example}
  
\end{example}

\subsection{Properties of cross product}

\subsubsection{Commutative property}
Cross product does not satisfy the commutative property.

\begin{gather*}
  \vec{A} \times \vec{B} \neq \vec{B} \times \vec{A} \\
  \vec{A} \times \vec{B} = - \vec{B} \times \vec{A} \\
\end{gather*}

In particular:
\[
  \vec{A} \times \vec{A} = \vec{0}
\]

\subsubsection{Associative property}
Cross product does not satisfy the associative property.

Counter example: Suppose $\vec{A} = <1, 0, 1>$, $\vec{B} = <1, 1, 0>$,
$\vec{C} = <0, 1, 1>$, then we can get
\begin{gather*}
  (\vec{A} \times \vec{B}) \times \vec{C} = <0, 1, -1> \\
  \vec{A} \times (\vec{B} \times \vec{C}) = <1, 0, -1> \\
\end{gather*}

\subsection{Application of cross product}

\begin{example}
Given the coordinates of three points that are not in the same line in 3D space,
find the equation of the plane determined by these three points.

Solution:

Finding the equation of a plane is equivalent to finding a method of
determining whether a random point is in the plane or not.

Suppose the three known points are $P_1$, $P_2$, and $P_3$, and a random point
is $P$.

Method 1: $P$ is in the plane if and only if the parallelepiped formed by
$\vec{P_1P_2}$, $\vec{P_1P_3}$, and $\vec{P_1P}$ has a volume $0$. Therefore,
according to the geometric interpretation of determinants in 3D space, we can
derive that $P$ is in the plane if and only if
\[
  det(\vec{P_1P_2}, \vec{P_1P_3}, \vec{P_1P}) = 0
\]

Method 2: Using the normal vector. The normal vector of a plane refers to the
vector that are perpendicular to the plane. A plane can have infinite normal
vectors. We usually use $\vec{N}$ to represent a normal vector. 

The result of a cross product of two vectos is a normal vector of the
parallelogram formed by the two vectos. With that being said, $P$ is in the
plane if and only if $\vec{P_1P} \perp \vec{N}$, which is equivalent to
\begin{gather*}
  \vec{P_1P} \cdot \vec{N} = 0 \\
  \vec{P_1P} \cdot (\vec{P_1P_2} \times \vec{P_1P_3}) = 0 \\
\end{gather*}

Therefore, the two equations are actually equivalent:
\[
  det(\vec{P_1P_2}, \vec{P_1P_3}, \vec{P_1P}) = \vec{P_1P} \cdot (\vec{P_1P_2} \times \vec{P_1P_3}) = 0
\]

\end{example}

\end{document}