\documentclass{article}
\usepackage[utf8]{inputenc}
\usepackage{amsmath}
\usepackage{amssymb}
\usepackage{amsthm}
\usepackage{tikz}
\setlength{\parindent}{0pt}

\newtheorem*{theorem}{Theorem}
\newtheorem*{definition}{Definition}
\newtheorem*{lemma}{Lemma}
\newtheorem*{corollary}{Corollary}
\newtheorem{example}{Example}
\newtheorem*{trick}{Trick}
\newtheorem*{question}{Question}

\title{Lecture 4: Square Systems}
\author{}
\date{}

\begin{document}
    
\maketitle

\section{Plane Equations}

An equation of the form 
$ax + by + cz = d \; \textnormal{(a, b, c, d are constants)}$ defines a plane.

\begin{example}
  Find the equation of the plane through the origin with a normal vector 
  $\vec{N} = <1, 5, 10>$.

  By thinking geometrically, a point $P$ is in the plane $\iff$ 
  $OP \perp \vec{N}$ $\iff$ $\vec{OP} \cdot \vec{N} = 0$. Therefore, the 
  equation of the described plane is
  \[
    <x, y, z> \cdot <1, 5, 10> = 0
  \]
  \[
    x + 5y + 10z = 0
  \]
\end{example}

\begin{example}
  Find the equation of the plane through the point $P_0$ (2, 1, -1) with a 
  normal vector $\vec{N} = <1, 5, 10>$.

  Similarly by thinking geometrically, a point $P$ (x, y, z) is in the plane 
  $\iff$ $\vec{P_0P} \perp \vec{N}$ $\iff$ $\vec{P_0P} \cdot \vec{N} = 0$. 
  Therefore, the equation of the described plane is 
  \[
    <x - 2, y - 1, z + 1> \cdot <1, 5, 10> = 0
  \]
  \[
    (x - 2) + 5(y - 1) + 10(z + 1) = 0
  \]
  \[
    x + 5y + 10z = -3
  \]
\end{example}

From the above example we can see that the coefficients of the equation of a 
plane are actually the components of one of its normal vectors.

The right-hand side constant in the equation of a plane is an indicator of the 
distance to its parallel plane through the origin. For example, we can derive 
that $x + 5y + 10z = 3$ and $x + 5y + 10z = -3$ are in the two sides of the 
plane $x + 5y + 10z = 0$ respectively, and $x + 5y + 10z = -1$ has a short 
distance to $x + 5y + 10z = 0$ than $x + 5y + 10z = -3$.

\begin{example}
  The vector $\vec{v} = <1, 2, -1>$ and the plane $x + y + 3z = 5$ are \underline{A}.

  A. parallel \\
  B. perpendicular \\
  C. neither

  Reasons: \\
  A normal vector of the plane $x + y + 3z = 5$ is $\vec{N} = <1, 1, 3>$. Then
  \[
    \begin{split}
    \vec{v} \cdot \vec{N} &= <1, 2, -1> \cdot <1, 1, 3> \\
                          &= 1 + 2 - 3 \\
                          &= 0
    \end{split}
  \]
  Hence $\vec{v} \perp \vec{N}$, from which we can derive that $\vec{v}$ is 
  perpendicular to the plane.
\end{example}

\section{Geometric Interpretation of Linear Systems}

\subsection{Linear systems in geometry}

A linear system describes the intersection of some objects or point sets in 
geometry.

Take a $3 \times 3$ linear system as an example:
\[
  \left\{ \begin{array}{ll}
  x + z = 1 \\
  x + y = 2 \\
  x + 2y + 3z = 3 \\
  \end{array} \right.
\]
The solution to this linear system is the set of points $(x, y, z)$ satisfying 
all three linear equations, where each of them defines a plane. Therefore, the 
points in the solution set are in all of three planes. Hence, the solution to 
the linear system describes the intersection of these three planes in geometry.

To find the solution to a linear system, algebraic methods are easier than 
geometric methods. Recall that a linear system can be expressed with matrix 
products as:
\[
  AX = B
\]
\[
  X = A^{-1}B
\]
$A^{-1}B$ is the unique solution to the linear system, which is the 
intersection point in geometry.

\subsection{Exceptions of unique solution in geometry}

However, there are exceptions to this algebraic method.
\begin{example}
  If the solution set to a $3 \times 3$ linear system is not a single point, it 
  could be \underline{A C E}.

  A. no solution \\
  B. two points \\
  C. a line \\
  D. a tetrahedron \\
  E. a plane \\
  F. I don't know

  Reasons:
  For A, the situation would be at least two of the three planes are parallel to 
  each other and not the same plane, or the intersection line of two planes is 
  parallel to and not contained in the third plane.

  For C, the situation would be the intersection of two planes, which is a line, 
  is contained in the third plane.

  For E, the situation would be that the three planes are the same.
\end{example}
The exceptions to a single solution to a linear system are described in the 
above example.

\subsection{Algebraic point of view on exceptions of unique solutions}

Recall that the formula of the unique solution to a linear system is
\[
  X = A^{-1}B
\]
It turns out this formula doesn't always hold. In those exception cases, 
$A^{-1}$ doesn't exist, i.e. $A$ is not invertible.

\[
  A^{-1} = \frac{1}{det(A)}adj(A)
\]
Since $adj(A)$ always exists, $A^{-1}$ exists $\iff$ $A$ is invertible $\iff$ 
$det(A) \neq 0$.

\bigskip

Further discussion on different cases:

1. Homogeneous cases: $AX = 0$

For Homogeneous cases, there is always a trivial solution $X = 0$ since all 
planes pass through the origin.

If $det(A) \neq 0$, then $A$ is invertible, then the linear system has a unique 
solution, which must be $0$ since $0$ is always a solution to the linear 
system. We can also derive the conclusion in algebra:
\[
  X = A^{-1}0 = 0
\]

If $det(A) = 0$, then since each row in $A$ is a normal vector of the 
corresponding plane, $det(\vec{N_1}, \vec{N_2}, \vec{N_3}) = 0$, which means 
that $\vec{N_1}$, $\vec{N_2}$, $\vec{N_3}$ are coplanar.

  If $\vec{N_1}$, $\vec{N_2}$, $\vec{N_3}$ are the same, then the planes in the 
  linear system are the same, so the solution set is the plane.

  If $\vec{N_1}$, $\vec{N_2}$, $\vec{N_3}$ are different, then the line passing 
  through the origin and perpendicular to the plane containing $\vec{N_1}$, 
  $\vec{N_2}$, and $\vec{N_3}$ must be in all three planes in the linear 
  system. Therefore, the solution is $\vec{N_1} \times \vec{N_2}$, or 
  $\vec{N_2} \times \vec{N_3}$, or $\vec{N_1} \times \vec{N_3}$.

2. General cases: $AX = B$

If $det(A) \neq 0$, there is a unique solution to the linear system.

If $det(A) = 0$, there is either none or infinite solution to the linear system.

\end{document}