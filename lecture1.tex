\documentclass{article}
\usepackage[utf8]{inputenc}
\usepackage{amsmath}
\usepackage{amssymb}
\usepackage{graphicx}
\setlength{\parindent}{0pt}

\title{Lecture 1: Dot Product}
\author{}
\date{}

\begin{document}

\maketitle

\section{Vectors}

Vectors: quantities that have both direction and magnitude/length.

Scalars: quantities that only have magnitude.

Identities of vectors: as long as the directions and magnitudes of two vectors 
are the same, we say these two vectors are equal. The equality of vectors has 
nothing to do with their start points and end points.

\bigskip

Notations for Vectors
\begin{itemize}
    \item $\vec{A}$: a vector with the name $A$.
    \item $\vec{AB}$: a vector from the point $A$ to the point $B$.
    \item $|\vec{A}|$: the magnitude/length of the vector $\vec{A}$.
    \item $dir(\vec{A})$: the direction of the vector $\vec{A}$.
\end{itemize}

\bigskip

\subsection{Vectors From A Geometric Perspective}

Geometrically, we typically express a vector as an arrow with a direction and 
a length.

\smallskip

Basic vector operations
\begin{itemize}
    \item Addition of vectors: The result of $\vec{A} + \vec{B}$ is defined as
    the vector from the start of $\vec{A}$ to the end of $\vec{B}$ after
    moving the start of $\vec{B}$ to the end of $\vec{A}$ parallel.

    \item Negative vectors: The result of $-\vec{A}$ is defined as the vector
    with the same length and reverse direction of $\vec{A}$.

    \item Subtraction of vectors: The result of $\vec{A} - \vec{B}$ is equal to
    $\vec{A} + (-\vec{B})$, which is the vector from the end of $\vec{B}$ to
    the end of $\vec{A}$.

    \item Multiplication by scalars: The result of $c \times \vec{A}$ is
    defined as the vector with the same direction and $c$ times of the length
    of $\vec{A}$. 
\end{itemize}

\bigskip

\subsection{Vectors From An Analytical Perspective}

To compute things with vectors, we need to put them into coordinate systems.
With the addition operation we defined for vectors, we can decompose a vector
to a unique set of component vectors along coordinate axes.

For the convenience of expression, we define a unit vector for each coordinate
axis, which has length $1$ and same direction as the corresponding axis. By
convention, the unit vector along the $x$ axis is $\vec{i}$, the unit vector
along the $y$ axis is $\vec{j}$, and the unit vector along the $z$ axis
$\vec{k}$. In this way, every vector in the three dimension space can be
expressed as

\[ \vec{A} = a_{1} * i + a_{2} * j + a_{3} * k \]

Or more generally

\[ \vec{A} = <a_{1}, a_{2}, a_{3}> \]

This kind of expression can be generalized into $N$ dimensional space as:

\[ \vec{A} = <a_{1}, a_{2}, a_{3}, ..., a_{n-1}, a_{n}> \]

\bigskip

Basic vector operations
\begin{itemize}
    \item Addition of vectors: Since the addition of vectors satisfies the
    commutative and associative laws, $\vec{A} + \vec{B}$ is equal to the sum
    of the addition result of their corresponding component vectors. Hence
    \[ \vec{A} + \vec{B} = <a_{1} + b_{1}, a_{2} + b_{2}, ..., a_{n} + b_{n}> \]

    \item Negative vectors: It is obvious that
    \[ -\vec{A} = <-a_{1}, -a_{2}, ..., -a_{n}> \]

    \item Subtraction of vectors: The result of $\vec{A} - \vec{B}$ is equal to
    $\vec{A} + (-\vec{B})$. Hence
    \[ \vec{A} - \vec{B} = <a_{1} - b_{1}, a_{2} - b_{2}, ..., a_{n} - b_{n}> \]

    \item Multiplication by scalars: It is obvious that
    \[ c \times \vec{A} = <c \times a_{1}, c \times a_{2}, ..., c \times a_{n} \]
\end{itemize}

Example 1. What is the length of the vector $\vec{A} = <3, 2, 1>$?

We can define an intermediate vector $\vec{A_{1}} = 3\vec{i} + 2\vec{j}$.
According to Pythagorean Theorem, $|\vec{A_{1}}| = \sqrt{3^2 + 2^2} = \sqrt{13}$.
Then $\vec{A} = \vec{A_{1}} + \vec{k}$, and $\vec{A_{1}}$ is perpendicular to
$\vec{k}$. Hence $|\vec{A}| = \sqrt{|\vec{A_{1}}|^2 + |\vec{k}|^2} =
\sqrt{3^2 + 2^2 + 1^2} = \sqrt{14}$.

In general,
\[ |\vec{A}| = \sqrt{a_{1}^{2} + a_{2}^{2} + ... + a_{n}^{2}} \]
This formula holds because of the orthogonality of the axes in coordinate
systems.

\section{Dot Product}

Definition: $\vec{A} \cdot \vec{B} = \sum a_{i}b_{i} = a_{1}b_{1} + a_{2}b_{2} +
... + a_{n}b_{n}$

Geometric Interpretation: Suppose the angle between $\vec{A}$ and $\vec{B}$ is
$\theta$, then $\vec{A} \cdot \vec{B} = |\vec{A}| \cdot |\vec{B}| \cdot \cos\theta$.

Proofs:\\
Firstly let's consider the dot product of a vector and itself. According to the
definition of dot product:
\[ \vec{A} \cdot \vec{A} = \sum_{i=1}^{n}a_{i}^2 = |\vec{A}|^2 \]
Therefore the above statement holds in this case.

Then let's consider the dot product of two different vectors. One important
insight about the definition of dot product is that it satisfies the commutative
and distributive law. Let $\vec{C}$ denotes the result vector of
$\vec{A} - \vec{B}$. Then
\[ \vec{C} \cdot \vec{C} = (\vec{A} - \vec{B}) \cdot (\vec{A} - \vec{B}) \]

\[ |\vec{C}|^2 = \vec{A} \cdot \vec{A} + \vec{B} \cdot \vec{B} - 2 \vec{A} \cdot \vec{B} \]

\[ |\vec{C}|^2 = |\vec{A}|^2 + |\vec{B}|^2 - 2 \vec{A} \cdot \vec{B} \]

According to Law of Cosine, we have
\[ |\vec{C}|^2 = |\vec{A}|^2 + |\vec{B}|^2 - 2|\vec{A}||\vec{B}|\cos\theta \]

Therefore
\[ \vec{A} \cdot \vec{B} = |\vec{A}| \cdot |\vec{B}| \cdot \cos\theta \]

\bigskip

Property of the sign of dot product:
\begin{itemize}
    \item $\vec{A} \cdot \vec{B} > 0$ if and only if $\theta < 90^{\circ}$.
    \item $\vec{A} \cdot \vec{B} = 0$ if and only if $\theta = 90^{\circ}$.
    \item $\vec{A} \cdot \vec{B} < 0$ if and only if $\theta > 90^{\circ}$.
\end{itemize}

\bigskip
Application of Dot Product

1. Computing lengths and angles (especially angles).
    
Example 2. Find the angle between $PQ$ and $PR$.
% TODO(jinxinwang): add the diagram of this example.

Solution:
\[ \vec{PQ} = <-1, 1, 0> \]
\[ \vec{PR} = < -1, 0, 2> \]
Then
\[ \begin{split}
    \cos(\angle QPR) &= \frac{\vec{PQ} \cdot \vec{PR}}{|\vec{PQ}| \cdot
                            |\vec{PR}|} \\
                     &= \frac{-1 \cdot (-1) + 1 \cdot 0 + 0 \cdot 2}
                            {\sqrt{(-1)^{2} + 1^{2} + 0^{2}} \cdot 
                            \sqrt{(-1)^{2} + 0^{2} + 2^{2}}} \\
                     &= \frac{1}{\sqrt{10}}
\end{split} \]

2. Detect orthogonality.

Example 3. The set of points where $x + 2y + 3z = 0$ describes \\
A. The empty set \\
B. A single point \\
C. A line \\
D. A plane \\
E. A sphere \\
F. None of the above

Solution:\\
The set of points where $x + 2y + 3z = 0$ describes a plane. We can find three
points which satisfy the equation and are not on a same line together, in order
to exclude the option A, B, and C.
\[ P_1 = (0, 0, 0) \]
\[ P_2 = (1, 1, -1) \]
\[ P_3 = (5, -1, -1) \]
If a set of points describes a plane, then we can find a vector which is
perpendicular to any line formed by two points on the plane. That vector is
called normal. The normal to the surface described by this equation is
$<1, 2, 3>$. Suppose we have two random points on the surface described by the
above equation: $P_{1} = (x_{1}, y_{1}, z_{1})$ and $P_{2} = (x_{2}, y_{2},
z_{2})$. Then we have
\[ x_{1} + 2y_{1} + 3z_{1} = 0 \]
\[ x_{2} + 2y_{2} + 3z_{2} = 0 \]
Therefore
\[ (x_{1} - x_{2}) + 2(y_{1} - y_{2}) + 3(z_{1} - z_{2}) = 0 \]
which proves that the vector $<1, 2, 3>$ is perpendicular to the vector
$\vec{P_{1}P_{2}}$. Since $P_{1}$ and $P_{2}$ are random points, we prove that
the vector $<1, 2, 3>$ is perpendicular to any vectors formed by two points on
that surface. Therefore, that surface is a plane.

\end{document}