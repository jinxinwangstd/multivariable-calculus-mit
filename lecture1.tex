\documentclass{article}
\usepackage[utf8]{inputenc}
\usepackage{amsmath}
\usepackage{amssymb}
\setlength{\parindent}{0pt}

\title{Lecture 1: Dot Product}
\date{}

\begin{document}

\maketitle

\section{Vectors}

Vectors: quantities that have both direction and magnitude/length. \\
Scalars: quantities that only have magnitude.

Identities of vectors: as long as the directions and magnitudes of two vectors are the same, we say these two vectors are equal. The equality of vectors has nothing to do with their start points and end points.

Common Operations of Vectors From a Geometric Perspective

Addition of Vectors: Suppose there are two vectors $\vec{A}$ and $\vec{B}$, according to the identities of vectors, we can construct a vector $\vec{B}^{'}$ equal to $\vec{B}$ at the end point of $\vec{A}$, then the result of $\vec{A} + \vec{B}$ is the vector from the start point of $\vec{A}$ to the end point of $\vec{B}^{'}$, i.e. moving

To compute things with vectors, we need to put them into coordinate systems. We can express a vector with its components along coordinate axes. 


\end{document}