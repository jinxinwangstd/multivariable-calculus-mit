\documentclass{article}
\usepackage[utf8]{inputenc}
\usepackage{amsmath}
\usepackage{amssymb}
\usepackage{amsthm}
\usepackage{tikz}
\setlength{\parindent}{0pt}

\newtheorem*{theorem}{Theorem}
\newtheorem*{definition}{Definition}
\newtheorem*{lemma}{Lemma}
\newtheorem*{corollary}{Corollary}
\newtheorem{example}{Example}
\newtheorem{trick}{Trick}
\newtheorem{question}{Question}

\title{Problem Set 3}
\author{}
\date{}

\begin{document}

\begin{center}
{\rmfamily\bfseries\Large 18.02 EXERCISES}

\vspace{25px}

{\rmfamily\bfseries\LARGE Problem Set 3: Parametric Equations for Curves}
\end{center}

\begin{center}
\section*{Part I}
\end{center}

\subsection*{Unit 1E Equations of Lines and Planes}

\bigskip

4. Where does the line through $(0, 1, 2)$ and $(2, 0, 3)$ intersect the plane 
$x + 4y + z = 4$?

Solution:

A vector along the line is $<2, -1, 1>$. Hence, the parametric equation of the 
line is
\begin{equation*}
  \begin{cases}
    x = x(t) = 0 + 2t = 2t \\
    y = y(t) = 1 - t \\
    z = z(t) = 2 + t \\
  \end{cases}
\end{equation*}

For the intersection point of the line and the plane, it satisfies
\begin{gather*}
  x(t) + 4y(t) + z(t) = 4 \\
  2t + 4(1 - t) + 2 + t = 4 \\
  -t = -2 \\
  t = 2 \\
\end{gather*}

Therefore, the coordinates of the intersection point is $(x(2), y(2), z(2))$, 
which is $(4, -1, 4)$.

\bigskip

7. Formulate a general method for finding the distance between two skew (i.e., 
non-intersecting) lines in space, and carry it out for two non-intersecting 
lines lying along the diagonals of two adjacent faces of the unit cube (place it 
in the first octant, with one vertex at the origin).

Solution:

7-1. Formulate a general method for finding the distance between two parallel 
planes in space:

Suppose that the equations of two parallel planes are respectively
\begin{gather*}
  n_1x + n_2y + n_3z = c_1 \\
  n_1x + n_2y + n_3z = c_2 \\
\end{gather*}
Then the vector $\vec{n} = <n_1, n_2, n_3>$ is one of the normal vectors of the 
two planes. Therefore, we can construct the parametric equation of a line 
perpendicular to the two planes as
\begin{equation*}
  \begin{cases}
    x(t) = n_1t \\
    y(t) = n_2t \\
    z(t) = n_3t \\
  \end{cases}
\end{equation*}
Therefore, the two intersection points the line has with the two planes 
respectively satisfy
\begin{gather*}
  n_1x(t_1) + n_2y(t_1) + n_3z(t_1) = c_1 \\
  n_1^2t_1 + n_2^2t_1 + n_3^2t_1 = c_1 \\
  t_1 = \frac{c_1}{n_1^2 + n_2^2 + n_3^2} \\
  n_1x(t_2) + n_2y(t_2) + n_3z(t_2) = c_2 \\
  n_1^2t_2 + n_2^2t_2 + n_3^2t_2 = c_1 \\
  t_2 = \frac{c_2}{n_1^2 + n_2^2 + n_3^2} \\
\end{gather*}
The distance between two points in the line can be calculated as 
\begin{equation*}
  \begin{split}
    d &= \sqrt{(x(t_1) - x(t_2))^2 + (y(t_1) - y(t_2))^2 + (z(t_1) - z(t_2))^2} \\
      &= \sqrt{(n_1^2 + n_2^2 + n_3^2)(t_1 - t_2)^2} \\
      &= \sqrt{(n_1^2 + n_2^2 + n_3^2)(\frac{c_1 - c_2}{n_1^2 + n_2^2 + n_3^2})^2} \\
      &= \frac{|c_1 - c_2|}{\sqrt{n_1^2 + n_2^2 + n_3^2}} \\
      &= \frac{|c_1 - c_2|}{|\vec{n}|} \\
  \end{split}
\end{equation*}

7-2. Formulate a general method for finding the distance between two 
non-parallel lines in space:

Suppose that there are two non-parallel lines in space named $l_a$ and $l_b$. 
Their normal vectors are respectively 
$\vec{n_a} = \langle n_{ax}, n_{ay}, n_{az} \rangle$ and 
$\vec{n_b} = \langle n_{bx}, n_{by}, n_{bz} \rangle$, and their parametric 
equations are respectively:
\begin{gather*}
  \begin{cases}
    x_a(t) = x_{a0} + n_{ax}t \\
    y_a(t) = y_{a0} + n_{ay}t \\
    z_a(t) = z_{a0} + n_{az}t \\
  \end{cases} \\
  \begin{cases}
    x_b(t) = x_{b0} + n_{bx}t \\
    y_b(t) = y_{b0} + n_{by}t \\
    z_b(t) = z_{b0} + n_{bz}t \\
  \end{cases} \\
\end{gather*}

Then we can find the vector $\vec{n}$ which is perpendicular to both $l_a$ and 
$l_b$ by
\begin{equation*}
  \begin{split}
    \vec{n} &= \vec{n_a} \times \vec{n_b} \\
            &= \langle n_x, n_y, n_z \rangle \\
  \end{split}
\end{equation*}

Then the distance between $l_a$ and $l_b$ is equal to the distance between the 
two planes $P_a$ and $P_b$ with $\vec{n}$ as their normal vectors where $l_a$ 
and $l_b$ lie within respectively.

The equation of $P_a$ is
\begin{equation*}
  n_x x + n_y y + n_z z = n_x x_{a0} + n_y y_{a0} + n_z z_{a0}
\end{equation*}

The equation of $P_b$ is 
\begin{equation*}
  n_x x + n_y y + n_z z = n_x x_{b0} + n_y y_{b0} + n_z z_{b0}
\end{equation*}

According to the previous part, the distance between $P_a$ and $P_b$ is
\begin{equation*}
  \begin{split}
    d &= \frac{|(n_x x_{a0} + n_y y_{a0} + n_z z_{a0}) - (n_x x_{b0} + n_y y_{b0} + n_z z_{b0})|}{|\vec{n}|} \\
      &= \frac{|\vec{n} \cdot \langle (x_{a0} - x_{b0}), (y_{a0} - y_{b0}), (z_{a0} - z_{b0}) \rangle|}{|\vec{n}|} \\
      &= \frac{|\vec{n} \cdot \vec{A_0B_0}|}{|\vec{n}|} \\
      &= \frac{|(\vec{n_a} \times \vec{n_b}) \cdot \vec{A_0B_0}|}{|\vec{n_a} \times \vec{n_b}|} \\
  \end{split}
\end{equation*}

Therefore, the distance between $l_a$ and $l_b$ is
\begin{equation*}
  d = \frac{|(\vec{n_a} \times \vec{n_b}) \cdot \vec{A_0B_0}|}{|\vec{n_a} \times \vec{n_b}|}
\end{equation*}

7-3 Apply the above method to find the distance between two non-intersecting 
lines lying along the diagonals of two adjacent faces of the unit cube (place it 
in the first octant, with one vertex at the origin).

The two non-intersecting diagonals of two adjacent faces of the unit cube I 
choose are the diagonal between $(0, 0, 0)$ and $(1, 0, 1)$, and the diagonal 
between $(0, 1, 0)$ and $(0, 0, 1)$. Therefore, the parametric equations of 
these two lines along the described diagonals are respectively:
\begin{gather*}
  \begin{cases}
    x_a(t) = t \\
    y_a(t) = 0 \\
    z_a(t) = t \\
  \end{cases} \\
  \begin{cases}
    x_b(t) = 0 \\
    y_b(t) = 1 - t \\
    z_b(t) = t \\
  \end{cases} \\
\end{gather*}

Therefore,
\begin{gather*}
  \vec{n_a} = \langle 1, 0, 1 \rangle \\
  \vec{n_b} = \langle 0, -1, 1 \rangle \\
  A_0 = (0, 0, 0) \\
  B_0 = (0, 1, 0) \\
  \vec{A_0B_0} = \langle 0, 1, 0 \rangle \\
\end{gather*}

Therefore, the distance between the two non-intersecting diagonals is
\begin{equation*}
  \begin{split}
    d &= \frac{|(\vec{n_a} \times \vec{n_b}) \cdot \vec{A_0B_0}|}{|\vec{n_a} \times \vec{n_b}|} \\
      &= \frac{|\langle 1, -1, -1 \rangle \cdot \langle 0, 1, 0 \rangle|}{|\langle 1, -1, -1 \rangle|} \\
      &= \frac{\sqrt{3}}{3} \\
  \end{split}
\end{equation*}

Therefore, the distance between the two non-intersecting diagonals of two 
adjacent faces of the unit cube is $\frac{\sqrt{3}}{3}$.

\subsection*{Unit 1I Vector Functions and Parametric Equations}

\bigskip

1. The point $P$ moves with constant speed $v$ in the direction of the constant 
vector $a \vec{i} + b \vec{j}$. If at time $t = 0$ it is at $(x_0, y_0)$, what 
is its position vector function $\vec{r}(t)$?

Solution:

The direction of the constant vector $a \vec{i} + b \vec{j}$ is
\begin{equation*}
  dir(A) = \frac{a}{\sqrt{a^2 + b^2}} \vec{i} + \frac{b}{\sqrt{a^2 + b^2}} \vec{j}
\end{equation*}

Hence in a period of time $t$, the point $P$ moves in a distance of $vt$ in the 
direction 
$\langle \frac{a}{\sqrt{a^2 + b^2}}, \frac{b}{\sqrt{a^2 + b^2}} \rangle$. Hence 
along the $x$-axis it moves in a distance of $\frac{a}{\sqrt{a^2 + b^2}} vt$, 
and along the $y$-axis it moves in a distance of 
$\frac{b}{\sqrt{a^2 + b^2}} vt$.

Therefore, the position vector function of the point $P$ is
\begin{equation*}
  \vec{r}(t) = \langle x_0 + \frac{a}{\sqrt{a^2 + b^2}} vt, y_0 + \frac{b}{\sqrt{a^2 + b^2}} vt \rangle
\end{equation*}

\bigskip

3. Describe the motions given by each of the following position vector 
functions, as $t$ goes from $-\infty$ to $\infty$. In each case, give the 
$xy$-equation of the curve along which $P$ travels, and tell what part of the 
curve is actually traced out by $P$. \\
a) $\vec{r} = 2 \cos^2 t \vec{i} + \sin^2 t \vec{j}$ \\
b) $\vec{r} = \cos 2t \vec{i} + \cos t \vec{j}$ \\
c) $\vec{r} = (t^2 + 1) \vec{i} + t^3 \vec{j}$ \\
d) $\vec{r} = \tan t \vec{i} + \sec t \vec{j}$

Solution:

a) According to the position vector function,
\begin{gather*}
  x(t) = 2 \cos^2 t \\
  y(t) = \sin^2 t \\
\end{gather*}

Therefore, we can find the relation between $x$ and $y$:
\begin{gather*}
  \frac{x}{2} + y = 1 \\
  y = -\frac{1}{2}x + 1 \\
\end{gather*}
which is a line in the $xy$-plane.

Since the ranges of $x$ and $y$ are respectively
\begin{gather*}
  x \in [0, 2] \\
  y \in [0, 1] \\
\end{gather*}
the part of the curve that is actually traced out by $P$ is the line segment 
from $(0, 0)$ to $(2, 1)$.

b) According to the position vector function,
\begin{gather*}
  x(t) = \cos 2t \\
  y(t) = \cos t \\
\end{gather*}

Therefore, we can find the relation between $x$ and $y$:
\begin{gather*}
  \cos 2t = 2 \cos^2 t - 1 \\
  x = 2 y^2 - 1 \\
\end{gather*}
which is a parabola in the $xy$-plane.

Since the range of $y$ is $y \in [-1, 1]$, and based on the characteristic of 
the parabola $x = 2 y^2 - 1$, the part of the curve that is actually traced out 
by $P$ is the part of the parabola between $(-1, 1)$ and $(1, 1)$.

c) According to the position vector function,
\begin{gather*}
  x(t) = t^2 + 1 \\
  y(t) = t^3 \\
\end{gather*}

Therefore, we can find the relation between $x$ and $y$:
\begin{equation*}
  y = (x - 1)^{\frac{3}{2}}
\end{equation*}
which is a symmetric graph related to the $x$ axis, in which the half above the 
$x$ axis is the graph of a power function.

Since the range of $x$ is $x \in [1, \infty)$, which is the domain of the 
function $y = (x - 1)^{\frac{3}{2}}$, the full curve is actually traced out by 
$P$.

d) According to the position vector function,
\begin{gather*}
  x(t) = \tan t \\
  y(t) = \sec t \\
\end{gather*}

Therefore, we can find the relation between $x$ and $y$:
\begin{gather*}
  \sec^2 t - \tan^2 t = 1 \\
  y^2 - x^2 = 1 \\
\end{gather*}
which is a hyperbola.

According to the graph of $x(t)$ and $y(t)$, the full hyperbola is actually 
traced out by $P$.

\begin{question}
  We know that the relation we find between $x$ and $y$ is correct, however, is 
  it possible that $x$ and $y$ also satisfy other relation which is not 
  equivalent to the one we find?
\end{question}

\bigskip

5. A string is wound clockwise around the circle of radius $a$ centered at the 
origin $O$; the initial position of the end $P$ of the string is $(a, 0)$. 
Unwind the string, always pulling it taut (so it stays tangent to the circle). 
Write parametric equations for the motion of $P$.

Solution:

Since the string is wound clockwise, when we unwind it, the direction is 
counter-clockwise. Also, since the string is unwound around a circle, it is 
natural to use the parameter $\theta$ to describe the motion of $P$, where 
$\theta$ is the corresponding angle of the unwound part of the string. Hence the 
length of the unwound part of the string is $a \theta$.

During the unwinding, the string is always pulled taut, hence there is always a 
point $P'$ on the circle which is the point of tangency between the circle and 
the unwound part of the string. The coordinates of $P'$ can be described as 
$(a \cos\theta, a \sin\theta)$.

Next we need to explore the direction of the vector $\vec{P'P}$ in order to 
describe the coordinates of $P$. According to the tangency relationship between 
the circle and the unwound part of the string,
\begin{gather*}
  \vec{P'P} \perp \vec{OP'} \\
  dir(\vec{OP'}) = \langle \cos\theta, \sin\theta \rangle \\
\end{gather*}

Therefore, we have the following system of equations
\begin{equation*}
  \begin{cases}
    dir(\vec{P'P}) \cdot \langle \cos\theta, \sin\theta \rangle = 0 \\
    |dir(\vec{P'P})| = 1 \\
  \end{cases}
\end{equation*}

After solving the above system of equations, there are two possible solutions:
\begin{gather*}
  dir(\vec{P'P}) = \langle \sin\theta, -\cos\theta \rangle \\
  dir(\vec{P'P}) = \langle -\sin\theta, \cos\theta \rangle \\
\end{gather*}

According to the geometric interpretation of unwinding the string 
counter-clockwise, the correct direction of the vector $\vec{P'P}$ is 
$\langle \sin\theta, -\cos\theta \rangle$.

Therefore, the parametric equation for the motion of $P$ is
\begin{equation*}
  \begin{cases}
    x(t) = a\cos\theta + a \theta \sin\theta \\
    y(t) = a\sin\theta - a \theta \cos\theta \\
  \end{cases}
\end{equation*}

\bigskip

7. The cycloid is the curve traced out by a fix point $P$ on a circle of radius 
$a$ which rolls along the $x$-axis in the positive direction, starting when $P$ 
is at the origin $O$. Find the vector function $OP$; use as variable the angle 
$\theta$ through which the circle has rolled.

Solution:

Let $C$ denote the center of the rolling circle. At a given value of $\theta$, 
suppose that the tangency point between the circle and the $x$ axis is $A$.

Since the circle is rolling along the $x$ axis without any slip, then
\begin{equation*}
  OA = a \theta
\end{equation*}

Also $AC = a$, hence the coordinates of $C$ at the given value of $\theta$ is 
\begin{equation*}
  C = (a \theta, a)
\end{equation*}

The direction of $\vec{CP}$ can be described by
\begin{gather*}
  dir(\vec{CP}) = \langle \cos(-\frac{\pi}{2} - \theta), \sin(-\frac{\pi}{2} - \theta) \rangle \\
  dir(\vec{CP}) = \langle -\sin\theta, -\cos\theta \rangle \\
\end{gather*}

Therefore, the parametric equation for the motion of $P$ is
\begin{equation*}
  \begin{cases}
    x(t) = a \theta - a \sin\theta \\
    y(t) = a - a \cos\theta \\
  \end{cases}
\end{equation*}

\subsection*{Unit 1J Differentiation of Vector Functions}

\bigskip

1. For each of the following vector functions of time, calculate the velocity, 
speed $|ds/dt|$, unit tangent vector (in the direction of velocity), and 
acceleration. \\
a) $e^t \vec{i} + e^{-t} \vec{j}$ \\
b) $t^2 \vec{i} + t^3 \vec{j}$ \\
c) $(1 - 2t^2) \vec{i} + t^2 \vec{j} + (-2 + 2t^2) \vec{k}$

Solution:

a) 
\begin{gather*}
  \begin{split}
    \vec{v} &= \frac{d\vec{r}}{dt} \\
            &= \frac{d(e^t \vec{i} + e^{-t} \vec{j})}{dt} \\
            &= e^t \vec{i} - e^{-t} \vec{j} \\
  \end{split} \\
  \begin{split}
    |ds/dt| &= |\vec{v}| \\
            &= |e^t \vec{i} - e^{-t} \vec{j}| \\
            &= \sqrt{(e^t)^2 + (e^{-t})^2} \\
            &= \frac{\sqrt{e^{4t} + 1}}{e^t} \\
  \end{split} \\
  \begin{split}
    \vec{T} &= dir(\vec{v}) \\
            &= \frac{\vec{v}}{|\vec{v}|} \\
            &= \frac{e^{2t}}{\sqrt{e^{4t} + 1}} \vec{i} + \frac{1}{\sqrt{e^{4t} + 1}} \vec{j} \\
  \end{split} \\
  \begin{split}
    \vec{a} &= \frac{d\vec{v}}{dt} \\
            &= \frac{d(e^t \vec{i} - e^{-t} \vec{j})}{dt} \\
            &= e^t \vec{i} + e^{-t} \vec{j} \\
  \end{split} \\
\end{gather*}

b) 
\begin{gather*}
  \begin{split}
    \vec{v} &= \frac{d\vec{r}}{dt} \\
            &= \frac{d(t^2 \vec{i} + t^3 \vec{j})}{dt} \\
            &= 2t \vec{i} + 3t^2 \vec{j} \\
  \end{split} \\
  \begin{split}
    |ds/dt| &= |\vec{v}| \\
            &= |2t \vec{i} + 3t^2 \vec{j}| \\
            &= \sqrt{(2t)^2 + (3t^2)^2} \\
            &= \sqrt{9t^4 + 4t^2} \\
  \end{split} \\
  \begin{split}
    \vec{T} &= dir(\vec{v}) \\
            &= \frac{\vec{v}}{|\vec{v}|} \\
            &= \frac{2t}{\sqrt{9t^4 + 4t^2}} \vec{i} + \frac{3t^2}{\sqrt{9t^4 + 4t^2}} \vec{j} \\
  \end{split} \\
  \begin{split}
    \vec{a} &= \frac{d\vec{v}}{dt} \\
            &= \frac{d(2t \vec{i} + 3t^2 \vec{j})}{dt} \\
            &= 2 \vec{i} + 6t \vec{j} \\
  \end{split} \\
\end{gather*}

c) 
\begin{gather*}
  \begin{split}
    \vec{v} &= \frac{d\vec{r}}{dt} \\
            &= \frac{d((1 - 2t^2) \vec{i} + t^2 \vec{j} + (-2 + 2t^2) \vec{k})}{dt} \\
            &= -4t \vec{i} + 2t \vec{j} + 4t \vec{k} \\
  \end{split} \\
  \begin{split}
    |ds/dt| &= |\vec{v}| \\
            &= |-4t \vec{i} + 2t \vec{j} + 4t \vec{k}| \\
            &= \sqrt{(-4t)^2 + (2t)^2 + (4t)^2} \\
            &= 6|t| \\
  \end{split} \\
  \begin{split}
    \vec{T} &= dir(\vec{v}) \\
            &= \frac{\vec{v}}{|\vec{v}|} \\
            &= \begin{cases}
                 -\frac{2}{3} \vec{i} + \frac{1}{3} \vec{j} + \frac{2}{3} \vec{k} ,\ t \geq 0 \\
                 \frac{2}{3} \vec{i} - \frac{1}{3} \vec{j} - \frac{2}{3} \vec{k} ,\ t < 0 \\
               \end{cases} \\
  \end{split} \\
  \begin{split}
    \vec{a} &= \frac{d\vec{v}}{dt} \\
            &= \frac{d(-4t \vec{i} + 2t \vec{j} + 4t \vec{k})}{dt} \\
            &= -4 \vec{i} + 2 \vec{j} + 4 \vec{k} \\
  \end{split} \\
\end{gather*}

\bigskip

2. Let $OP = \frac{1}{1 + t^2} \vec{i} + \frac{t}{1 + t^2} \vec{j}$ be the 
position vector for a motion. \\
a) Calculate $\vec{v}$, $|ds/dt|$, and $\vec{T}$. \\
b) At what point in the speed greatest? smallest? \\
c) Find the $xy$-equation of the curve along which the point $P$ is moving, and 
describe it geometrically.

Solution:

a) 
\begin{gather*}
  \begin{split}
    \vec{v} &= \frac{d\vec{r}}{dt} \\
            &= \frac{d(\frac{1}{1 + t^2} \vec{i} + \frac{t}{1 + t^2} \vec{j})}{dt} \\
            &= -\frac{2t}{(t^2 + 1)^2} \vec{i} + \frac{1 - t^2}{(t^2 + 1)^2} \vec{j} \\
  \end{split} \\
  \begin{split}
    |ds/dt| &= |\vec{v}| \\
            &= |-\frac{2t}{(t^2 + 1)^2} \vec{i} + \frac{1 - t^2}{(t^2 + 1)^2} \vec{j}| \\
            &= \sqrt{(-\frac{2t}{(t^2 + 1)^2})^2 + (\frac{1 - t^2}{(t^2 + 1)^2})^2} \\
            &= \frac{1}{t^2 + 1} \\
  \end{split} \\
  \begin{split}
    \vec{T} &= dir(\vec{v}) \\
            &= \frac{\vec{v}}{|\vec{v}|} \\
            &= -\frac{2t}{t^2 + 1} \vec{i} + \frac{1 - t^2}{t^2 + 1} \vec{j} \\
  \end{split} \\
\end{gather*}

b) According to the expression of the speed $|ds/dt|$, it is obvious that when 
$t = 0$, the speed has its greatest value which is $1$, and when $t = \infty$ or 
$t = -\infty$, the speed has its smallest value which is $0$.

c) According to the position vector function,
\begin{gather*}
  x(t) = \frac{1}{1 + t^2} \\
  y(t) = \frac{t}{1 + t^2} \\
\end{gather*}

Therefore, we can find the relation between $x$ and $y$ as
\begin{gather*}
  (\frac{1}{1 + t^2})^2 + (\frac{t}{1 + t^2})^2 = \frac{1 + t^2}{(1 + t^2)^2} = \frac{1}{1 + t^2} \\
  x^2 + y^2 = x \\
  x^2 - x + \frac{1}{4} + y^2 = \frac{1}{4} \\
  (x - \frac{1}{2})^2 + y^2 = \frac{1}{4} \\
\end{gather*}
which is a circle of radius $\frac{1}{2}$ centered at $(\frac{1}{2}, 0)$.

According to the graph of $x$ and $y$, the full circle is actually traced out by 
the motion of $P$.

\bigskip

3. Prove the rule for differentiating the scalar product of two plane vector 
functions:
\begin{equation*}
  \frac{d}{dt} \vec{r} \cdot \vec{s} = \frac{d\vec{r}}{dt} \cdot \vec{s} + \vec{r} \cdot \frac{d\vec{s}}{dt}
\end{equation*}
by calculating with components, letting $\vec{r} = x_1 \vec{i} + y_1 \vec{j}$, 
and $\vec{s} = x_2 \vec{i} + y_2 \vec{j}$.

Solution:

\begin{gather*}
  \vec{r} \cdot \vec{s} = x_1 x_2 + y_1 y_2 \\
  \begin{split}
    \frac{d}{dt} \vec{r} \cdot \vec{s} &= \frac{d(x_1 x_2 + y_1 y_2)}{dt} \\
                                       &= \frac{d(x_1 x_2)}{dt} + \frac{d(y_1 y_2)}{dt} \\
                                       &= \frac{d x_1}{dt} x_2 + x_1 \frac{d x_2}{dt} + \frac{d y_1}{dt} y_2 + y_1 \frac{d y_2}{dt} \\
                                       &= (\frac{d x_1}{dt} x_2 + \frac{d y_1}{dt} y_2) + (x_1 \frac{d x_2}{dt} + y_1 \frac{d y_2}{dt}) \\
                                       &= (\langle \frac{d x_1}{dt}, \frac{d y_1}{dt} \rangle \cdot \langle x_2, y_2 \rangle) + (\langle x_1, y_1\rangle \cdot \langle \frac{d x_2}{dt}, \frac{d y_2}{dt}\rangle) \\
                                       &= \frac{d\vec{r}}{dt} \cdot d\vec{s} + \vec{r} \cdot \frac{d\vec{s}}{dt} \\
  \end{split} \\
\end{gather*}

Q.E.D.

\bigskip

4. Suppose a point $P$ moves on the surface of a sphere with center at the 
origin; let
\begin{equation*}
  OP = \vec{r}(t) = x(t) \vec{i} + y(t) \vec{j} + z(t) \vec{k}
\end{equation*}

Show that the velocity vector $v$ is always perpendicular to $\vec{r}$ in two 
different ways: \\
a) using the $x,y,z$-coordinates \\
b) without coordinates (use the formula in \textbf{1J-3}, which is also valid in 
space).

Also Prove the converse: if $\vec{r}$ and $\vec{v}$ are perpendicular, then the 
motion of $P$ is on the surface of a sphere centered at the origin.

Solution:

To prove that the velocity vector $v$ is always perpendicular to $\vec{r}$ for 
the described motion of $P$:

a)
\begin{gather*}
  \begin{split}
    \vec{v}(t) &= \frac{d(\vec{r}(t))}{dt} \\
               &= \frac{d(x(t))}{dt} \vec{i} + \frac{d(y(t))}{dt} \vec{j} + \frac{d(z(t))}{dt} \vec{k} \\
  \end{split} \\
  \begin{split}
    \vec{r} \cdot \vec{v} &= (x(t) \vec{i} + y(t) \vec{j} + z(t) \vec{k}) \cdot (\frac{d(x(t))}{dt} \vec{i} + \frac{d(y(t))}{dt} \vec{j} + \frac{d(z(t))}{dt} \vec{k}) \\
                          &= x(t) \frac{d(x(t))}{dt} + y(t) \frac{d(y(t))}{dt} + z(t) \frac{d(z(t))}{dt} \\
  \end{split} \\
\end{gather*}

Since 
\begin{gather*}
  \frac{d((x(t))^2)}{dt} = \frac{d(x(t))}{dt} x(t) + x(t) \frac{d(x(t))}{dt} \\
  \frac{d((x(t))^2)}{dt} = 2 x(t) \frac{d(x(t))}{dt} \\
  x(t) \frac{d(x(t))}{dt} = \frac{1}{2} \frac{d((x(t))^2)}{dt} \\
\end{gather*}
then
\begin{equation*}
  \begin{split}
    \vec{r} \cdot \vec{v} &= x(t) \frac{d(x(t))}{dt} + y(t) \frac{d(y(t))}{dt} + z(t) \frac{d(z(t))}{dt} \\
                          &= \frac{1}{2} \frac{d((x(t))^2)}{dt} + \frac{1}{2} \frac{d((y(t))^2)}{dt} + \frac{1}{2} \frac{d((z(t))^2)}{dt} \\
                          &= \frac{1}{2} \frac{d}{dt} ((x(t))^2 + (y(t))^2 + (z(t))^2) \\
  \end{split}
\end{equation*}

Since the point $P$ moves on the surface of a sphere with center at the origin, then
\begin{equation*}
  (x(t))^2 + (y(t))^2 + (z(t))^2 = R^2 ,\ \textnormal{R is the radius of the sphere}
\end{equation*}

Therefore,
\begin{equation*}
  \begin{split}
    \vec{r} \cdot \vec{v} &= \frac{1}{2} \frac{d}{dt} ((x(t))^2 + (y(t))^2 + (z(t))^2) \\
                          &= \frac{1}{2} \frac{d(R^2)}{dt} \\
                          &= 0 \\
  \end{split}
\end{equation*}
Hence the velocity vector $v$ is always perpendicular to $\vec{r}$. Q.E.D.

b)
\begin{gather*}
  \frac{d}{dt} (\vec{r} \cdot \vec{r}) = \frac{d\vec{r}}{dt} \cdot \vec{r} + \vec{r} \cdot \frac{d\vec{r}}{dt} \\
  \frac{d}{dt} (\vec{r} \cdot \vec{r}) = 2 \frac{d\vec{r}}{dt} \cdot \vec{r} \\
  \frac{d}{dt} (\vec{r} \cdot \vec{r}) = 2 \vec{v} \cdot \vec{r} \\
  \begin{split}
    \vec{v} \cdot \vec{r} &= \frac{1}{2} \frac{d}{dt} \vec{r} \cdot \vec{r} \\
                          &= \frac{1}{2} \frac{d}{dt} |\vec{r}|^2 \\
  \end{split}
\end{gather*}

Since the point $P$ moves on the surface of a sphere with center at the origin, then
\begin{equation*}
  |\vec{r}|^2 = R^2 ,\ \textnormal{R is the radius of the sphere}
\end{equation*}

Therefore,
\begin{equation*}
  \begin{split}
    \vec{r} \cdot \vec{v} &= \frac{1}{2} \frac{d}{dt} |\vec{r}|^2 \\
                          &= \frac{1}{2} \frac{d(R^2)}{dt} \\
                          &= 0 \\
  \end{split}
\end{equation*}
Hence the velocity vector $v$ is always perpendicular to $\vec{r}$. Q.E.D.

To prove that if $\vec{r}$ and $\vec{v}$ are perpendicular, then the motion of 
$P$ is on the surface of a sphere centered at the origin:

\begin{gather*}
  \frac{d}{dt} (\vec{r} \cdot \vec{r}) = \frac{d\vec{r}}{dt} \cdot \vec{r} + \vec{r} \cdot \frac{d\vec{r}}{dt} \\
  \frac{d}{dt} (\vec{r} \cdot \vec{r}) = 2 \frac{d\vec{r}}{dt} \cdot \vec{r} \\
  \frac{d}{dt} (\vec{r} \cdot \vec{r}) = 2 \vec{v} \cdot \vec{r} \\
  \frac{d}{dt} (|\vec{r}|^2) = 2 \vec{v} \cdot \vec{r} \\
\end{gather*}

Since $\vec{r}$ and $\vec{v}$ are perpendicular, then
\begin{equation*}
  \vec{r} \cdot \vec{v} = 0
\end{equation*}

Therefore,
\begin{gather*}
  \begin{split}
    \frac{d}{dt} (|\vec{r}|^2) &= 2 \vec{v} \cdot \vec{r} \\
                               &= 0 \\
  \end{split} \\
  |\vec{r}|^2 = c ,\ \textnormal{c is a constant}
\end{gather*}

Therefore, the motion of the point $P$ is on a sphere of radius $\sqrt{c}$ with 
center at the origin. Q.E.D.

\bigskip

5. a) Suppose a point moves with constant speed. Show that its velocity vector 
and acceleration vector are perpendicular. (Use the formula in \textbf{1J-3}) \\
b) Show the converse: if the velocity and acceleration vectors are perpendicular, 
the point $P$ moves with constant speed.

Solution:

a) 
\begin{gather*}
  \frac{d}{dt} (\vec{v} \cdot \vec{v}) = \frac{d\vec{v}}{dt} \cdot \vec{v} + \vec{v} \cdot \frac{d\vec{v}}{dt} \\
  \frac{d}{dt} (\vec{v} \cdot \vec{v}) = 2 \frac{d\vec{v}}{dt} \cdot \vec{v} \\
  \frac{d}{dt} (\vec{v} \cdot \vec{v}) = 2 \vec{a} \cdot \vec{v} \\
  \begin{split}
    \vec{a} \cdot \vec{v} &= \frac{1}{2} \frac{d}{dt} \vec{v} \cdot \vec{v} \\
                          &= \frac{1}{2} \frac{d}{dt} |\vec{v}|^2 \\
  \end{split}
\end{gather*}

Since the described point moves with constant, then
\begin{gather*}
  |\vec{v}|^2 = c ,\ \textnormal{c is a constant} \\
  \begin{split}
    \vec{a} \cdot \vec{v} &= \frac{1}{2} \frac{d}{dt} |\vec{v}|^2 \\
                          &= 0 \\
  \end{split}
\end{gather*}

Hence the velocity vector $v$ is always perpendicular to the acceleration vector 
$\vec{a}$. Q.E.D.

b)
\begin{gather*}
  \frac{d}{dt} (\vec{v} \cdot \vec{v}) = \frac{d\vec{v}}{dt} \cdot \vec{v} + \vec{v} \cdot \frac{d\vec{v}}{dt} \\
  \frac{d}{dt} (\vec{v} \cdot \vec{v}) = 2 \frac{d\vec{v}}{dt} \cdot \vec{v} \\
  \frac{d}{dt} (\vec{v} \cdot \vec{v}) = 2 \vec{a} \cdot \vec{v} \\
  \frac{d}{dt} (|\vec{v}|^2) = 2 \vec{a} \cdot \vec{v} \\
\end{gather*}

Since $\vec{a}$ and $\vec{v}$ are perpendicular, then
\begin{gather*}
  \vec{a} \cdot \vec{v} = 0 \\
  \begin{split}
    \frac{d}{dt} (|\vec{v}|^2) &= 2 \vec{a} \cdot \vec{v} \\
                               &= 0 \\
  \end{split} \\
  |\vec{v}|^2 = c ,\ \textnormal{c is a constant}
\end{gather*}

Therefore, the described point $P$ moves with constant speed. Q.E.D.

\bigskip

6. For the helical motion 
$\vec{r}(t) = a \cos t \vec{i} + a \sin t \vec{j} + bt \vec{k}$, \\
a) calculate $\vec{v}$, $\vec{a}$, $\vec{T}$, $|ds/dt|$ \\
b) show that $\vec{v}$ and $\vec{a}$ are perpendicular; explain using 
\textbf{1J-5}

Solution:

a) 
\begin{gather*}
  \begin{split}
    \vec{v} &= \frac{d(\vec{r}(t))}{dt} \\
            &= \frac{d(a \cos t \vec{i} + a \sin t \vec{j} + bt \vec{k})}{dt} \\
            &= - a \sin t \vec{i} + a \cos t \vec{j} + b \vec{k} \\
  \end{split} \\
  \begin{split}
    \vec{a} &= \frac{d(\vec{v}(t))}{dt} \\
            &= \frac{d(- a \sin t \vec{i} + a \cos t \vec{j} + b \vec{k})}{dt} \\
            &= - a \cos t \vec{i} - a \sin t \vec{j} \\
  \end{split} \\
  \begin{split}
    |ds/dt| &= |\vec{v}(t)| \\
            &= |- a \sin t \vec{i} + a \cos t \vec{j} + b \vec{k}| \\
            &= \sqrt{(-a \sin t)^2 + (a \cos t)^2 + b^2} \\
            &= \sqrt{a^2 + b^2} \\
  \end{split} \\
  \begin{split}
    \vec{T} &= dir(\vec{v}) \\
            &= \frac{\vec{v}}{|\vec{v}|} \\
            &= \frac{- a \sin t \vec{i} + a \cos t \vec{j} + b \vec{k}}{\sqrt{a^2 + b^2}} \\
            &= - \frac{a}{\sqrt{a^2 + b^2}} \sin t \vec{i} + \frac{a}{\sqrt{a^2 + b^2}} \cos t \vec{j} + \frac{b}{\sqrt{a^2 + b^2}} \vec{k} \\
  \end{split}
\end{gather*}

b) 
Since
\begin{equation*}
  \begin{split}
    \vec{v} \cdot \vec{a} &= (- a \sin t \vec{i} + a \cos t \vec{j} + b \vec{k}) \cdot (- a \cos t \vec{i} - a \sin t \vec{j}) \\
                          &= a^2 \sin t \cos t - a^2 \sin t \cos t + 0 \cdot b \\
                          &= 0 \\
  \end{split}
\end{equation*}
$\vec{v}$ and $\vec{a}$ are perpendicular.

According to the conclusion in \textbf{1J-5}, since we know that the speed of 
the movement is constant, i.e.
\begin{equation*}
  |ds/dt| = \sqrt{a^2 + b^2}
\end{equation*}
then the velocity $\vec{v}$ and the accceleration $\vec{a}$ are perpendicular.

\bigskip

9. A point $P$ is moving in space, with position vector
\begin{equation*}
  \vec{r} = OP = 3 \cos t \vec{i} + 5 \sin t \vec{j} + 4 \cos t \vec{k}
\end{equation*} \\
a) Show it moves on the surface of a sphere. \\
b) Show its speed is constant. \\
c) Show the acceleration is directed toward the origin. \\
d) Show it moves in a plane through the origin. \\
e) Describe the path of the point.

Solution:

a) According to the position vector of the point $P$, the parametric equation of 
the movement of the point $P$ is
\begin{equation*}
  \begin{cases}
    x(t) = 3 \cos t \\
    y(t) = 5 \sin t \\
    z(t) = 4 \cos t \\
  \end{cases}
\end{equation*}

Then the coordinates of the point $P$ satisfies
\begin{equation*}
  \begin{split}
    (x(t))^2 + (y(t))^2 + (z(t))^2 &= (3 \cos t)^2 + (5 \sin t)^2 + (4 \cos t)^2 \\
                                   &= 9 \cos^2 t + 25 \sin^2 t + 16 \cos^2 t \\
                                   &= 25 \\
  \end{split}
\end{equation*}

Therefore, the movement of the point $P$ is always on the surface of 
$x^2 + y^2 + z^2 = 25$, which is a sphere.

b) 
\begin{gather*}
  \begin{split}
    \vec{v} &= \frac{d(\vec{r}(t))}{dt} \\
            &= \frac{d(3 \cos t \vec{i} + 5 \sin t \vec{j} + 4 \cos t \vec{k})}{dt} \\
            &= - 3 \sin t \vec{i} + 5 \cos t \vec{j} - 4 \sin t \vec{k} \\
  \end{split} \\
  \begin{split}
    |ds/dt| &= |\vec{v}| \\
            &= |- 3 \sin t \vec{i} + 5 \cos t \vec{j} - 4 \sin t \vec{k}| \\
            &= \sqrt{(-3 \sin t)^2 + (5 \cos t)^2 + (-4 \sin t)^2} \\
            &= 5 \\
  \end{split}
\end{gather*}

c)
\begin{equation*}
  \begin{split}
    \vec{a} &= \frac{d(\vec{v}(t))}{dt} \\
            &= \frac{d(- 3 \sin t \vec{i} + 5 \cos t \vec{j} - 4 \sin t \vec{k})}{dt} \\
            &= -3 \cos t \vec{i} - 5 \sin t \vec{j} - 4 \cos t \vec{k} \\
  \end{split}
\end{equation*}

Therefore,
\begin{equation*}
  \vec{a} = - \vec{r} = -OP = PO
\end{equation*}

Since $\vec{r}$ is the vector from the origin to the point $P$, then at the 
point $P$, $\vec{a}$, which has the reverse direction of $\vec{r}$, points back 
to the origin.

d) According to the parametric equation of the point $P$ in part (a), the 
coordinates of the point $P$ satisfy
\begin{equation*}
  4 x(t) + 0 y(t) - 3 z(t) = 4 \cdot 3 \cos t + 0 \cdot 5 \sin t - 3 \cdot 4 \cos t = 0
\end{equation*}

Therefore, the point $P$ moves in a plane through the origin whose normal vector is 
$\langle 4, 0, -3 \rangle$.

e) Since the movement of the point $P$ is on the surface of a sphere whose 
center is the origin, and also in a plane through the origin, then the path of 
the point $P$ is the intersection of the two geometric objects, which is a 
circle of radius 5 whose center is the origin.

The equation of the circle can be given as
\begin{equation*}
  \begin{cases}
    x^2 + y^2 + z^2 = 25 \\
    4x - 3z = 0 \\
  \end{cases}
\end{equation*}

\subsection*{Unit 1K. Kepler's Second Law}

\bigskip

2. Let $\vec{s}(t)$ be a vector function. Prove by using components that
\begin{equation*}
  \frac{d\vec{s}}{dt} = \vec{0} \ \Rightarrow \ \vec{s}(t) = \vec{K},\ \textnormal{where $\vec{K}$ is a constant vector.}
\end{equation*}

Solution:

Suppose that $\vec{s}(t) = x(t) \vec{i} + y(t) \vec{j} + z(t) \vec{k}$, then
\begin{equation*}
  \frac{d\vec{s}}{dt} = \frac{d(x(t))}{dt} \vec{i} + \frac{d(y(t))}{dt} \vec{j} + \frac{d(z(t))}{dt} \vec{k} = \vec{0}
\end{equation*}

which is equivalent to
\begin{equation*}
  \begin{cases}
    \frac{d(x(t))}{dt} = 0 \\
    \frac{d(y(t))}{dt} = 0 \\
    \frac{d(z(t))}{dt} = 0 \\
  \end{cases}
\end{equation*}

which is equivalent to
\begin{equation*}
  \begin{cases}
    x(t) = x_0,\ \textnormal{$x_0$ is a constant} \\
    y(t) = y_0,\ \textnormal{$y_0$ is a constant} \\
    z(t) = z_0,\ \textnormal{$z_0$ is a constant} \\
  \end{cases}
\end{equation*}

Therefore,
\begin{equation*}
  \begin{split}
    \vec{s}(t) &= x(t) \vec{i} + y(t) \vec{j} + z(t) \vec{k} \\
               &= x_0 \vec{i} + y_0 \vec{j} + z_0 \vec{k} \\
  \end{split}
\end{equation*}
where $x_0$, $y_0$, and $z_0$ are constants. Hence, $\vec{s}(t)$ is a constant 
vector. Moreover, this conclusion is valid no matter how many dimensions the 
vector has.

\begin{question}
  I believe the converse is also true. Prove it.
\end{question}

\bigskip

3. In our proof that Kepler's second law is equivalent to the force being 
central, used the following steps to show the second law implies a central force. 
Kepler's second law says the motion is in a plane and 
\begin{equation*}
  2 \frac{dA}{dt} = |\vec{r} \times \vec{v}| \textnormal{ is constant.}
\end{equation*}

This implies $\vec{r} \times \vec{v}$ is constant. So,
\begin{equation*}
  0 = \frac{d}{dt}(\vec{r} \times \vec{v}) = \vec{v} \times \vec{v} + \vec{r} \times \vec{a} = \vec{r} \times \vec{a}
\end{equation*}

This implies $\vec{a}$ and $\vec{r}$ are parallel, i.e. the force is central.

Reverse the two steps to prove the converse: for motion under any type of 
central force, the path of motion will lie in a plane and area will be swept out 
by the radius vector at a constant rate.

Solution:

The motion is only under a central force \\
$\iff$ $\vec{r} \parallel \vec{a}$ \\
$\iff$ $\vec{r} \times \vec{a} = 0$

Therefore,
\begin{gather*}
  \vec{r} \times \vec{a} = 0 \\
  \vec{r} \times \vec{a} + \vec{v} \times \vec{v} = 0 \\
  \frac{d}{dt} (\vec{r} \times \vec{v}) = 0 \\
  \vec{r} \times \vec{v} = \vec{K},\ \textnormal{where $\vec{K}$ is a constant vector} \\
  \begin{cases}
    |\vec{r} \times \vec{v}| \textnormal{ is constant.} \\
    dir(\vec{r} \times \vec{v}) \textnormal{ is constant.} \\
  \end{cases}
\end{gather*}

Since
\begin{equation*}
  \begin{split}
    |\vec{r} \times \vec{v}| &= |\vec{r} \times \frac{d\vec{r}}{dt}| \\
                             &= \frac{|\vec{r} \times d\vec{r}|}{dt} \\
                             &= 2 \frac{dA}{dt} \\
  \end{split}
\end{equation*}

Then
\begin{equation*}
  |\vec{r} \times \vec{v}| \textnormal{ is constant.} \iff \frac{dA}{dt} \textnormal{ is constant.}
\end{equation*}

Since
\begin{equation*}
  dir(\vec{r} \times \vec{v}) \textnormal{ is constant.} 
\end{equation*}

Then the normal vector of the plane determined by the position vector and the 
velocity vector is constant. Hence the movement is always in the plane.

Q.E.D.

\begin{center}
\section*{Part II}
\end{center}

\bigskip

1. A circular disk of radius 2 has a dot marked at a point half-way between the 
center and the circumference. Denote this point by $P$. Suppose that the disk is 
tangent to the $x$-axis with the center initially at $(0, 2)$ and $P$ initially 
at $(0, 1)$ and that it starts to roll to the right on the $x$-axis at unit 
speed. Let $C$ be the curve traced out by the point $P$. \\
a) Make a sketch of what you think the curve $C$ will look like. \\
b) Use vectors to find the parametric equations for $\vec{OP}$ as a function of 
time $t$. \\
c) Open the 'Mathlet' Wheel (with link on course webpage) and set the parameters 
to view an animation of this particular motion problem. Then activate the 
'Trace' function to see a graph of the curve $C$. If this graph is substantially 
different from your hand sketch, sketch it also and then describe what led you 
to produce your first idea of the graph.


\end{document}