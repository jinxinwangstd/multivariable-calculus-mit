\documentclass{article}
\usepackage[utf8]{inputenc}
\usepackage{amsmath}
\usepackage{amssymb}
\usepackage{amsthm}
\usepackage{tikz}
\setlength{\parindent}{0pt}

\newtheorem*{theorem}{Theorem}
\newtheorem*{definition}{Definition}
\newtheorem*{lemma}{Lemma}
\newtheorem*{corollary}{Corollary}
\newtheorem{example}{Example}
\newtheorem{trick}{Trick}
\newtheorem{question}{Question}

\title{Problem Set 3}
\author{}
\date{}

\begin{document}

\begin{center}
{\rmfamily\bfseries\Large 18.02 EXERCISES}

\vspace{25px}

{\rmfamily\bfseries\LARGE Problem Set 3: Parametric Equations for Curves}
\end{center}

\begin{center}
\section*{Part I}
\end{center}

\subsection*{Unit 1E Equations of Lines and Planes}

\bigskip

4. Where does the line through $(0, 1, 2)$ and $(2, 0, 3)$ intersect the plane 
$x + 4y + z = 4$?

Solution:

A vector along the line is $<2, -1, 1>$. Hence, the parametric equation of the 
line is
\begin{equation*}
  \begin{cases}
    x = x(t) = 0 + 2t = 2t \\
    y = y(t) = 1 - t \\
    z = z(t) = 2 + t \\
  \end{cases}
\end{equation*}

For the intersection point of the line and the plane, it satisfies
\begin{gather*}
  x(t) + 4y(t) + z(t) = 4 \\
  2t + 4(1 - t) + 2 + t = 4 \\
  -t = -2 \\
  t = 2 \\
\end{gather*}

Therefore, the coordinates of the intersection point is $(x(2), y(2), z(2))$, 
which is $(4, -1, 4)$.

\bigskip

7. Formulate a general method for finding the distance between two skew (i.e., 
non-intersecting) lines in space, and carry it out for two non-intersecting 
lines lying along the diagonals of two adjacent faces of the unit cube (place it 
in the first octant, with one vertex at the origin).

Solution:

7-1. Formulate a general method for finding the distance between two parallel 
planes in space:

Suppose that the equations of two parallel planes are respectively
\begin{gather*}
  n_1x + n_2y + n_3z = c_1 \\
  n_1x + n_2y + n_3z = c_2 \\
\end{gather*}
Then the vector $\vec{n} = <n_1, n_2, n_3>$ is one of the normal vectors of the 
two planes. Therefore, we can construct the parametric equation of a line 
perpendicular to the two planes as
\begin{equation*}
  \begin{cases}
    x(t) = n_1t \\
    y(t) = n_2t \\
    z(t) = n_3t \\
  \end{cases}
\end{equation*}
Therefore, the two intersection points the line has with the two planes 
respectively satisfy
\begin{gather*}
  n_1x(t_1) + n_2y(t_1) + n_3z(t_1) = c_1 \\
  n_1^2t_1 + n_2^2t_1 + n_3^2t_1 = c_1 \\
  t_1 = \frac{c_1}{n_1^2 + n_2^2 + n_3^2} \\
  n_1x(t_2) + n_2y(t_2) + n_3z(t_2) = c_2 \\
  n_1^2t_2 + n_2^2t_2 + n_3^2t_2 = c_1 \\
  t_2 = \frac{c_2}{n_1^2 + n_2^2 + n_3^2} \\
\end{gather*}
The distance between two points in the line can be calculated as 
\begin{equation*}
  \begin{split}
    d &= \sqrt{(x(t_1) - x(t_2))^2 + (y(t_1) - y(t_2))^2 + (z(t_1) - z(t_2))^2} \\
      &= \sqrt{(n_1^2 + n_2^2 + n_3^2)(t_1 - t_2)^2} \\
      &= \sqrt{(n_1^2 + n_2^2 + n_3^2)(\frac{c_1 - c_2}{n_1^2 + n_2^2 + n_3^2})^2} \\
      &= \frac{|c_1 - c_2|}{\sqrt{n_1^2 + n_2^2 + n_3^2}} \\
      &= \frac{|c_1 - c_2|}{|\vec{n}|} \\
  \end{split}
\end{equation*}

7-2. Formulate a general method for finding the distance between two 
non-parallel lines in space:

Suppose that there are two non-parallel lines in space named $l_a$ and $l_b$. 
Their normal vectors are respectively 
$\vec{n_a} = \langle n_{ax}, n_{ay}, n_{az} \rangle$ and 
$\vec{n_b} = \langle n_{bx}, n_{by}, n_{bz} \rangle$, and their parametric 
equations are respectively:
\begin{gather*}
  \begin{cases}
    x_a(t) = x_{a0} + n_{ax}t \\
    y_a(t) = y_{a0} + n_{ay}t \\
    z_a(t) = z_{a0} + n_{az}t \\
  \end{cases} \\
  \begin{cases}
    x_b(t) = x_{b0} + n_{bx}t \\
    y_b(t) = y_{b0} + n_{by}t \\
    z_b(t) = z_{b0} + n_{bz}t \\
  \end{cases} \\
\end{gather*}

Then we can find the vector $\vec{n}$ which is perpendicular to both $l_a$ and 
$l_b$ by
\begin{equation*}
  \begin{split}
    \vec{n} &= \vec{n_a} \times \vec{n_b} \\
            &= \langle n_x, n_y, n_z \rangle \\
  \end{split}
\end{equation*}

Then the distance between $l_a$ and $l_b$ is equal to the distance between the 
two planes $P_a$ and $P_b$ with $\vec{n}$ as their normal vectors where $l_a$ 
and $l_b$ lie within respectively.

The equation of $P_a$ is
\begin{equation*}
  n_x x + n_y y + n_z z = n_x x_{a0} + n_y y_{a0} + n_z z_{a0}
\end{equation*}

The equation of $P_b$ is 
\begin{equation*}
  n_x x + n_y y + n_z z = n_x x_{b0} + n_y y_{b0} + n_z z_{b0}
\end{equation*}

According to the previous part, the distance between $P_a$ and $P_b$ is
\begin{equation*}
  \begin{split}
    d &= \frac{|(n_x x_{a0} + n_y y_{a0} + n_z z_{a0}) - (n_x x_{b0} + n_y y_{b0} + n_z z_{b0})|}{|\vec{n}|} \\
      &= \frac{|\vec{n} \cdot \langle (x_{a0} - x_{b0}), (y_{a0} - y_{b0}), (z_{a0} - z_{b0}) \rangle|}{|\vec{n}|} \\
      &= \frac{|\vec{n} \cdot \vec{A_0B_0}|}{|\vec{n}|} \\
      &= \frac{|(\vec{n_a} \times \vec{n_b}) \cdot \vec{A_0B_0}|}{|\vec{n_a} \times \vec{n_b}|} \\
  \end{split}
\end{equation*}

Therefore, the distance between $l_a$ and $l_b$ is
\begin{equation*}
  d = \frac{|(\vec{n_a} \times \vec{n_b}) \cdot \vec{A_0B_0}|}{|\vec{n_a} \times \vec{n_b}|}
\end{equation*}

7-3 Apply the above method to find the distance between two non-intersecting 
lines lying along the diagonals of two adjacent faces of the unit cube (place it 
in the first octant, with one vertex at the origin).

The two non-intersecting diagonals of two adjacent faces of the unit cube I 
choose are the diagonal between $(0, 0, 0)$ and $(1, 0, 1)$, and the diagonal 
between $(0, 1, 0)$ and $(0, 0, 1)$. Therefore, the parametric equations of 
these two lines along the described diagonals are respectively:
\begin{gather*}
  \begin{cases}
    x_a(t) = t \\
    y_a(t) = 0 \\
    z_a(t) = t \\
  \end{cases} \\
  \begin{cases}
    x_b(t) = 0 \\
    y_b(t) = 1 - t \\
    z_b(t) = t \\
  \end{cases} \\
\end{gather*}

Therefore,
\begin{gather*}
  \vec{n_a} = \langle 1, 0, 1 \rangle \\
  \vec{n_b} = \langle 0, -1, 1 \rangle \\
  A_0 = (0, 0, 0) \\
  B_0 = (0, 1, 0) \\
  \vec{A_0B_0} = \langle 0, 1, 0 \rangle \\
\end{gather*}

Therefore, the distance between the two non-intersecting diagonals is
\begin{equation*}
  \begin{split}
    d &= \frac{|(\vec{n_a} \times \vec{n_b}) \cdot \vec{A_0B_0}|}{|\vec{n_a} \times \vec{n_b}|} \\
      &= \frac{|\langle 1, -1, -1 \rangle \cdot \langle 0, 1, 0 \rangle|}{|\langle 1, -1, -1 \rangle|} \\
      &= \frac{\sqrt{3}}{3} \\
  \end{split}
\end{equation*}

Therefore, the distance between the two non-intersecting diagonals of two 
adjacent faces of the unit cube is $\frac{\sqrt{3}}{3}$.

\subsection*{Unit 1I Vector Functions and Parametric Equations}

\bigskip

1. The point $P$ moves with constant speed $v$ in the direction of the constant 
vector $a \vec{i} + b \vec{j}$. If at time $t = 0$ it is at $(x_0, y_0)$, what 
is its position vector function $\vec{r}(t)$?

Solution:

The direction of the constant vector $a \vec{i} + b \vec{j}$ is
\begin{equation*}
  dir(A) = \frac{a}{\sqrt{a^2 + b^2}} \vec{i} + \frac{b}{\sqrt{a^2 + b^2}} \vec{j}
\end{equation*}

Hence in a period of time $t$, the point $P$ moves in a distance of $vt$ in the 
direction 
$\langle \frac{a}{\sqrt{a^2 + b^2}}, \frac{b}{\sqrt{a^2 + b^2}} \rangle$. Hence 
along the $x$-axis it moves in a distance of $\frac{a}{\sqrt{a^2 + b^2}} vt$, 
and along the $y$-axis it moves in a distance of 
$\frac{b}{\sqrt{a^2 + b^2}} vt$.

Therefore, the position vector function of the point $P$ is
\begin{equation*}
  \vec{r}(t) = \langle x_0 + \frac{a}{\sqrt{a^2 + b^2}} vt, y_0 + \frac{b}{\sqrt{a^2 + b^2}} vt \rangle
\end{equation*}

\bigskip

3. Describe the motions given by each of the following position vector 
functions, as $t$ goes from $-\infty$ to $\infty$. In each case, give the 
$xy$-equation of the curve along which $P$ travels, and tell what part of the 
curve is actually traced out by $P$. \\
a) $\vec{r} = 2 \cos^2 t \vec{i} + \sin^2 t \vec{j}$ \\
b) $\vec{r} = \cos 2t \vec{i} + \cos t \vec{j}$ \\
c) $\vec{r} = (t^2 + 1) \vec{i} + t^3 \vec{j}$ \\
d) $\vec{r} = \tan t \vec{i} + \sec t \vec{j}$

Solution:

a) According to the position vector function,
\begin{gather*}
  x(t) = 2 \cos^2 t \\
  y(t) = \sin^2 t \\
\end{gather*}

Therefore, we can find the relation between $x$ and $y$:
\begin{gather*}
  \frac{x}{2} + y = 1 \\
  y = -\frac{1}{2}x + 1 \\
\end{gather*}
which is a line in the $xy$-plane.

Since the ranges of $x$ and $y$ are respectively
\begin{gather*}
  x \in [0, 2] \\
  y \in [0, 1] \\
\end{gather*}
the part of the curve that is actually traced out by $P$ is the line segment 
from $(0, 0)$ to $(2, 1)$.

b) According to the position vector function,
\begin{gather*}
  x(t) = \cos 2t \\
  y(t) = \cos t \\
\end{gather*}

Therefore, we can find the relation between $x$ and $y$:
\begin{gather*}
  \cos 2t = 2 \cos^2 t - 1 \\
  x = 2 y^2 - 1 \\
\end{gather*}
which is a parabola in the $xy$-plane.

Since the range of $y$ is $y \in [-1, 1]$, and based on the characteristic of 
the parabola $x = 2 y^2 - 1$, the part of the curve that is actually traced out 
by $P$ is the part of the parabola between $(-1, 1)$ and $(1, 1)$.

c) According to the position vector function,
\begin{gather*}
  x(t) = t^2 + 1 \\
  y(t) = t^3 \\
\end{gather*}

Therefore, we can find the relation between $x$ and $y$:
\begin{equation*}
  y = (x - 1)^{\frac{3}{2}}
\end{equation*}
which is a symmetric graph related to the $x$ axis, in which the half above the 
$x$ axis is the graph of a power function.

Since the range of $x$ is $x \in [1, \infty)$, which is the domain of the 
function $y = (x - 1)^{\frac{3}{2}}$, the full curve is actually traced out by 
$P$.

d) According to the position vector function,
\begin{gather*}
  x(t) = \tan t \\
  y(t) = \sec t \\
\end{gather*}

Therefore, we can find the relation between $x$ and $y$:
\begin{gather*}
  \sec^2 t - \tan^2 t = 1 \\
  y^2 - x^2 = 1 \\
\end{gather*}
which is a hyperbola.

According to the graph of $x(t)$ and $y(t)$, the full hyperbola is actually 
traced out by $P$.

\bigskip

5. A string is wound clockwise around the circle of radius $a$ centered at the 
origin $O$; the initial position of the end $P$ of the string is $(a, 0)$. 
Unwind the string, always pulling it taut (so it stays tangent to the circle). 
Write parametric equations for the motion of $P$.

Solution:

Since the string is wound clockwise, when we unwind it, the direction is 
counter-clockwise. Also, since the string is unwound around a circle, it is 
natural to use the parameter $\theta$ to describe the motion of $P$, where 
$\theta$ is the corresponding angle of the unwound part of the string. Hence the 
length of the unwound part of the string is $a \theta$.

During the unwinding, the string is always pulled taut, hence there is always a 
point $P'$ on the circle which is the point of tangency between the circle and 
the unwound part of the string. The coordinates of $P'$ can be described as 
$(a \cos\theta, a \sin\theta)$.

Next we need to explore the direction of the vector $\vec{P'P}$ in order to 
describe the coordinates of $P$. According to the tangency relationship between 
the circle and the unwound part of the string,
\begin{gather*}
  \vec{P'P} \perp \vec{OP'} \\
  dir(\vec{OP'}) = \langle \cos\theta, \sin\theta \rangle \\
\end{gather*}

Therefore, we have the following system of equations
\begin{equation*}
  \begin{cases}
    dir(\vec{P'P}) \cdot \langle \cos\theta, \sin\theta \rangle = 0 \\
    |dir(\vec{P'P})| = 1 \\
  \end{cases}
\end{equation*}

After solving the above system of equations, there are two possible solutions:
\begin{gather*}
  dir(\vec{P'P}) = \langle \sin\theta, -\cos\theta \rangle \\
  dir(\vec{P'P}) = \langle -\sin\theta, \cos\theta \rangle \\
\end{gather*}

According to the geometric interpretation of unwinding the string 
counter-clockwise, the correct direction of the vector $\vec{P'P}$ is 
$\langle \sin\theta, -\cos\theta \rangle$.

Therefore, the parametric equation for the motion of $P$ is
\begin{equation*}
  \begin{cases}
    x(t) = a\cos\theta + a \theta \sin\theta \\
    y(t) = a\sin\theta - a \theta \cos\theta \\
  \end{cases}
\end{equation*}

\bigskip

7. The cycloid is the curve traced out by a fix point $P$ on a circle of radius 
$a$ which rolls along the $x$-axis in the positive direction, starting when $P$ 
is at the origin $O$. Find the vector function $OP$; use as variable the angle 
$\theta$ through which the circle has rolled.

Solution:

Let $C$ denote the center of the rolling circle. At a given value of $\theta$, 
suppose that the tangency point between the circle and the $x$ axis is $A$.

Since the circle is rolling along the $x$ axis without any slip, then
\begin{equation*}
  OA = a \theta
\end{equation*}

Also $AC = a$, hence the coordinates of $C$ at the given value of $\theta$ is 
\begin{equation*}
  C = (a \theta, a)
\end{equation*}

The direction of $\vec{CP}$ can be described by
\begin{gather*}
  dir(\vec{CP}) = \langle \cos(-\frac{\pi}{2} - \theta), \sin(-\frac{\pi}{2} - \theta) \rangle \\
  dir(\vec{CP}) = \langle -\sin\theta, -\cos\theta \rangle \\
\end{gather*}

Therefore, the parametric equation for the motion of $P$ is
\begin{equation*}
  \begin{cases}
    x(t) = a \theta - a \sin\theta \\
    y(t) = a - a \cos\theta \\
  \end{cases}
\end{equation*}

\begin{center}
\section*{Part II}
\end{center}

\bigskip

1. A circular disk of radius 2 has a dot marked at a point half-way between the 
center and the circumference. Denote this point by $P$. Suppose that the disk is 
tangent to the $x$-axis with the center initially at $(0, 2)$ and $P$ initially 
at $(0, 1)$ and that it starts to roll to the right on the $x$-axis at unit 
speed. Let $C$ be the curve traced out by the point $P$. \\
a) Make a sketch of what you think the curve $C$ will look like. \\
b) Use vectors to find the parametric equations for $\vec{OP}$ as a function of 
time $t$. \\
c) Open the 'Mathlet' Wheel (with link on course webpage) and set the parameters 
to view an animation of this particular motion problem. Then activate the 
'Trace' function to see a graph of the curve $C$. If this graph is substantially 
different from your hand sketch, sketch it also and then describe what led you 
to produce your first idea of the graph.


\end{document}