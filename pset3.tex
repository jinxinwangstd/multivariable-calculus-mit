\documentclass{article}
\usepackage[utf8]{inputenc}
\usepackage{amsmath}
\usepackage{amssymb}
\usepackage{amsthm}
\usepackage{tikz}
\setlength{\parindent}{0pt}

\newtheorem*{theorem}{Theorem}
\newtheorem*{definition}{Definition}
\newtheorem*{lemma}{Lemma}
\newtheorem*{corollary}{Corollary}
\newtheorem{example}{Example}
\newtheorem{trick}{Trick}
\newtheorem{question}{Question}

\title{Problem Set 3}
\author{}
\date{}

\begin{document}

\begin{center}
{\rmfamily\bfseries\Large 18.02 EXERCISES}

\vspace{25px}

{\rmfamily\bfseries\LARGE Problem Set 3: Parametric Equations for Curves}
\end{center}

\begin{center}
\section*{Part I}
\end{center}

\subsection*{Unit 1E Equations of Lines and Planes}

\bigskip

4. Where does the line through $(0, 1, 2)$ and $(2, 0, 3)$ intersect the plane 
$x + 4y + z = 4$?

Solution:

A vector along the line is $<2, -1, 1>$. Hence, the parametric equation of the 
line is
\begin{equation*}
  \begin{cases}
    x = x(t) = 0 + 2t = 2t \\
    y = y(t) = 1 - t \\
    z = z(t) = 2 + t \\
  \end{cases}
\end{equation*}

For the intersection point of the line and the plane, it satisfies
\begin{gather*}
  x(t) + 4y(t) + z(t) = 4 \\
  2t + 4(1 - t) + 2 + t = 4 \\
  -t = -2 \\
  t = 2 \\
\end{gather*}

Therefore, the coordinates of the intersection point is $(x(2), y(2), z(2))$, 
which is $(4, -1, 4)$.

\bigskip

7. Formulate a general method for finding the distance between two skew (i.e., 
non-intersecting) lines in space, and carry it out for two non-intersecting 
lines lying along the diagonals of two adjacent faces of the unit cube (place it 
in the first octant, with one vertex at the origin).

Solution:

7-1. Formulate a general method for finding the distance between two parallel 
planes in space:

Suppose that the equations of two parallel planes are respectively
\begin{gather*}
  n_1x + n_2y + n_3z = c_1 \\
  n_1x + n_2y + n_3z = c_2 \\
\end{gather*}
Then the vector $<n_1, n_2, n_3>$ is one of the normal vectors of the two 
planes. Therefore, we can construct the parametric equation of a line 
perpendicular to the two planes as
\begin{equation*}
  \begin{cases}
    x(t) = n_1t \\
    y(t) = n_2t \\
    z(t) = n_3t \\
  \end{cases}
\end{equation*}
Therefore, the two intersection points the line has with the two planes 
respectively satisfy
\begin{gather*}
  n_1x(t_1) + n_2y(t_1) + n_3z(t_1) = c_1 \\
  n_1^2t_1 + n_2^2t_1 + n_3^2t_1 = c_1 \\
  t_1 = \frac{c_1}{n_1^2 + n_2^2 + n_3^2} \\
  n_1x(t_2) + n_2y(t_2) + n_3z(t_2) = c_2 \\
  n_1^2t_2 + n_2^2t_2 + n_3^2t_2 = c_1 \\
  t_2 = \frac{c_2}{n_1^2 + n_2^2 + n_3^2} \\
\end{gather*}
The distance between two points in the line can be calculated as 
\begin{equation*}
  \begin{split}
    d &= \sqrt{(x(t_1) - x(t_2))^2 + (y(t_1) - y(t_2))^2 + (z(t_1) - z(t_2))^2} \\
      &= \sqrt{(n_1^2 + n_2^2 + n_3^2)(t_1 - t_2)^2} \\
      &= \sqrt{(n_1^2 + n_2^2 + n_3^2)(\frac{c_1 - c_2}{n_1^2 + n_2^2 + n_3^2})^2} \\
      &= \frac{|c_1 - c_2|}{\sqrt{n_1^2 + n_2^2 + n_3^2}} \\
  \end{split}
\end{equation*}

7-2. Formulate a general method for finding the distance between two parallel 
lines in space:

Suppose that the equations of two parallel lines are respectively
\begin{gather*}
  \begin{cases}
    x(t) = x_1 + n_1t \\
    y(t) = y_1 + n_2t \\
    z(t) = z_1 + n_3t \\
  \end{cases} \\
  \begin{cases}
    x(t) = x_2 + n_1t \\
    y(t) = y_2 + n_2t \\
    z(t) = z_2 + n_3t \\
  \end{cases} \\
\end{gather*}

The two parallel lines determine a plane in space. The normal vector of the 
plane can be calculated as the cross product of the two vectors:
\begin{equation*}
  \begin{split}
    &<x_1 - x_2, y_1 - y_2, z_1 - z_2> \times <n_1, n_2, n_3> \\
    &= \begin{vmatrix}
         \vec{i} & \vec{j} & \vec{k} \\
         n_1 & n_2 & n_3 \\
         (x_1 - x_2) & (y_1 - y_2) & (z_1 - z_2) \\
       \end{vmatrix} \\
    &= <n_2(z_1 - z_2) - n_3(y_1 - y_2), n_3(x_1 - x_2) - n_1(z_1 - z_2), n_1(y_1 - y_2) - n_2(x_1 - x_2)>
  \end{split}
\end{equation*}

Then we can find the vector perpendicular to the two parallel lines and lying in 
the plane determined by the two parallel lines by calculating the cross product 
of the normal vector and the line vector $<n_1, n_2, n_3>$:


\begin{center}
\section*{Part II}
\end{center}

\bigskip

1. A circular disk of radius 2 has a dot marked at a point half-way between the 
center and the circumference. Denote this point by $P$. Suppose that the disk is 
tangent to the $x$-axis with the center initially at $(0, 2)$ and $P$ initially 
at $(0, 1)$ and that it starts to roll to the right on the $x$-axis at unit 
speed. Let $C$ be the curve traced out by the point $P$. \\
a) Make a sketch of what you think the curve $C$ will look like. \\
b) Use vectors to find the parametric equations for $\vec{OP}$ as a function of 
time $t$. \\
c) Open the 'Mathlet' Wheel (with link on course webpage) and set the parameters 
to view an animation of this particular motion problem. Then activate the 
'Trace' function to see a graph of the curve $C$. If this graph is substantially 
different from your hand sketch, sketch it also and then describe what led you 
to produce your first idea of the graph.


\end{document}