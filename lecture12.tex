\documentclass{article}
\usepackage[utf8]{inputenc}
\usepackage{amsmath}
\usepackage{amssymb}
\usepackage{amsthm}
\usepackage{tikz}
\setlength{\parindent}{0pt}

\newtheorem*{theorem}{Theorem}
\newtheorem*{definition}{Definition}
\newtheorem*{lemma}{Lemma}
\newtheorem*{corollary}{Corollary}
\newtheorem{example}{Example}
\newtheorem*{trick}{Trick}
\newtheorem*{question}{Question}

\title{Lecture 12: Gradient}
\author{}
\date{}

\begin{document}
    
\maketitle

\section{Gradient}

\subsection{Introduce Gradient}

According to the chain rule, suppose that there is a function $w = w(x, y, z)$, 
where $x = x(t)$, $y = y(t)$, and $z = z(t)$, then
\begin{gather*}
  \begin{split}
    \frac{dw}{dt} &= w_x \frac{dx}{dt} + w_y \frac{dy}{dt} + w_z \frac{dz}{dt} \\
                  &= <w_x, w_y, w_z> \cdot <\frac{dx}{dt}, \frac{dy}{dt}, \frac{dz}{dt}> \\
                  &= \nabla w \cdot \frac{d\vec{r}}{dt} \\
  \end{split} \\
  \nabla w = <w_x, w_y, w_z> \\
  \frac{d\vec{r}}{dt} = <\frac{dx}{dt}, \frac{dy}{dt}, \frac{dz}{dt}> \\
\end{gather*}
The vector $<w_x, w_y, w_z>$ is called gradient, denoted as $\nabla w$. 
Gradients on different positions can have different directions and magnitude.

\subsection{A Property of Gradient}

\begin{theorem}
  The gradient of a function is always perpendicular to the level surfaces of 
  the function, where a level surface means the set of points on the function 
  domain whose function values are equal to a constant.
\end{theorem}
Notice that the level surfaces refer to points in the function domain, rather 
than on the function graph. It doesn't include the dimension of the function 
value.

\begin{example}
  Examine the relation between the gradient and the level surfaces for the 
  function $w = a_1x + a_2y + a_3z$.

  \begin{gather*}
    \frac{\partial w}{\partial x} = a_1 \\
    \frac{\partial w}{\partial y} = a_2 \\
    \frac{\partial w}{\partial z} = a_3 \\
    \nabla w = <a_1, a_2, a_3> \\
  \end{gather*}

  The level surfaces of the function $w = a_1x + a_2y + a_3z = c$, where $c$ is 
  any constant, are a series of planes parallel to each other. For any level 
  surfaces of the function, the gradient $\nabla w$ is a normal vector to it. 
  Therefore, the gradient is always perpendicular to the level surfaces for the 
  function $w = a_1x + a_2y + a_3z$.
\end{example}

\begin{example}
  Examine the relation between the gradient and the level surfaces for the 
  function $w = x^2 + y^2$.

  \begin{gather*}
    \frac{\partial w}{\partial x} = 2x \\
    \frac{\partial w}{\partial y} = 2y \\
    \nabla w = <2x, 2y> = <x, y> \\
  \end{gather*}

  The level surfaces of the function $w = x^2 + y^2 = c$, where $c$ is any 
  constant, are a series of circles whose centers are the origin. At any point 
  $(x_0, y_0)$ in the function domain, the gradient vector is $<x_0, y_0>$, 
  which has the same direction as the radius through $(x_0, y_0)$ on the level 
  surface $x^2 + y^2 = x_0^2 + y_0^2$. A circle's radius is always perpendicular 
  to the circle, hence the gradient vector is perpendicular to the level surface. 
  Therefore, the gradient is always perpendicular to the level surfaces for the 
  function $w = x^2 + y^2$.
\end{example}

Proof of the theorem:

Suppose there is a curve $\vec{r} = \vec{r}(t)$ that always stays on a level 
surface of a function $f$. The velocity vector $\frac{d\vec{r}}{dt}$ is tangent 
to the curve, and therefore tangent to the level surface on which the curve 
stays.

By chain rule, on any level surfaces it satisfies the following equation:
\begin{equation*}
  \frac{df}{dt} = \nabla f \cdot \frac{d\vec{r}}{dt} = 0
\end{equation*}
Therefore, at any points of a curve that stays on a level surface:
\begin{equation*}
  \nabla f \perp \frac{d\vec{r}}{dt}
\end{equation*}
The same reasoning applies to any curves on a level surface, so at any points on 
a level surface, the gradient $\nabla f$ is perpendicular to velocity vectors of 
every direction, which are tangent to the level surface. Therefore, at any 
points on a level surface, the gradient $\nabla f$ is perpendicular to the 
tangent plane to the level surface.

\subsection{Application of Gradient}

We can use gradients to find the equation of the tangent line of a function 
graph at any points.

\begin{example}
  Find the equation of the tangent plane to the surface $x^2 + y^2 - z^2 = 4$ at 
  the point (2, 1, 1).

  Solution:

  The surface $x^2 + y^2 - z^2 = 4$ is a level surface of the function 
  $w = x^2 + y^2 - z^2$.
  \begin{gather*}
    \frac{\partial w}{\partial x} = 2x \\
    \frac{\partial w}{\partial y} = 2y \\
    \frac{\partial w}{\partial z} = -2z \\
    \nabla w = <2x, 2y, -2z> = <x, y, -z> \\
  \end{gather*}
  Hence the gradient of the function at the point (2, 1, 1) is $<2, 1, -1>$. 
  According to the property of gradients, the gradient is perpendicular to the 
  tangent plane of the corresponding level surface at the point. Therefore, the 
  equation of the tangent plane is 
  \[ 2x + y - z = c, \textnormal{where c is a constant} \]
  Then we can substitute the coordinate of the point (2, 1, 1) into the plane 
  equation:
  \[ c = 2 \times 2 + 1 \times 1 - 1 \times 1 = 4 \]
  Therefore, the equation of the tangent plane to the surface 
  $x^2 + y^2 - z^2 = 4$ at the point (2, 1, 1) is
  \[ 2x + y - z = 4 \]
\end{example}

\section{Directional Derivatives}

\subsection{Introduce Directional Derivatives}

How to calculate the partial derivative towards any direction $\vec{u}$, rather 
than only along the $x$ axis and $y$ axis?

Suppose there is a multivariable function $f(x, y)$, and an unit vector 
$\hat{\vec{u}} = <a, b>$. The partial derivative along a direction is the rate 
of change of the function value $f(x, y)$ over the arclength, hence we need the 
differential of the arclength along the direction of $\hat{\vec{u}}$, which is 
denoted by $ds$. 

By chain rule, we have
\begin{gather*}
  df = \frac{\partial f}{\partial x}dx + \frac{\partial f}{\partial y}dy \\
  \frac{df}{ds} = \frac{\partial f}{\partial x}\frac{dx}{ds} + \frac{\partial f}{\partial y}\frac{dy}{ds} \\
  \frac{df}{ds} = <\frac{\partial f}{\partial x}, \frac{\partial f}{\partial y}> \cdot <\frac{dx}{ds}, \frac{dy}{ds}> \\
  \frac{df}{ds} = \nabla f \cdot \frac{d\vec{r}}{ds} \\
\end{gather*}
Therefore, we need to describe the position vector $\vec{r}$ along the straight 
line trajectory with the direction $\vec{u}$ using the parameter $s$, the 
arclength.
\begin{gather*}
  \begin{cases}
    x(s) = x_0 + as \\
    y(s) = y_0 + bs \\
  \end{cases} \\
  \frac{d\vec{r}}{ds} = <a, b> \\
  \frac{d\vec{r}}{ds} = \vec{u} \\
\end{gather*}
Therefore,
\begin{equation*}
  \begin{split}
    \frac{df}{ds} &= \nabla f \cdot \frac{d\vec{r}}{ds} \\
                  &= \nabla f \cdot \vec{u} \\
  \end{split}
\end{equation*}
which is the formula for directional derivatives.

\begin{definition}
  The instantaneous rate of change of a multivariable function $f$ at a given 
  direction $\vec{u}$ is called the directional derivative, whose notation is 
  $\frac{df}{ds}|_{\vec{u}}$. The formula of directional derivatives is 
  \[ \frac{df}{ds}|_{\vec{u}} = \nabla f \cdot \vec{u} \]
\end{definition}

\subsection{Geometric Interpretation of Directional Derivatives}

For a multivariable function $f$ with two independent variables, the geometric 
interpretation of the function toward a direction $\vec{u}$ is the slope of the 
slice of the function graph by a vertical plane parallel to $\vec{u}$.

\begin{example}
  Verify the formula of directional derivatives by finding the partial 
  derivative along the $x$ axis.

  Suppose there is a multivariable function $f(x, y)$.
  \begin{equation*}
    \begin{split}
      \frac{\partial f}{\partial x} &= \nabla f \cdot \vec{i} \\
                                    &= <\frac{\partial f}{\partial x}, \frac{\partial f}{\partial y}> \cdot \vec{i} \\
                                    &= \frac{\partial f}{\partial x} \\
    \end{split}
  \end{equation*}
\end{example}

\subsection{Max/Min of Directional Derivatives}

According to the geometric interpretation of dot product of vectors:
\begin{equation*}
  \begin{split}
    \frac{df}{ds}|_{\vec{u}} &= \nabla f \cdot \vec{u} \\
                             &= |\nabla f| \times |\vec{u}| \times \cos\theta \\
                             &= \cos\theta \times |\nabla f| \\
  \end{split}
\end{equation*}
We can see that the directional derivative changes as the angle $\theta$ between 
its direction and the gradient vector changes. It is obvious that at a given 
point:
\begin{itemize}
  \item The directional derivative with the maximum value has the same direction 
    as the gradient, and the maximum value is the magnitude of the gradient. 
    Since the directional derivative along the direction of the gradient is 
    positive and has the maximum value, the gradient points to the direction 
    where the function value increases fastest.
  \item The directional derivative with the minimum value has the opposite 
    direction to the gradient, and the minimum value is the negative value of 
    the magnitude of the gradient.
\end{itemize}

\end{document}