\documentclass{article}
\usepackage[utf8]{inputenc}
\usepackage{amsmath}
\usepackage{amssymb}
\usepackage{amsthm}
\usepackage{tikz}
\setlength{\parindent}{0pt}

\newtheorem*{theorem}{Theorem}
\newtheorem*{definition}{Definition}
\newtheorem*{lemma}{Lemma}
\newtheorem*{corollary}{Corollary}
\newtheorem{example}{Example}
\newtheorem{trick}{Trick}
\newtheorem{question}{Question}

\title{Lecture 12: Gradient}
\author{}
\date{}

\begin{document}
    
\maketitle

\section{Gradient}

According to the chain rule, suppose that there is a function $w = w(x, y, z)$, 
where $x = x(t)$, $y = y(t)$, and $z = z(t)$, then
\begin{gather*}
  \begin{split}
    \frac{dw}{dt} &= w_x \frac{dx}{dt} + w_y \frac{dy}{dt} + w_z \frac{dz}{dt} \\
                  &= <w_x, w_y, w_z> \cdot <\frac{dx}{dt}, \frac{dy}{dt}, \frac{dz}{dt}> \\
                  &= \nabla w \cdot \frac{d\vec{r}}{dt} \\
  \end{split} \\
  \nabla w = <w_x, w_y, w_z> \\
\end{gather*}
The vector $<w_x, w_y, w_z>$ is called gradient, denoted as $\nabla w$.

\section{Directional Derivatives}

\end{document}