\documentclass{article}
\usepackage[utf8]{inputenc}
\usepackage{amsmath}
\usepackage{amssymb}
\usepackage{amsthm}
\usepackage{bm}
\usepackage{tikz}
\setlength{\parindent}{0pt}

\newtheorem*{theorem}{Theorem}
\newtheorem*{definition}{Definition}
\newtheorem*{lemma}{Lemma}
\newtheorem*{corollary}{Corollary}
\newtheorem{example}{Example}
\newtheorem*{trick}{Trick}
\newtheorem*{question}{Question}

\newcommand{\uvec}[1]{\boldsymbol{\hat{\textbf{#1}}}}

\title{Lecture 19: Vector Fields}
\author{}
\date{}

\begin{document}
    
\maketitle

\section{Definition of Vector Fields}

A vector field can be described with such a formula:
\begin{equation*}
  \vec{F} = M \uvec{\i} + N \uvec{\j}
\end{equation*}
where $M$ and $N$ are functions of coordinates.

In a vector field, at each point $(x, y)$ there is a corresponding vector. In 
other words, vectors in a vector field are a function of the position.

Real world examples of a vector field:
\begin{itemize}
  \item Velocity field in fluid $\vec{v}$.
  \item Force field $\vec{F}$.
\end{itemize}

\section{Plot of Vector Fields}

Plot of a vector field enables us to understand the vector field in a concrete 
way and probably provides some important insights.

\begin{example}
  Plot the vector field $\vec{F} = 2 \uvec{\i} + \uvec{j}$.

  TODO(jinxinwang): add the graph.
\end{example}

\begin{example}
  Plot the vector field $\vec{F} = x \uvec{\i}$.

  TODO(jinxinwang): add the graph.

  Notice that we usually use the length of arrows in the graph to relatively 
  indicate the magnitudes of vectors in a vector field.
\end{example}

\begin{example}
  Plot the vector field $\vec{F} = x \uvec{\i} + y \uvec{\j}$.

  TODO(jinxinwang): add the graph.
\end{example}

\begin{example}
  Plot the vector field $\vec{F} = -y \uvec{\i} + x \uvec{\j}$.

  TODO(jinxinwang): add the graph.

  This is the vector field of the uniform rotation at unit angular velocity.
\end{example}

\begin{question}
  Vectors in mathematics have no starting point. However, in vector fields each 
  vector is associated with a starting point. Is it contradictory with the 
  definition of vectors?
\end{question}

\section{Line Integrals}

\subsection{Definition of Line Integrals}

The motivation of introducing line integrals is to calculate the work done along 
a trajectory in a force field.

Recall from physics:
\begin{equation*}
  W = \vec{F} \cdot \Delta \vec{r}
\end{equation*}

For a motion with changing force or curved trajectory, to calculate the total 
work, we need to apply the idea of integrals. We can divide the trajectory into 
many small pieces, with each piece as $\Delta \vec{r}$. For a piece of small 
trajectory $\Delta \vec{r}_i$, the work is
\begin{equation*}
  \Delta W = \vec{F} \cdot \Delta \vec{r}_i
\end{equation*}

With the number of divided pieces approaching infinity and each piece of 
trajectory approaching infinitesimal, we can add them up to get the total work:
\begin{equation*}
  \begin{split}
    W &= \lim_{\Delta \vec{r}_i \to \vec{0}} \sum_i \vec{F} \cdot \Delta \vec{r}_i \\
      &= \int_C \vec{F} \cdot d\vec{r} \\
  \end{split}
\end{equation*}
which is the definition of a line integral.

\subsection{Calculation of Line Integrals}

To calculate a line integral,
\begin{equation*}
  \begin{split}
    W &= \int_C \vec{F} \cdot d\vec{r} \\
      &= \int_C \vec{F} \cdot \frac{d\vec{r}}{dt} \cdot dt \\
      &= \int_{t_1}^{t_2} \vec{F} \cdot \frac{d\vec{r}}{dt} \cdot dt \\
  \end{split}
\end{equation*}

Notice that in the definition of line integrals, $\vec{F}$ and the trajectory 
$C$ are independent from each other.

\begin{example}
  Suppose that there is a force field as $\vec{F} = -y \uvec{\i} + x \uvec{\j}$, 
  and a trajectory $C$ as
  \begin{equation*}
    \begin{cases}
      x = t \\
      y = t^2 \\
    \end{cases}
  \end{equation*}
  where $0 \leq t \leq 1$. Calculate the work along the trajectory in the force 
  field.

  Solution:

  We can use the line integral to calculate the total work.
  \begin{equation*}
    \begin{split}
      W &= \int_C \vec{F} \cdot d\vec{r} \\
        &= \int_0^1 \vec{F} \cdot \frac{d\vec{r}}{dt} dt \\
    \end{split}
  \end{equation*}

  According to the problem description,
  \begin{gather*}
    \vec{F} = \langle -y, x \rangle = \langle -t^2, t \rangle \\
    \vec{r} = \langle x, y \rangle = \langle t, t^2 \rangle \\
    \frac{d\vec{r}}{dt} = \langle 1, 2t \rangle \\
  \end{gather*}
\end{example}

Therefore,
\begin{equation*}
  \begin{split}
    W &= \int_0^1 \vec{F} \cdot \frac{d\vec{r}}{dt} dt \\
      &= \int_0^1 \langle -t^2, t \rangle \cdot \langle 1, 2t \rangle dt \\
      &= \int_0^1 t^2 dt \\
      &= \frac{t^3}{3}|_0^1 \\
      &= \frac{1}{3} \\
  \end{split}
\end{equation*}

\subsection{Another Way of Calcuating Line Integrals}

Another way to look at the calculation of line integrals. For a line integral 
$\int_C \vec{F} \cdot d\vec{r}$, we have
\begin{gather*}
  \vec{F} = \langle M, N \rangle \\
  d\vec{r} = \langle dx, dy \rangle \\
\end{gather*}

\begin{question}
  Why the following equation holds $d\vec{r} = \langle dx, dy \rangle$?
\end{question}

Therefore,
\begin{equation*}
  \begin{split}
    \int_C \vec{F} \cdot d\vec{r} &= \int_C \langle M, N \rangle \cdot \langle dx, dy \rangle \\
                                  &= \int_C M dx + N dy \\
  \end{split}
\end{equation*}

To evaluate the expression $\int_C M dx + N dy$, we need to express $x$ and $y$ 
with a single variable, and do substitution in the integrand.

\begin{question}
  Why is it incorrect to evaluate $\int_C M dx + N dy$ by the following method:
  \begin{equation*}
    \begin{split}
      \int_C M dx + N dy &= \int_C M dx + \int_C N dy \\
                         &= \int_{x_1}^{x_2} M dx + \int_{y_1}^{y_2} N dy \\
    \end{split}
  \end{equation*}

  Answer:

  I can think of a counterexample to prove the above method can produce 
  incorrect result. Consider the following case:

  There is a force field as $\vec{F} = -y \uvec{\i} + x \uvec{\j}$, and a trajectory as
  \begin{equation*}
    \begin{cases}
      x = \sin t \\
      y = \cos t \\
    \end{cases}
  \end{equation*}
  where $0 \leq t \leq 2 \pi$. 

  Obviously the total along the described trajectory in the force field is not 
  $0$ because at any points of the trajectory, $\vec{F}$ and $d\vec{r}$ have the 
  same direction. However, using the above method
  \begin{equation*}
    \int_C M dx + N dy = \int_{x_1}^{x_2} M dx + \int_{y_1}^{y_2} N dy
  \end{equation*}
  the result would be $0$ because $x_1 = x_2$ and $y_1 = y_2$.

  In summary, the incorrect evaluation method cannot correctly evaluate line 
  integrals with back-and-forth trajectory.
\end{question}

\begin{example}
  Suppose that there is a force field as $\vec{F} = -y \uvec{\i} + x \uvec{\j}$, 
  and a trajectory $C$ as
  \begin{equation*}
    \begin{cases}
      x = t \\
      y = t^2 \\
    \end{cases}
  \end{equation*}
  where $0 \leq t \leq 1$. Calculate the work along the trajectory in the force 
  field.

  Solution:

  We can use line intergrals to calculate the total work:
  \begin{equation*}
    \begin{split}
      W &= \int_C \vec{F} \cdot d\vec{r} \\
        &= \int_C \langle -y, x \rangle \cdot \langle dx, dy \rangle \\
        &= \int_C -ydx + xdy \\
    \end{split}
  \end{equation*}

  According to the problem description,
  \begin{gather*}
    x = t \\
    y = t^2 \\
    dx = 1dt \\
    dy = 2t dt \\
  \end{gather*}

  Therefore,
  \begin{equation*}
    \begin{split}
      W &= \int_C -ydx + xdy \\
        &= \int_0^1 -t^2 \cdot 1 \cdot dt + t \cdot 2t \cdot dt \\
        &= \int_0^1 t^2 dt \\
        &= \frac{1}{3} \\
    \end{split}
  \end{equation*}

  Note that the result of a line integral $\int_C \vec{F} \cdot d\vec{r}$ 
  doesn't depend on the parameterization, in other words, the substitution of 
  any parameters should produce the same results. For the above example, we can 
  also apply the following parameterization, which would yield the same result.
  \begin{gather*}
    x = \sin\theta \\
    y = \sin^2 \theta \\
    0 \leq \theta \leq \frac{\pi}{2} \\
  \end{gather*}

  Therefore, we should choose the parameterization which is the easiest to 
  calculate.
\end{example}

\subsection{Geometric Approach}

\begin{gather*}
  \vec{v} = \frac{d\vec{r}}{dt} = \vec{T} |\frac{ds}{dt}| = \langle \frac{dx}{dt}, \frac{dy}{dt} \rangle \\
  d\vec{r} = \vec{T} ds = \langle dx, dy \rangle \\
  \int_C \vec{F} \cdot d\vec{r} = \int_C \vec{F} \cdot \vec{T} \cdot ds \\
\end{gather*}

\end{document}