\documentclass{article}
\usepackage[utf8]{inputenc}
\usepackage{amsmath}
\usepackage{amssymb}
\usepackage{amsthm}
\usepackage{bm}
\usepackage{tikz}
\setlength{\parindent}{0pt}

\newtheorem*{theorem}{Theorem}
\newtheorem*{definition}{Definition}
\newtheorem*{lemma}{Lemma}
\newtheorem*{corollary}{Corollary}
\newtheorem{example}{Example}
\newtheorem*{trick}{Trick}
\newtheorem*{question}{Question}

\newcommand{\uvec}[1]{\boldsymbol{\hat{\textbf{#1}}}}

\title{Lecture 19: Vector Fields}
\author{}
\date{}

\begin{document}
    
\maketitle

\section{Definition of Vector Fields}

A vector field can be described with such a formula:
\begin{equation*}
  \vec{F} = M \uvec{\i} + N \uvec{\j}
\end{equation*}
where $M$ and $N$ are functions of coordinates.

In a vector field, at each point $(x, y)$ there is a corresponding vector. In 
other words, vectors in a vector field are a function of the position.

Real world examples of a vector field:
\begin{itemize}
  \item Velocity field in fluid $\vec{v}$.
  \item Force field $\vec{F}$.
\end{itemize}

\section{Plot of Vector Fields}

Plot of a vector field enables us to understand the vector field in a concrete 
way and probably provides some important insights.

\begin{example}
  Plot the vector field $\vec{F} = 2 \uvec{\i} + \uvec{j}$.
\end{example}

\begin{example}
  Plot the vector field $\vec{F} = x \uvec{\i}$.
\end{example}

\end{document}