\documentclass{article}
\usepackage[utf8]{inputenc}
\usepackage{amsmath}
\usepackage{amssymb}
\usepackage{amsthm}
\usepackage{tikz}
\setlength{\parindent}{0pt}

\newtheorem*{theorem}{Theorem}
\newtheorem*{definition}{Definition}
\newtheorem*{lemma}{Lemma}
\newtheorem*{corollary}{Corollary}
\newtheorem{example}{Example}
\newtheorem*{trick}{Trick}
\newtheorem*{question}{Question}

\title{Lecture 11: Chain Rule}
\author{}
\date{}

\begin{document}
    
\maketitle

Now with partial derivatives, we can learn more tools to study multivariable 
functions.

\section{Total Differential}

Total differential describes the rate of change of the value of a function in 
terms of all variables.

Suppose there is a function $f(x, y, z)$:
\begin{gather*}
  \begin{split}
    df &= f_xdx + f_ydy + f_zdz \\
       &= \frac{\partial f}{\partial x}dx + \frac{\partial f}{\partial y}dy + \frac{\partial f}{\partial z}dz \\
  \end{split}
\end{gather*}
Notice that $df$ is not a real number, nor is it $\Delta f$. It is an abstract 
notion. So are $dx$, $dy$, and $dz$.

\begin{question}
  What is the intuitive way to understand the total differential formula? It is 
  hard to reason about whether it is valid or not.
\end{question}

\begin{question}
  If we read the total differential formula from left to right, it seems that we 
  distribute the change in the function value into different dimensions 
  according to partial derivatives. Is this the correct way to understand this 
  formula?
\end{question}

How do we use the total differential?
\begin{itemize}
  \item The total differential encodes how changes in independent variables 
    affect the function value.
  \item We can replace $dx$, $dy$, and $dz$ with small variations $\Delta x$, 
    $\Delta y$, and $\Delta z$ to get approximated change of the function value, 
    which is the approximation formula:
    \begin{equation*}
      \Delta f \approx f_x \Delta x + f_y \Delta y + f_z \Delta z
    \end{equation*}
  \item We can divide both sides of the total differential by the differential 
    of a common parameter, such as $dt$, to get the rate of change in terms of 
    this common parameter.
\end{itemize}

\section{Chain Rule}

Actually, the third usage of the total differential described above derives the 
formula of chain rule in multivariable functions.

Suppose there is a function $f(x, y, z)$, and there exists $x = x(t)$, 
$y = y(t)$, and $z = z(t)$, then
\begin{equation*}
  \begin{split}
    \frac{df}{dt} &= f_x \frac{dx}{dt} + f_y \frac{dy}{dt} + f_z \frac{dz}{dt} \\
                  &= \frac{\partial f}{\partial x}\frac{dx}{dt} + \frac{\partial f}{\partial y}\frac{dy}{dt} + \frac{\partial f}{\partial z}\frac{dz}{dt} \\
  \end{split}
\end{equation*}

Why is the formula of chain rule valid?

We can start with the approximation formula. We know that
\begin{equation*}
  \Delta f \approx f_x \Delta x + f_y \Delta y + f_z \Delta z
\end{equation*}
Then we can divide both sides of the equation by a change of the parameter $t$, 
which is $\Delta t$:
\begin{equation*}
  \frac{\Delta f}{\Delta t} \approx f_x \frac{\Delta x}{\Delta t} + f_y \frac{\Delta y}{\Delta t} + f_z \frac{\Delta z}{\Delta t}
\end{equation*}
With $\Delta t \to 0$, according to the definition of derivatives:
\begin{gather*}
  \lim_{\Delta t \to 0}\frac{\Delta f}{\Delta t} = \frac{df}{dt} \\
  \lim_{\Delta t \to 0}\frac{\Delta x}{\Delta t} = \frac{dx}{dt} \\
  \lim_{\Delta t \to 0}\frac{\Delta y}{\Delta t} = \frac{dy}{dt} \\
  \lim_{\Delta t \to 0}\frac{\Delta z}{\Delta t} = \frac{dz}{dt} \\
\end{gather*}
and the approximation becomes more and more accurate. Eventually
\begin{equation*}
  \frac{df}{dt} = f_x \frac{dx}{dt} + f_y \frac{dy}{dt} + f_z \frac{dz}{dt}
\end{equation*}

\begin{example}
  Given that $w = f(x, y, z) = x^2y + z$, $x = t$, $y = e^t$, and $z = \sin t$, 
  find $\frac{dw}{dt}$.

  Solution:

  \begin{gather*}
    \frac{dw}{dt} = f_x \frac{dx}{dt} + f_y \frac{dy}{dt} + f_z \frac{dz}{dt} \\
    f_x = 2xy = 2te^t \\
    f_y = x^2 = t^2 \\
    f_z = 1 \\
    \frac{dx}{dt} = 1 \\
    \frac{dy}{dt} = e^t \\
    \frac{dz}{dt} = \cos t \\
    \begin{split}
      \frac{dw}{dt} &= f_x \frac{dx}{dt} + f_y \frac{dy}{dt} + f_z \frac{dz}{dt} \\
                    &= 2t \cdot e^t \cdot 1 + t^2 \cdot e^t + 1 \cdot \cos t \\
                    &= 2te^t + t^2e^t + \cos t \\
    \end{split} \\
  \end{gather*}

  We can verify the result by first find the formula of $w$ with respect to $t$, 
  and then calculate its derivative:
  \begin{gather*}
    \begin{split}
      w &= x^2y + z \\
        &= t^2e^t + \sin t \\
    \end{split} \\
    \begin{split}
      \frac{dw}{dt} &= \frac{d(t^2e^t + \sin t)}{dt} \\
                    &= 2te^t + t^2e^t + \cos t \\
    \end{split} \\
  \end{gather*}
  The results from the two methods are the same.
\end{example}

We can use chain rule of multivariable functions to derive the product and 
quotient rule of derivatives of single-variable functions:
\begin{itemize}
  \item Product rule:
    \begin{gather*}
      f = u \cdot v \\
      \begin{split}
        \frac{df}{dt} &= \frac{d(u \cdot v)}{dt} \\
                      &= f_u \frac{du}{dt} + f_v \frac{dv}{dt} \\
                      &= \frac{du}{dt} v + u \frac{dv}{dt} \\
      \end{split} \\
    \end{gather*}
  \item Quotient rule:
    \begin{gather*}
      f = \frac{u}{v} \\
      \begin{split}
        \frac{df}{dt} &= \frac{d(\frac{u}{v})}{dt} \\
                      &= f_u \frac{du}{dt} + f_v \frac{dv}{dt} \\
                      &= \frac{1}{v} \cdot \frac{du}{dt} + \frac{-u}{v^2} \cdot \frac{dv}{dt} \\
                      &= \frac{\frac{du}{dt}v - u\frac{dv}{dt}}{v^2} \\
      \end{split} \\
    \end{gather*}
\end{itemize}

\section{Chain Rule with More Variables}

\subsection{Formula of Chain Rule with More Variable}

Suppose there is a function $w = f(x, y)$, where $x = x(u, v)$, and 
$y = y(u, v)$, how to find the total differential of $w$ with respect to $u$ and 
$v$?

According to the total differential formula,
\begin{equation*}
  dw = \frac{\partial f}{\partial x}dx + \frac{\partial f}{\partial y}dy
\end{equation*}
If we apply the total differential formula to the independent variables, then we 
get
\begin{gather*}
  dx = \frac{\partial x}{\partial u}du + \frac{\partial x}{\partial v}dv \\
  dy = \frac{\partial y}{\partial u}du + \frac{\partial y}{\partial v}dv \\
\end{gather*}
Therefore
\begin{gather*}
  \begin{split}
    dw &= \frac{\partial f}{\partial x}dx + \frac{\partial f}{\partial y}dy \\
       &= \frac{\partial f}{\partial x}(\frac{\partial x}{\partial u}du + \frac{\partial x}{\partial v}dv) + \frac{\partial f}{\partial y}(\frac{\partial y}{\partial u}du + \frac{\partial y}{\partial v}dv) \\
       &= (\frac{\partial f}{\partial x} \cdot \frac{\partial x}{\partial u} + \frac{\partial f}{\partial y} \cdot \frac{\partial y}{\partial u})du + (\frac{\partial f}{\partial x} \cdot \frac{\partial x}{\partial v} + \frac{\partial f}{\partial y} \cdot \frac{\partial y}{\partial v})dv \\
  \end{split} \\
  \frac{\partial w}{\partial u} = \frac{\partial f}{\partial x} \cdot \frac{\partial x}{\partial u} + \frac{\partial f}{\partial y} \cdot \frac{\partial y}{\partial u} \\
  \frac{\partial w}{\partial v} = \frac{\partial f}{\partial x} \cdot \frac{\partial x}{\partial v} + \frac{\partial f}{\partial y} \cdot \frac{\partial y}{\partial v} \\
\end{gather*}
That's the chain rule of multivariable functions where the independent variables 
are also multivariable functions of some parameters.

Notice that
\begin{equation*}
  \frac{\partial f}{\partial x} \cdot \frac{\partial x}{\partial u} \neq \frac{\partial f}{\partial u}
\end{equation*}
Since they are only part of the change, rather than the full change $dx$, they 
cannot be cancelled.

\subsection{Intuition behind the Formula}

Why the formula is valid?

Suppose that there is a small change in the independent variable $u$, the change 
affects the values of both $x$ and $y$, described by their corresponding partial 
derivatives. Then the change of $x$ and $y$ affects the value of $w$ eventually, 
and the relation is described by the total differential formula.

So the chain rule actually describes the propogation relation between 
independent variables and some function values.

\begin{example}
  Conversion between Cartesian coordinate system and polar coordinate system.

  Suppose there is a function formula described in Cartesian coordinate system 
  $f(x, y)$.
  \begin{gather*}
    x = r \cos\theta \\
    y = r \sin\theta \\
    \begin{split}
      \frac{\partial f}{\partial r} &= \frac{\partial f}{\partial x}\frac{\partial x}{\partial r} + \frac{\partial f}{\partial y}\frac{\partial y}{\partial r} \\
                                    &= \frac{\partial f}{\partial x} \cos\theta + \frac{\partial f}{\partial y} \sin\theta \\
    \end{split} \\
    \begin{split}
      \frac{\partial f}{\partial \theta} &= \frac{\partial f}{\partial x}\frac{\partial x}{\partial \theta} + \frac{\partial f}{\partial y}\frac{\partial y}{\partial \theta} \\
                                         &= -r \sin\theta \frac{\partial f}{\partial x} + r \cos\theta \frac{\partial f}{\partial y} \\
    \end{split} \\
  \end{gather*}
\end{example}

\end{document}