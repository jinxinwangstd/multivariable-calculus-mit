\documentclass{article}
\usepackage[utf8]{inputenc}
\usepackage{amsmath}
\usepackage{amssymb}
\usepackage{amsthm}
\usepackage{tikz}
\setlength{\parindent}{0pt}

\newtheorem*{theorem}{Theorem}
\newtheorem*{definition}{Definition}
\newtheorem*{lemma}{Lemma}
\newtheorem*{corollary}{Corollary}
\newtheorem{example}{Example}
\newtheorem{trick}{Trick}
\newtheorem{question}{Question}

\title{Lecture 10: Second Derivative Test}
\author{}
\date{}

\begin{document}
    
\maketitle

How do we know the global maximum or minimum values of a function?

Similar to single-variable functions, the global max/min value of a 
multivariable function can be the function value of:
\begin{itemize}
  \item Either a local max/min point.
  \item Or the boundary or infinity.
\end{itemize}

We know that local max/min points are critical points. Then how do we determine 
the type of a critical point? Except the method we talked about in the previous 
lecture, there is a systematic way to do it, called 
\textbf{Second Derivative Test}.

\section{A Simple Case: Quadratic Functions}

Suppose that we have a quadratic multivariable function 
\begin{equation*}
  z = f(x, y) = ax^2 + bxy + cy^2, a \neq 0
\end{equation*}
We can calculate its partial derivatives and find that $(0, 0)$ is a critical 
point of the function:
\begin{gather*}
  \frac{\partial f}{\partial x} = 2ax + by \\
  \frac{\partial f}{\partial y} = 2cy + bx \\
  \frac{\partial f}{\partial x}(0, 0) = 0 \\
  \frac{\partial f}{\partial y}(0, 0) = 0 \\
\end{gather*}
The value of the function on the critical point $(0, 0)$ is 
\begin{equation*}
  z = f(0, 0) = a \cdot 0^2 + b \cdot 0 \cdot 0 + c \cdot 0^2 = 0
\end{equation*}
Let's try to find the type of this critical point.
\begin{gather*}
  \begin{split}
    z &= ax^2 + bxy + cy^2 \\
      &= a(x^2 + \frac{b}{a}xy) + cy^2 \\
      &= a(x^2 + \frac{b}{a}xy + \frac{b^2}{4a^2}y^2) + (c - \frac{b^2}{4a})y^2 \\
      &= a(x + \frac{b}{2a}y)^2 + \frac{1}{4a}(4ac - b^2)y^2 \\
      &= \frac{1}{4a}[4a^2(x + \frac{b}{2a}y)^2 + (4ac - b^2)y^2] \\
  \end{split}
\end{gather*}
We can see that the quadratic function is actually a sum of two terms:
\begin{itemize}
  \item $4a^2(x + \frac{b}{2a}y)^2$, which is always greater than or equal to 0.
  \item $(4ac - b^2)y^2$, which is either always greater than or equal to 0, or 
    always less than or equal to 0, depending on the sign of $(4ac - b^2)$.
\end{itemize}
Therefore, we can discuss different cases here:
\begin{itemize}
  \item If $4ac - b^2 < 0$, the two terms would have different signs, no matter 
    what sign the factor $\frac{1}{4a}$ has. Therefore, starting from the 
    critical point $(0, 0)$, in one direction (perpendicular to the direction 
    represented by the positive term) the function value would increase, and in 
    another direction (perpendicular to the direction represented by the 
    negative term) the function value would decrease. Therefore, the critical 
    point $(0, 0)$ is actually a \textbf{saddle point}.
  \item If $4ac - b^2 = 0$, the function is actually $a(x + \frac{b}{2a}y)^2$. 
    Therefore, the function would have degenerate critical points along a 
    direction, which is perpendicular to the direction represented by the left 
    term, and it means along the direction the function value won't change.
  \item If $4ac - b^2 > 0$, both of the two terms are either $\leq 0$, or 
    $\geq 0$, depending on the sign of the factor $\frac{1}{4a}$. Therefore,
    \begin{itemize}
      \item When $a > 0$, both terms are positive, then the critical point 
        $(0, 0)$ is a \textbf{local minimum point}.
      \item When $a < 0$, both terms are negative, then the critical point
        $(0, 0)$ is a \textbf{local maximum point}.
    \end{itemize}
\end{itemize}

\section{General Cases: Second Derivative Test}

\subsection{Second Derivatives}

\textbf{Second derivatives} are partial derivatives of partial derivatives of 
multivariable functions. For multivariable functions with two independent 
variables, there are four second derivatives:
\begin{gather*}
  \frac{\partial^2 f}{\partial x^2} = f_{xx} \\
  \frac{\partial^2 f}{\partial x \partial y} = f_{xy} \\
  \frac{\partial^2 f}{\partial y \partial x} = f_{yx} \\
  \frac{\partial^2 f}{\partial y^2} = f_{yy} \\
\end{gather*}

A property of second derivatives is:
\begin{equation*}
  \frac{\partial^2 f}{\partial x \partial y} = \frac{\partial^2 f}{\partial y \partial x}
\end{equation*}
which means the order of partial derivatives taken doesn't affect the result.

\subsection{Rules of Second Derivative Test}

At a critical point $(x_0, y_0)$ of a multivariable function $f$, let
\begin{gather*}
  A = f_{xx}(x_0, y_0) \\
  B = f_{xy}(x_0, y_0) \\
  C = f_{yy}(x_0, y_0) \\
\end{gather*}
The conclusions are as follows:
\begin{itemize}
  \item If $AC - B^2 < 0$, then the critical point is a saddle point.
  \item If $AC - B^2 = 0$, then the type of the critical point is unknown.
  \item If $AC - B^2 > 0$, then
    \begin{itemize}
      \item If $A > 0$, then the critical point is a local minimum point.
      \item If $A < 0$, then the critical point is a local maximum point.
    \end{itemize}
\end{itemize}

We can verify that the second derivative test is consistent with the conclusions 
we get from the simple quadratic function case in the previous section.
\begin{gather*}
  z = f(x, y) = ax^2 + bxy + cy^2 \\
  f_x = 2ax + by \\
  f_y = 2cy + bx \\
  f_{xx} = 2a \\
  f_{xy} = b \\
  f_{yx} = b \\
  f_{yy} = 2c \\
\end{gather*}
Therefore
\begin{gather*}
  A = f_{xx} = 2a \\
  B = f_{xy} = b \\
  C = f_{yy} = 2c \\
  AC - B^2 = 4ac - b^2 \\
\end{gather*}

\subsection{Ideas behind Second Derivative Test}

The reason why the second derivative test holds can be explained by quadratic 
approximation. According to quadratic approximation, the function value around 
a point $(x_0, y_0)$ can be approximated as
\begin{gather*}
  f \approx f(x_0, y_0) + f_x(x - x_0) + f_y(y - y_0) + \frac{1}{2}f_{xx}(x - x_0)^2 + f_{xy}(x - x_0)(y - y_0) + \frac{1}{2}f_{yy}(y - y_0)^2
\end{gather*}

For critical points, $f_x = f_y = 0$, hence the approximation formula becomes 
\begin{gather*}
  f \approx f(x_0, y_0) + \frac{1}{2}f_{xx}(x - x_0)^2 + f_{xy}(x - x_0)(y - y_0) + \frac{1}{2}f_{yy}(y - y_0)^2
\end{gather*}
Therefore, the general case reduces to a quadratic function case.

Quadratic approximation also explains why we cannot draw any conclusion about 
the critical point when $AC - B^2 = 0$, i.e. the degenerate cases. In those cases, 
the function values along the direction are actually affected by the high-order 
terms that we ignored in the quadratic approximation. Therefore, we do not 
really know what the shape of the function graph is around the critical point.

\begin{example}
  Find the global maximum and minimum values of the function 
  $f(x, y) = x + y + \frac{1}{xy}, x > 0, y > 0$.

  Solution: \\
  \begin{gather*}
    f_x = 1 - \frac{1}{x^2y} \\
    f_y = 1 - \frac{1}{xy^2} \\
    f_{xx} = \frac{2}{x^3y} \\
    f_{xy} = \frac{1}{x^2y^2} \\
    f_{yy} = \frac{2}{xy^3} \\
  \end{gather*}
  Solving the system of equations
  \begin{equation*}
    \begin{cases}
      f_x = 1 - \frac{1}{x^2y} = 0 \\
      f_y = 1 - \frac{1}{xy^2} = 0 \\
    \end{cases}
  \end{equation*}
  we find that there are only one critical point $(1, 1)$, whose value is $3$.

  According to the second derivative test,
  \begin{gather*}
    A = f_{xx}(1, 1) = 2 \\
    B = f_{xy}(1, 1) = 1 \\
    C = f_{yy}(1, 1) = 2 \\
    AC - B^2 = 3 > 0 \\
    A > 0 \\
  \end{gather*}
  Therefore, the critical point $(1, 1)$ is a local minimum point. 
  
  Then let's check the boundary and infinity:
  \begin{gather*}
    \lim_{x \to 0}f(x, y) = \infty \\
    \lim_{y \to 0}f(x, y) = \infty \\
    \lim_{x \to \infty}f(x, y) = \infty \\
    \lim_{y \to \infty}f(x, y) = \infty \\
  \end{gather*}

  Therefore, the global minimum point is the only local minimum point $(1, 1)$, 
  whose value is $3$, and the global maximum point is the boundary or infinity, 
  where the function value is $\infty$.
\end{example}

\end{document}