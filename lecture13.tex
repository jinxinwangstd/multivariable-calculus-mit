\documentclass{article}
\usepackage[utf8]{inputenc}
\usepackage{amsmath}
\usepackage{amssymb}
\usepackage{amsthm}
\usepackage{tikz}
\setlength{\parindent}{0pt}

\newtheorem*{theorem}{Theorem}
\newtheorem*{definition}{Definition}
\newtheorem*{lemma}{Lemma}
\newtheorem*{corollary}{Corollary}
\newtheorem{example}{Example}
\newtheorem*{trick}{Trick}
\newtheorem*{question}{Question}

\title{Lecture 13: Lagrange Multipliers}
\author{}
\date{}

\begin{document}
    
\maketitle

\section{Motivation of Lagrange Multipliers}

The usage of Lagrange multipliers is to maximize/minimize the value of a 
function $f(x, y, z)$ where $x$, $y$, and $z$ are not independent, in other 
words there is a constraint $g(x, y, z) = c$.

Real world example: In thermodynamics, we often deal with a system with 
parameters such as temperature T, pressure P, and volume V. These parameters are 
not independent, and they satisfy a relation $PV = nRT$.

Those kinds of problems cannot be solved by only checking the critical points of 
the function $f(x, y, z)$, because they probably don't satisfy the existing 
constraint $g(x, y, z) = c$.

\begin{example}
  Find the point closest to the origin on the hyperbola $xy = 3$.

  We need to minimize the function $f(x, y) = \sqrt{x^2 + y^2}$, or more 
  conveniently, the function $f(x, y) = x^2 + y^2$, with the constraint 
  $g(x, y) = xy = 3$.

  From geometric perspective, we can plot the function graph of 
  $g(x, y) = xy = 3$, and the contour plot of the function 
  $f(x, y) = x^2 + y^2$. We can see that with a large constant $c_1$, the graphs 
  of $g(x, y) = 3$ and $f(x, y) = c_1$ have four intersection points; with a 
  small constant $c_2$, the two graphs have no intersection point. With the 
  correct solution $c_0$, the two graphs have exactly two intersection points.
\end{example}

\section{Solution with Lanrange Multipliers}

One key observation to the above example: At the minimum, the level curve of the 
function $f(x, y)$ is tangent to the hyperbola $xy = 3$, which is another level 
curve of the function $g(x, y)$.

Then how to find the point $(x, y)$ where the level curves of $f(x, y)$ and 
$g(x, y)$ are tangent to each other?

Notice: The level curves of $f(x, y)$ and $g(x, y)$ are tangent to each other \\
$\iff$ The level curves of $f(x, y)$ and $g(x, y)$ have the same tangent line \\
$\iff$ The gradients of $f(x, y)$ and $g(x, y)$ are parallel to each other \\
$\iff$ $\nabla f = \lambda \nabla g$, where $\lambda \neq 0$.

Therefore, we can derive a system of equations from the above statement:

From $\nabla f = \lambda \nabla g$, we can derive that
\begin{equation*}
  \begin{cases}
    f_x = \lambda g_x \\
    f_y = \lambda g_y \\
  \end{cases}
\end{equation*}
Also we have the constraint $g(x, y) = c$.

Therefore, for a optimization problem of a function with two variables, the 
derived system of equations is
\begin{equation*}
  \begin{cases}
    f_x = \lambda g_x \\
    f_y = \lambda g_y \\
    g(x, y) = c \\
  \end{cases}
\end{equation*}
It is sufficient to solve the point $(x, y)$ as well as the factor $\lambda$. 
This factor $\lambda$ is called the Lagrange multiplier.

\begin{example}
  Find the point closest to the origin on the hyperbola $xy = 3$ (Continue'd).

  \begin{gather*}
    f(x, y) = x^2 + y^2 \\
    f_x = 2x \\
    f_y = 2y \\
    g(x, y) = xy \\
    g_x = y \\
    g_y = x \\
  \end{gather*}
  Therefore, we can derive the following system of equations to solve the 
  minimum point of $f(x, y)$ under the constraint $g(x, y) = xy = 3$.
  \begin{equation*}
    \begin{cases}
      2x = \lambda y \\
      2y = \lambda x \\
      xy = 3 \\
    \end{cases}
  \end{equation*}
  If we focus on the first two equations, we can rewrite them as
  \begin{gather*}
    \begin{cases}
      2x - \lambda y = 0 \\
      \lambda x - 2y = 0 \\
    \end{cases} \\
    \begin{bmatrix}
      2 & -\lambda \\
      \lambda & -2 \\
    \end{bmatrix} \cdot
    \begin{bmatrix}
      x \\
      y \\
    \end{bmatrix} = 
    \begin{bmatrix}
      0 \\
      0 \\
    \end{bmatrix} \\
  \end{gather*}
  For this linear system, we have a trivial solution $x = y = 0$, but it doesn't 
  satisfy the constraint. Based on our exploration from geometric perspective, 
  we know that there are two solutions other than $x = y = 0$ for this system of 
  equations. Hence the linear system has more than one solution, which is 
  equivalent to
  \begin{gather*}
    det(A) = 0 \\
    \begin{vmatrix}
      2 & -\lambda \\
      \lambda & -2 \\
    \end{vmatrix} = 0 \\
    \lambda^2 - 4 = 0 \\
    \lambda = \pm 2 \\
  \end{gather*}
  When $\lambda = 2$,
  \begin{gather*}
    \begin{cases}
      2x - 2y = 0 \\
      2x - 2y = 0 \\
      xy = 3 \\
    \end{cases}\\
    x = \sqrt{3}, y = \sqrt{3} \\
    or\, x = -\sqrt{3}, y = -\sqrt{3} \\
  \end{gather*}
  When $\lambda = -2$,
  \begin{gather*}
    \begin{cases}
      2x + 2y = 0 \\
      2x + 2y = 0 \\
      xy = 3 \\
    \end{cases} \\
    \textnormal{There is no solution.}
  \end{gather*}
  Therefore, there are two points closest to the origin on the hyperbola 
  $xy = 3$, which are $(\sqrt{3}, \sqrt{3})$ and $(-\sqrt{3}, -\sqrt{3})$.
\end{example}

\section{Justification of Lagrange Multipliers}

At the constrained maximum/minimum points of a function $f$, in any direction 
along the level sets of $g = c$, the rate of change of $f$ must be 0. If not, 
then an adjacent point along the direction must be greater than or less than the 
maximum/minimum point, which is a contradiction to the fact of maximum/minimum 
point.

For any direction $\hat{u}$ tangent to the level set $g = c$, it holds that
\begin{gather*}
  \frac{df}{ds}|_{\hat{u}} = 0 \\
  \nabla f \cdot \hat{u} = 0 \\
\end{gather*}
Therefore, $\nabla f$ is perpendicular to any direction $\hat{u}$ tangent to the 
level set of $g$. Therefore, $\nabla f$ is perpendicular to the level set of 
$g$. Also according to the property of gradient, $\nabla g$ is also 
perpendicular to the level set of $g$. Therefore 
\[ \nabla f \parallel \nabla g \]

\section{Restriction of Lagrange Multipliers}

WARNING: the method of Lagrange multipliers doesn't tell whether a solution is a 
maximum or minimum points. We need to check other solutions as well as boundary 
values to determine whether it is a maximum point or minimum point.

Also, we cannot use the second derivative test to determine the type of a 
solution by the method of Lagrange multipliers.

\begin{example}
  Suppose that we want to build a pyramid with the given triangular base and the 
  given volume, find the shape with the minimum total surface area.

  Recall that the formula of the volume of a tetraheron is
  \[ V = \frac{1}{3}(\textnormal{base area})(\textnormal{height}) \]

  Since we are given the triangular base and the given volume, the height of the 
  tetraheron is actually fixed. Let $h$ denote the fixed height.

  One method is to suppose that the coordinate of the top vertex is $(x, y, h)$. 
  The coordinates of all other vertices are known: $(x_1, y_1, 0)$, 
  $(x_2, y_2, 0)$, and $(x_3, y_3, 0)$. So we can get the function of the total 
  surface area, which is a function of $x$ and $y$, and then try to minimize the 
  function. The problem with this method is that the function can be too 
  complex to minimize.

  \begin{tikzpicture}
    [help line/.style={dashed}]
    \draw (-3, -1.732) node[below left] {$P_1$};
    \draw (3, -1.732) node[below right] {$P_2$};
    \draw (0, 3.464) node[above] {$P_3$};
    \draw (-3, -1.732) -- (0, -1.732) node[below] {$a_1$} -- (3, -1.732);
    \draw (-3, -1.732) -- (-1.5, 0.866) node[above left] {$a_2$} -- (0, 3.464);
    \draw (3, -1.732) -- (1.5, 0.866) node[above right] {$a_3$} -- (0, 3.464);
    \draw (0, 0) node[right] {Q};
    \draw[help line] (0, 0) -- (0, -0.866) node[right] {$u_1$} -- (0, -1.732);
    \draw[help line] (0, 0) -- (-0.75, 0.433) node[right] {$u_2$} -- (-1.5, 0.866);
    \draw[help line] (0, 0) -- (0.75, 0.433) node[right] {$u_3$} -- (1.5, 0.866);
  \end{tikzpicture}

  Another method is to use different variables to calculate the total surface 
  area. As shown in the above diagram, suppose that the cast point of the top 
  vertex in the base is the point $Q$, and the lengths of the three edges are 
  $a_1$, $a_2$, and $a_3$. From $Q$ we construct perpendicular lines to the 
  three edges of the base, and let $u_1$, $u_2$, and $u_3$ denote their lengths.

  The heights of the three faces can be calculated as
  \begin{gather*}
    h_1 = \sqrt{u_1^2 + h^2} \\
    h_2 = \sqrt{u_2^2 + h^2} \\
    h_3 = \sqrt{u_3^2 + h^2} \\
  \end{gather*}
  The area of the three faces can be calculated as
  \begin{gather*}
    s_1 = \frac{1}{2}a_1\sqrt{u_1^2 + h^2} \\
    s_2 = \frac{1}{2}a_2\sqrt{u_2^2 + h^2} \\
    s_3 = \frac{1}{2}a_3\sqrt{u_3^2 + h^2} \\
  \end{gather*}
  Therefore, the total area of the three faces can be calculated as
  \begin{equation*}
    s = f(u_1, u_2, u_3) = \frac{1}{2}a_1\sqrt{u_1^2 + h^2} + \frac{1}{2}a_2\sqrt{u_2^2 + h^2} + \frac{1}{2}a_3\sqrt{u_3^2 + h^2}
  \end{equation*}
  Also we have the constraint about the base area:
  \begin{equation*}
    s_b = g(u_1, u_2, u_3) = \frac{1}{2}a_1u_1 + \frac{1}{2}a_2u_2 + \frac{1}{2}a_3u_3
  \end{equation*}
  We need to find the minimum point of the total area of the three faces 
  $s = f(u_1, u_2, u_3)$. According to the method of Lagrange multipliers, we 
  can derive the following system of equations:
  \begin{gather*}
    \begin{cases}
      f_{u_1} = \lambda g_{u_1} \\
      f_{u_2} = \lambda g_{u_2} \\
      f_{u_3} = \lambda g_{u_3} \\
      g(u_1, u_2, u_3) = c \\
    \end{cases} \\
    \begin{cases}
      \frac{a_1}{2}\frac{u_1}{\sqrt{u_1^2 + h^2}} = \lambda \frac{a_1}{2} \\
      \frac{a_2}{2}\frac{u_2}{\sqrt{u_2^2 + h^2}} = \lambda \frac{a_2}{2} \\
      \frac{a_3}{2}\frac{u_3}{\sqrt{u_3^2 + h^2}} = \lambda \frac{a_3}{2} \\
      g(u_1, u_2, u_3) = c \\
    \end{cases} \\
    \begin{cases}
      \frac{u_1}{\sqrt{u_1^2 + h^2}} = \lambda \\
      \frac{u_2}{\sqrt{u_2^2 + h^2}} = \lambda \\
      \frac{u_3}{\sqrt{u_3^2 + h^2}} = \lambda \\
      g(u_1, u_2, u_3) = c \\
    \end{cases} \\
  \end{gather*}
  Therefore, when the total surface area is minimum, $u_1 = u_2 = u_3$.
\end{example}

\end{document}