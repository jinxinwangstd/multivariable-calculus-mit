\documentclass{article}
\usepackage[utf8]{inputenc}
\usepackage{amsmath}
\usepackage{amssymb}
\usepackage{amsthm}
\usepackage{tikz}
\setlength{\parindent}{0pt}

\newtheorem*{theorem}{Theorem}
\newtheorem*{definition}{Definition}
\newtheorem*{lemma}{Lemma}
\newtheorem*{corollary}{Corollary}
\newtheorem{example}{Example}
\newtheorem*{trick}{Trick}
\newtheorem*{question}{Question}

\title{Lecture 13: Lagrange Multipliers}
\author{}
\date{}

\begin{document}
    
\maketitle

\section{Motivation of Lagrange Multipliers}

The usage of Lagrange multipliers is to maximize/minimize the value of a 
function $f(x, y, z)$ where $x$, $y$, and $z$ are not independent, in other 
words there is a constraint $g(x, y, z) = c$.

Real world example: In thermodynamics, we often deal with a system with 
parameters such as temperature T, pressure P, and volume V. These parameters are 
not independent, and they satisfy a relation $PV = nRT$.

Those kinds of problems cannot be solved by only checking the critical points of 
the function $f(x, y, z)$, because they probably don't satisfy the existing 
constraint $g(x, y, z) = c$.

\begin{example}
  Find the point closest to the origin on the hyperbola $xy = 3$.

  We need to minimize the function $f(x, y) = \sqrt{x^2 + y^2}$, or more 
  conveniently, the function $f(x, y) = x^2 + y^2$, with the constraint 
  $g(x, y) = xy = 3$.

  From geometric perspective, we can plot the function graph of 
  $g(x, y) = xy = 3$, and the contour plot of the function 
  $f(x, y) = x^2 + y^2$. We can see that with a large constant $c_1$, the graphs 
  of $g(x, y) = 3$ and $f(x, y) = c_1$ have four intersection points; with a 
  small constant $c_2$, the two graphs have no intersection point. With the 
  correct solution $c_0$, the two graphs have exactly two intersection points.
\end{example}

\section{Solution with Lanrange Multipliers}

One key observation to the above example: At the minimum, the level curve of the 
function $f(x, y)$ is tangent to the hyperbola $xy = 3$, which is another level 
curve of the function $g(x, y)$.

Then how to find the point $(x, y)$ where the level curves of $f(x, y)$ and 
$g(x, y)$ are tangent to each other?

Notice: The level curves of $f(x, y)$ and $g(x, y)$ are tangent to each other \\
$\iff$ The level curves of $f(x, y)$ and $g(x, y)$ have the same tangent line \\
$\iff$ The gradients of $f(x, y)$ and $g(x, y)$ are parallel to each other \\
$\iff$ $\nabla f = \lambda \nabla g$, where $\lambda \neq 0$.

Therefore, we can derive a system of equations from the above statement:

From $\nabla f = \lambda \nabla g$, we can derive that
\begin{equation*}
  \begin{cases}
    f_x = \lambda g_x \\
    f_y = \lambda g_y \\
  \end{cases}
\end{equation*}
Also we have the constraint $g(x, y) = c$.

Therefore, for a optimization problem of a function with two variables, the 
derived system of equations is
\begin{equation*}
  \begin{cases}
    f_x = \lambda g_x \\
    f_y = \lambda g_y \\
    g(x, y) = c \\
  \end{cases}
\end{equation*}
It is sufficient to solve the point $(x, y)$ as well as the factor $\lambda$. 
This factor $\lambda$ is called the Lagrange multiplier.

\end{document}