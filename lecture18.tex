\documentclass{article}
\usepackage[utf8]{inputenc}
\usepackage{amsmath}
\usepackage{amssymb}
\usepackage{amsthm}
\usepackage{tikz}
\setlength{\parindent}{0pt}

\newtheorem*{theorem}{Theorem}
\newtheorem*{definition}{Definition}
\newtheorem*{lemma}{Lemma}
\newtheorem*{corollary}{Corollary}
\newtheorem{example}{Example}
\newtheorem*{trick}{Trick}
\newtheorem*{question}{Question}

\title{Lecture 18: Change of Variables}
\author{}
\date{}

\begin{document}
    
\maketitle

\section{Introduction to Change of Variables in Double Integrals}

\begin{example}
  Calculate the area of the ellipse with semiaxes $a$ and $b$.

  We can use double integrals to calculate the area. By definition of double 
  integrals, the area of the described ellipse 
  \begin{equation*}
    \begin{split}
      A &= \iint_R 1 dA \\
        &= \iint_{(\frac{x}{a})^2 + (\frac{y}{b})^2 = 1} dx dy \\
    \end{split}
  \end{equation*}
  
  To simplify the equation, we can set
  \begin{gather*}
    u = \frac{x}{a} \\
    v = \frac{y}{b} \\
  \end{gather*}
  Then
  \begin{gather*}
    du = \frac{dx}{a} \\
    dv = \frac{dy}{b} \\
    du dv = \frac{dx dy}{ab} \\
    dx dy = ab du dv \\
  \end{gather*}
  
  Therefore, the equation of the double integral is equivalent to
  \begin{equation*}
    \begin{split}
      A &= \iint_{(\frac{x}{a})^2 + (\frac{y}{b})^2 = 1} dx dy \\
        &= \iint_{u^2 + v^2 = 1} ab du dv \\
        &= ab \iint_{u^2 + v^2 = 1} du dv \\
        &= \pi ab \\
    \end{split}
  \end{equation*}

  Therefore, the area of the ellipse with semiaxes $a$ and $b$ is $\pi ab$.
\end{example}

From the above example we can see, in general when we change the variables in a 
double integral, we need to find the scaling factor between $dA = dx dy$ and 
$dA' = du dv$. In order to ensure the correctness of the double integral after 
changing the variables, we need to multiply the scaling factor with 
differentials of changed variables, to ensure the differential of area is still 
$dA$ rather than $dA'$. Also note that the scaling factor between $dA$ and $dA'$ 
might not be constant in the region $R$.

\begin{question}
  Why when we make sure $dA$ is correctly expressed by $du dv$ the double 
  integral with change of variables yields the correct result?

  Answer:

  Probably think about it from the definition of double integral, and the 
  geometric interpretation of change of variables.
\end{question}

\section{Change of Variables with Linear Transformation}

\begin{example}
  Find the scaling factor between $dA = dx dy$ and $dA' = du dv$ with the 
  following linear transformation:
  \begin{equation*}
    \begin{cases}
      u = 3x - 2y \\
      v = x + y \\
    \end{cases}
  \end{equation*}

  In the $xy$-coordinates, $dA = dx dy$ represents a small rectangle. Regarding 
  the geometric interpretation of the linear transformation, we have the 
  following two observations:
  \begin{itemize}
    \item A rectangle in the $xy$-coordinates corresponds to a parallelogram in 
    the $uv$-coordinates. The ratio of the area of the parallelogram in the 
    $uv$-coordinates to the area of the rectangle in the $xy$-coordinates is the 
    area scaling factor.
    \item The area scaling factor of the linear transformation doesn't depend on 
    the choice of rectangles, either the position or the area.
  \end{itemize}

  Based on the above two observations, we can study a special case to find the 
  scaling factor of the linear transformation. The special case we choose is a 
  unit rectangle with the left bottom corner on the origin.

  \begin{tikzpicture}
    [help line/.style={dashed}]
    \draw[->] (-1, 0) -- (2, 0) node[right] {x};
    \draw[->] (0, -1) -- (0, 2) node[right] {y};
    \draw (0, 0) node[below right] {O};
    \draw[-] (0, 0) -- (0, 1) -- (1, 1) -- (1, 0) -- (0, 0);
  \end{tikzpicture}

  After the linear transformation,
  \begin{gather*}
    (0, 0) \rightarrow (0, 0) \\
    (1, 0) \rightarrow (3, 1) \\
    (1, 1) \rightarrow (1, 2) \\
    (0, 1) \rightarrow (-2, 1) \\
  \end{gather*}

  \begin{tikzpicture}
    [help line/.style={dashed}]
    \draw[->] (-2, 0) -- (3, 0) node[right] {u};
    \draw[->] (0, -1) -- (0, 3) node[right] {v};
    \draw (0, 0) node[below right] {O};
    \draw[-] (0, 0) -- (3, 1) -- (1, 2) -- (-2, 1) -- (0, 0);
  \end{tikzpicture}

  We can use the determinant to calculate the area of a parallelogram in a plane
  \begin{equation*}
    \begin{split}
      A &= \det(<3, 1>, <-2, 1>) \\
        &= \begin{vmatrix}
             3 & 1 \\
             -2 & 1 \\ 
           \end{vmatrix} \\
        &= 5 \\
    \end{split}
  \end{equation*}

  Therefore, the scaling factor of the linear transformation is $5$:
  \begin{gather*}
    dA' = 5dA \\
    dA = \frac{1}{5} dA' \\
    dx dy = \frac{1}{5} du dv \\
  \end{gather*}

\end{example}

\begin{question}
  Why the scaling factor in the above example is $5$ rather than $\frac{1}{5}$?

  Answer:
\end{question}

\section{Change of Variables in General}

For general cases of change of variables:
\begin{gather*}
  u = u(x, y) \\
  v = v(x, y) \\
\end{gather*}

By linear approximation, we have
\begin{gather*}
  \Delta u \approx u_x \Delta x + u_y \Delta y \\
  \Delta v \approx v_x \Delta x + v_y \Delta y \\
  \begin{bmatrix}
    \Delta u \\
    \Delta v \\
  \end{bmatrix} \approx 
  \begin{bmatrix}
    u_x & u_y \\
    v_x & v_y \\
  \end{bmatrix}
  \begin{bmatrix}
    \Delta x \\
    \Delta y \\
  \end{bmatrix} \\
\end{gather*}

Therefore, for any transformation or change of variables, a small rectangle in 
the original coordinate system corresponds to a parallelogram in the new 
coordinate system.

The correspondence of the vertices between the small rectange in the original 
coordinate system and the parallelogram in the new coordinate system is
\begin{gather*}
  (0, 0) \rightarrow (0, 0) \\
  (\Delta x, 0) \rightarrow (u_x \Delta x, v_x \Delta x) \\
  (0, \Delta y) \rightarrow (u_y \Delta y, v_y \Delta y) \\
  (\Delta x, \Delta y) \rightarrow (u_x \Delta x + u_y \Delta y, v_x \Delta x + v_y \Delta y) \\
\end{gather*}

Therefore, the area of the parallelogram is
\begin{equation*}
  \begin{split}
    A &= \det(<u_x \Delta x, v_x \Delta x>, <u_y \Delta y, v_y \Delta y>) \\
      &= \begin{vmatrix}
           u_x \Delta x & v_x \Delta x \\
           u_y \Delta y & v_y \Delta y \\
         \end{vmatrix} \\
      &= (u_x v_y - u_y v_x) \Delta x \Delta y \\
      &= \begin{vmatrix}
           u_x & u_y \\
           v_x & v_y \\
         \end{vmatrix} \Delta x \Delta y \\
  \end{split}
\end{equation*}

Therefore, the scaling factor between two coordinates is 
$\begin{vmatrix}
   u_x & u_y \\
   v_x & v_y \\
\end{vmatrix}$. Since the scaling factor consists of partial derivatives, it 
varies with different positions of the differential of area.

Here we define the Jacobian determinant as
\begin{equation*}
  J = \frac{\partial(u, v)}{\partial(x, y)} = \begin{vmatrix}
                                                u_x & u_y \\
                                                v_x & v_y \\ 
                                              \end{vmatrix}
\end{equation*}
Then
\begin{equation*}
  du dv = |J| dx dy = |\frac{\partial(u, v)}{\partial(x, y)}| dx dy
\end{equation*}

\begin{example}
  Find the scaling factor between the Cartesian coordinates and the polar 
  coordinates.

  According to the definition of the polar coordinates
  \begin{gather*}
    x = r \cos\theta \\
    y = r \sin\theta \\
  \end{gather*}

  Then we can find the Jacobian determinant as
  \begin{equation*}
    \begin{split}
      J &= \frac{\partial(x, y)}{\partial(r, \theta)} \\
        &= \begin{vmatrix}
             \frac{\partial x}{\partial r} & \frac{\partial x}{\partial \theta} \\
             \frac{\partial y}{\partial r} & \frac{\partial y}{\partial \theta} \\
           \end{vmatrix} \\
        &= \begin{vmatrix}
             \cos\theta & -r \sin\theta \\
             \sin\theta & r \cos\theta \\
           \end{vmatrix} \\
        &= r \cos^2\theta + r \sin^2\theta \\
        &= r \\
    \end{split}
  \end{equation*}

  Therefore, 
  \begin{equation*}
    dx dy = |\frac{\partial(x, y)}{\partial(r, \theta)}| dr d\theta = r dr d\theta
  \end{equation*}
\end{example}

Note that the Jacobian determinant of a change of variables works in both ways, 
which means
\begin{gather*}
  J_1 = \frac{\partial(x, y)}{\partial(u, v)} = \frac{dx dy}{du dv} \\
  J_2 = \frac{\partial(u, v)}{\partial(x, y)} = \frac{du dv}{dx dy} \\
  \frac{dx dy}{du dv} \cdot \frac{du dv}{dx dy} = 1 \\
  \frac{\partial(x, y)}{\partial(u, v)} \cdot \frac{\partial(u, v)}{\partial(x, y)} = 1 \\
\end{gather*}

Therefore, the two Jacobian determinants are inverse to each other, and we can 
choose the easier one to calculate.

\begin{example}
  Calculate $\int_0^1 \int_0^1 x^2 y dx dy$ by changing the variables as
  \begin{gather*}
    u = x \\
    v = xy \\
  \end{gather*}

  First we need to calculate the differential of area.
  \begin{gather*}
    \begin{split}
      J &= \frac{\partial(u, v)}{\partial(x, y)} \\
        &= \begin{vmatrix}
             1 & 0 \\
             y & x \\
           \end{vmatrix} \\
        &= x \\
    \end{split} \\
    \begin{split}
      du dv &= |J| dx dy \\
            &= |x| dx dy \\
            &= x dx dy \ (x \geq 0)
    \end{split} \\
  \end{gather*}

  Then we need to calculate the integrand
  \begin{equation*}
    \begin{split}
      x^2 y dx dy &= x^2 y \frac{1}{x} du dv \\
                  &= xy du dv \\
                  &= v du dv \\
    \end{split}
  \end{equation*}

  Then we need to determine the region of the double integral in the new 
  coordinates.

  In the $xy$-coordinates, the region of the double integral is a rectangle as 
  shown in the below diagram.

  \begin{tikzpicture}
    [help line/.style={dashed}]
    \draw[->] (-1, 0) -- (2, 0) node[right] {x};
    \draw[->] (0, -1) -- (0, 2) node[right] {y};
    \draw (0, 0) node[below right] {O};
    \draw[-] (0, 0) -- (1, 0) -- (1, 1) -- (0, 1) -- (0, 0);
  \end{tikzpicture}

  After changing the variables, if we calculate the double integral in the order 
  of $du dv$, it means for each possible value of $v$, we will keep it constant, 
  and vary the value of $u$.

  Therefore, the way we determine the boundary of the region is to find the 
  range of $v$, and for each value of $v$, find the possible values of $u$ based 
  on the region in $xy$-coordinates.

  According to the diagram, we can find that the range of $v$ is $(0, 1)$, and 
  for each value of $v$, the range of $u$ is $(v, 1)$.

  Therefore, the double integral after changing the variables is
  \begin{equation*}
    \begin{split}
      \int_0^1 \int_0^1 x^2 y dx dy &= \int_0^1 \int_v^1 v du dv \\
                                    &= \int_0^1 (uv)|_{u = v}^{u = 1} dv \\
                                    &= \int_0^1 v - v^2 dv \\
                                    &= (\frac{v^2}{2} - \frac{v^3}{3})|_0^1 \\
                                    &= \frac{1}{6} \\
    \end{split}
  \end{equation*}
\end{example}

\end{document}