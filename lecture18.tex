\documentclass{article}
\usepackage[utf8]{inputenc}
\usepackage{amsmath}
\usepackage{amssymb}
\usepackage{amsthm}
\usepackage{tikz}
\setlength{\parindent}{0pt}

\newtheorem*{theorem}{Theorem}
\newtheorem*{definition}{Definition}
\newtheorem*{lemma}{Lemma}
\newtheorem*{corollary}{Corollary}
\newtheorem{example}{Example}
\newtheorem*{trick}{Trick}
\newtheorem*{question}{Question}

\title{Lecture 18: Change of Variables}
\author{}
\date{}

\begin{document}
    
\maketitle

\section{Introduction to Change of Variables in Double Integrals}

\begin{example}
  Calculate the area of the ellipse with semiaxes $a$ and $b$.

  We can use double integrals to calculate the area. By definition of double 
  integrals, the area of the described ellipse 
  \begin{equation*}
    \begin{split}
      A &= \iint_R 1 dA \\
        &= \iint_{(\frac{x}{a})^2 + (\frac{y}{b})^2 = 1} dx dy \\
    \end{split}
  \end{equation*}
  
  To simplify the equation, we can set
  \begin{gather*}
    u = \frac{x}{a} \\
    v = \frac{y}{b} \\
  \end{gather*}
  Then
  \begin{gather*}
    du = \frac{dx}{a} \\
    dv = \frac{dy}{b} \\
    du dv = \frac{dx dy}{ab} \\
    dx dy = ab du dv \\
  \end{gather*}
  
  Therefore, the equation of the double integral is equivalent to
  \begin{equation*}
    \begin{split}
      A &= \iint_{(\frac{x}{a})^2 + (\frac{y}{b})^2 = 1} dx dy \\
        &= \iint_{u^2 + v^2 = 1} ab du dv \\
        &= ab \iint_{u^2 + v^2 = 1} du dv \\
        &= \pi ab \\
    \end{split}
  \end{equation*}

  Therefore, the area of the ellipse with semiaxes $a$ and $b$ is $\pi ab$.
\end{example}

From the above example we can see, in general when we change the variables in a 
double integral, we need to find the scaling factor between $dA = dx dy$ and 
$dA' = du dv$. In order to ensure the correctness of the double integral after 
changing the variables, we need to multiply the scaling factor with 
differentials of changed variables, to ensure the differential of area is still 
$dA$ rather than $dA'$. Also note that the scaling factor between $dA$ and $dA'$ 
might not be constant in the region $R$.

\begin{question}
  Why when we make sure $dA$ is correctly expressed by $du dv$ the double 
  integral with change of variables yields the correct result?

  Answer:

  Probably think about it from the definition of double integral, and the 
  geometric interpretation of change of variables.
\end{question}

\section{Change of Variables with Linear Transformation}

\begin{example}
  Find the scaling factor between $dA = dx dy$ and $dA' = du dv$ with the 
  following linear transformation:
  \begin{equation*}
    \begin{cases}
      u = 3x - 2y \\
      v = x + y \\
    \end{cases}
  \end{equation*}

  In the $xy$-coordinates, $dA = dx dy$ represents a small rectangle. Regarding 
  the geometric interpretation of the linear transformation, we have the 
  following two observations:
  \begin{itemize}
    \item A rectangle in the $xy$-coordinates corresponds to a parallelogram in 
    the $uv$-coordinates. The ratio of the area of the parallelogram in the 
    $uv$-coordinates to the area of the rectangle in the $xy$-coordinates is the 
    area scaling factor.
    \item The area scaling factor of the linear transformation doesn't depend on 
    the choice of rectangles, either the position or the area.
  \end{itemize}

  Based on the above two observations, we can study a special case to find the 
  scaling factor of the linear transformation. The special case we choose is a 
  unit rectangle with the left bottom corner on the origin.

  \begin{tikzpicture}
    [help line/.style={dashed}]
    \draw[->] (-1, 0) -- (2, 0) node[right] {x};
    \draw[->] (0, -1) -- (0, 2) node[right] {y};
    \draw (0, 0) node[below right] {O};
    \draw[-] (0, 0) -- (0, 1) -- (1, 1) -- (1, 0) -- (0, 0);
  \end{tikzpicture}

  After the linear transformation,
  \begin{gather*}
    (0, 0) \rightarrow (0, 0) \\
    (1, 0) \rightarrow (3, 1) \\
    (1, 1) \rightarrow (1, 2) \\
    (0, 1) \rightarrow (-2, 1) \\
  \end{gather*}

  \begin{tikzpicture}
    [help line/.style={dashed}]
    \draw[->] (-2, 0) -- (3, 0) node[right] {u};
    \draw[->] (0, -1) -- (0, 3) node[right] {v};
    \draw (0, 0) node[below right] {O};
    \draw[-] (0, 0) -- (3, 1) -- (1, 2) -- (-2, 1) -- (0, 0);
  \end{tikzpicture}

  We can use the determinant to calculate the area of a parallelogram in a plane
  \begin{equation*}
    \begin{split}
      A &= \det(<3, 1>, <-2, 1>) \\
        &= \begin{vmatrix}
             3 & 1 \\
             -2 & 1 \\ 
           \end{vmatrix} \\
        &= 5 \\
    \end{split}
  \end{equation*}

  Therefore, the scaling factor of the linear transformation is $5$:
  \begin{gather*}
    dA' = 5dA \\
    dA = \frac{1}{5} dA' \\
    dx dy = \frac{1}{5} du dv \\
  \end{gather*}

\end{example}

\begin{question}
  Why the scaling factor in the above example is $5$ rather than $\frac{1}{5}$?

  Answer:
\end{question}

\section{Change of Variables in General}

\end{document}