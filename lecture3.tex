\documentclass{article}
\usepackage[utf8]{inputenc}
\usepackage{amsmath}
\usepackage{amssymb}
\usepackage{amsthm}
\usepackage{tikz}
\setlength{\parindent}{0pt}

\newtheorem*{theorem}{Theorem}
\newtheorem*{definition}{Definition}
\newtheorem*{lemma}{Lemma}
\newtheorem*{corollary}{Corollary}
\newtheorem{example}{Example}

\title{Lecture 3: Matrices}
\author{}
\date{}

\begin{document}
    
\maketitle

\section{Matrices}

A motivation for bringing in matrices is better expressing linear relations
between variables.

\begin{example}
  The coordinates of a point in two different Cartesian coordinate systems have
  linear relations. Suppose in the first coordinate system, the point
  $P = (x_1, x_2, x_3)$, and in the second one, the point $P = (u_1, u_2, u_3)$.
  The relations between them can be described by linear equations, such as
  \[
    \left\{ \begin{array}{ll}
    u_1 = 2x_1 + x_2 + 5x_3 \\
    u_2 = x_1 + 3x_2 + 2x_3 \\
    u_3 = x_1 + 2x_2 + x_3
    \end{array} \right.
  \]
  The reason why their relation is linear is that each unit vector in the
  second coordinate system can be decomposed into three vectors along the
  directions of the unit vectors in the first coordinate system, i.e. each unit
  vector in the second coordinate system has a unique coordinate in the first
  system. Since $x_1$, $x_2$, $x_3$ are just scalars in three directions of the
  first coordinate system, the coordinates of the unit vectors of the second
  system can also be expressed with $x_1$, $x_2$, $x_3$, a linear combination
  of these three scalars. Then the coordinate of the point in the second
  coordinate system is also a linear combination of the unit vectors in the
  system, so the coordinate of the point in the second coordinate system is a
  linear combination of the coordinate of the point in the first one.
\end{example}

\end{document}