\documentclass{article}
\usepackage[utf8]{inputenc}
\usepackage{amsmath}
\usepackage{amssymb}
\usepackage{amsthm}
\usepackage{tikz}
\setlength{\parindent}{0pt}

\newtheorem*{theorem}{Theorem}
\newtheorem*{definition}{Definition}
\newtheorem*{lemma}{Lemma}
\newtheorem*{corollary}{Corollary}
\newtheorem{example}{Example}
\newtheorem*{trick}{Trick}
\newtheorem*{question}{Question}

\title{Lecture 16: Double Integrals}
\author{}
\date{}

\begin{document}
    
\maketitle

\section{Definition of Double Integrals}

For single-variable functions, we use integral to calculate the area under the 
function curve. For a multivariable function with two independent variables 
$z = f(x, y)$, we define double integrals to calculate the volume under the 
function graph over a region $R$ in the $x$$y$-plane. The notation is 
$\iint_R f(x, y)dA$.

\bigskip

Definition of double integrals:

If we cut the region $R$ into small pieces of area $\Delta A_i$, then the volume 
can be calculated as 
\begin{equation*}
  V \approx \sum_i f(x_i, y_i) \Delta A_i
\end{equation*}
If we take the limit of $\Delta A_i$ to 0, then it becomes the double integral
\begin{equation*}
  \iint_R f(x, y) dA = \lim_{\Delta A_i \to 0} \sum_i f(x_i, y_i) \Delta A_i
\end{equation*}

\section{Calculation of Double Integrals}

To compute a double integral $\iint_R f(x, y) dA$, we can take slices of the 
function graph along one dimension and use integral to get the area of slices. 
In this way, the area of all slices becomes a function of another dimension, 
then we can use the integral again to sum up those slices to get the volume.

More concretely, let $S(x)$ denote the area of slices by plane parallel to 
$y$$z$-plane. Then
\begin{equation*}
  \iint_R f(x, y) dA = \int_{x_min}^{x_max} S(x) dx
\end{equation*}
For a given $x_0$, the area of the slice
\begin{equation*}
  S(x_0) = \int_{y_min(x_0)}^{y_max(x_0)} f(x, y) dy
\end{equation*}
Therefore, the double integral
\begin{equation*}
  \begin{split}
    \iint_R f(x, y) dA &= \int_{x_min}^{x_max} S(x) dx \\
                       &= \int_{x_min}^{x_max} \int_{y_min(x_0)}^{y_max(x_0)} f(x, y) dy dx
  \end{split}
\end{equation*}
This calculation method is called \textbf{iterated integral}, because we iterate 
twice through both dimensions to get the integral.

Notice that the bounds of the outer integral are numbers, and the bounds of the 
inner integral are numbers or functions depending on the variable of the outer 
integral. The bounds of both outer and inner integrals are determined by the 
region of the double integral.

\section{Examples of Double Integral}

\begin{example}
  Find the value of the double integral of the function $z = 1 - x^2 - y^2$ over 
  the region $R$ as $0 \leq x \leq 1$, $0 \leq y \leq 1$.

  Solution:

  By using iterated integrals,
  \begin{equation*}
    \iint_R f(x, y) dA = \int_0^1 \int_0^1 1 - x^2 - y^2 dy dx
  \end{equation*}
  For the inner integral,
  \begin{equation*}
    \begin{split}
      \int_0^1 1 - x^2 - y^2 dy &= (y - x^2y - \frac{y^3}{3})|_0^1 \\
                                &= 1 - x^2 - \frac{1}{3} \\
                                &= \frac{2}{3} - x^2 \\
    \end{split}
  \end{equation*}
  For the outer integral,
  \begin{equation*}
    \begin{split}
      \int_0^1 \frac{2}{3} - x^2 dx &= (\frac{2x}{3} - \frac{x^3}{3})|_0^1 \\
                                    &= \frac{2}{3} - \frac{1}{3} \\
                                    &= \frac{1}{3} \\
    \end{split}
  \end{equation*}
  Therefore, the value of the double integral is $\frac{1}{3}$.
\end{example}

\begin{example}
  Find the value of the double integral of the function $z = 1 - x^2 - y^2$ over 
  the region $R$ as 
  \begin{equation*}
    \begin{cases}
      x^2 + y^2 \leq 1 \\
      x \geq 0 \\
      y \geq 0 \\
    \end{cases}
  \end{equation*}

  By the equation of the region $R$, we can get that for any given $x_0$, the 
  range of $y$ in the region $R$ is $[0, \sqrt{1 - x^2}]$.

  By using iterated integrals,
  \begin{equation*}
    \iint_R f(x, y) dA = \int_0^1 \int_0^{\sqrt{1 - x^2}} 1 - x^2 - y^2 dy dx
  \end{equation*}
  For the inner integral,
  \begin{equation*}
    \begin{split}
      \int_0^{\sqrt{1 - x^2}} 1 - x^2 - y^2 dy &= (y - x^2y - \frac{y^3}{3})|_0^{\sqrt{1 - x^2}} \\
                                               &= \sqrt{1 - x^2} - x^2\sqrt{1 - x^2} - \frac{(1 - x^2)^{\frac{3}{2}}}{3}\\
                                               &= \frac{2}{3}(1 - x^2)^{\frac{3}{2}} \\
    \end{split}
  \end{equation*}
  For the outer integral,
  \begin{equation*}
    \begin{split}
      \int_0^1 \frac{2}{3}(1 - x^2)^{\frac{3}{2}} dx &= \int_0^{\frac{\pi}{2}} \frac{2}{3}(1 - \sin^2\theta)^{\frac{3}{2}} d(\sin\theta) \\
                                                     &= \frac{2}{3} \int_0^{\frac{\pi}{2}} \cos^4\theta d\theta \\
                                                     &= \frac{2}{3} \int_0^{\frac{\pi}{2}} (\frac{1 + \cos 2\theta}{2})^2 d\theta \\
                                                     &= \frac{1}{6} \int_0^{\frac{\pi}{2}} (1 + 2\cos2\theta + \cos^2 2\theta) d\theta \\
                                                     &= \frac{1}{6} \int_0^{\frac{\pi}{2}} (1 + 2\cos2\theta + \frac{1 + \cos 4\theta}{2}) d\theta \\
                                                     &= \frac{1}{12} \int_0^{\frac{\pi}{2}} (3 + 4\cos2\theta + \cos 4\theta) d\theta \\
                                                     &= \frac{1}{12} (3\theta + 2\sin2\theta + \frac{1}{4}\cos 4\theta)|_0^{\frac{\pi}{2}} \\
                                                     &= \frac{1}{12} \cdot \frac{3\pi}{2} \\
                                                     &= \frac{\pi}{8} \\
    \end{split}
  \end{equation*}
  Therefore, the value of the double integral is $\frac{\pi}{8}$.
\end{example}

\section{Exchange the Order of Iterated Integrals}

In theory, no matter in which order we calculate the iterated integral, the 
values always exist and are the same. However, in practice an order requires 
much more complex computation than the other; sometimes we might not even be 
able to compute the double integral with a specific order, but able to do it 
with the other order. Therefore, we need to be mindful about the order of 
iterated integral we use.

In some cases, we can exchange the order of an iterated integral, with some 
adaption to the bounds of integrals.

\begin{example}
  \begin{equation*}
    \int_0^1 \int_0^2 dx dy = \int_0^2 \int_0^1 dy dx
  \end{equation*}
  For rectangle regions, the bounds of the inner integral don't depend on the 
  variable of the outer integral, so we can exchange the inner integral and 
  outer integral freely.

  From the perspective of the definition of iterated integrals, for rectangle 
  regions, slices of either direction don't change the set of small pieces of 
  area $dA_i$ to sum up.
\end{example}

\begin{example}
  Find the value of $\int_0^1 \int_x^{\sqrt{x}} \frac{e^y}{y} dy dx$.

  It is too difficult to calculate $\int_x^{\sqrt{x}} \frac{e^y}{y} dy$, so we 
  need to consider exchanging the order of integrals.

  By looking at the region, we derive that
  \begin{equation*}
    \int_0^1 \int_x^{\sqrt{x}} \frac{e^y}{y} dy dx = \int_0^1 \int_{y^2}^y \frac{e^y}{y} dx dy
  \end{equation*}
  and the one on the right-hand side is doable.

  For the inner integral,
  \begin{equation*}
    \begin{split}
      \int_{y^2}^y \frac{e^y}{y} dx &= \frac{xe^y}{y}|_{y^2}^y \\
                                    &= e^y - ye^y \\
    \end{split}
  \end{equation*}
  For the outer integral,
  \begin{equation*}
    \begin{split}
      \int_0^1 e^y - ye^y dy &= (2e^y - ye^y)|_0^1 \\
                             &= 2e - e - 2 \\
                             &= e - 2 \\
    \end{split}
  \end{equation*}
  Therefore, the value of $\int_0^1 \int_x^{\sqrt{x}} \frac{e^y}{y} dy dx$ is 
  $e - 2$.
\end{example}

\end{document}