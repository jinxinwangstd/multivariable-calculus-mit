\documentclass{article}
\usepackage[utf8]{inputenc}
\usepackage{amsmath}
\usepackage{amssymb}
\usepackage{amsthm}
\usepackage{tikz}
\setlength{\parindent}{0pt}

\newtheorem*{theorem}{Theorem}
\newtheorem*{definition}{Definition}
\newtheorem*{lemma}{Lemma}
\newtheorem*{corollary}{Corollary}
\newtheorem{example}{Example}
\newtheorem{trick}{Trick}
\newtheorem{question}{Question}

\title{Problem Set 2}
\author{}
\date{}

\begin{document}

\begin{center}
{\rmfamily\bfseries\Large 18.02 EXERCISES}

\vspace{25px}

{\rmfamily\bfseries\LARGE Problem Set 2: Matrices and Systems of Equations}
\end{center}

\begin{center}
\section*{Part I}
\end{center}

\subsection*{Unit 1E Equations of Lines and Planes}

1. Find the equations of the following planes: \\
a) through (2, 0, -1) and perpendicular to $\vec{i} + 2\vec{j} - 2\vec{k}$. \\
b) through the origin, (1, 1, 0), and (2, -1, 3) \\
c) through (1, 0, 1), (2, -1, 2), (-1, 3, 2) \\
d) through the points on the $x$, $y$ and $z$-axes where $x = a$, $y = b$, 
$z = c$ respectively (give the equation in the form $Ax + By + Cz = 1$ and 
remember it) \\
e) through (1, 0, 1) and (0, 1, 1) and parallel to $\vec{i} - \vec{j} + 2\vec{k}$

Solution:

a) According to the problem description, the vector $<1, 2, -2>$ is a normal 
vector to the plane. Therefore, the equation of the plane is
\[ x + 2y - 2z = c, \textnormal{where c is a constant} \]
Then we can put the point (2, 0, -1) into the equation:
\[ c = 2 + 2 \times 0 - 2 \times (-1) = 4 \]
Hence, the equation of the plane is 
\[ x + 2y - 2z = 4 \]

b) According to the problem description, two vectors on the plane are
\[ <1, 1, 0>, <2, -1, 3> \]
Therefore, a normal vector to the plane can be calculated as
\begin{gather*}
  \begin{split}
    <1, 1, 0> \times <2, -1, 3> 
    &= \begin{vmatrix}
         \vec{i} & \vec{j} & \vec{k} \\
         1 & 1 & 0 \\
         2 & -1 & 3 \\
       \end{vmatrix} \\
    &= <3, -3, -3> \\
  \end{split}
\end{gather*}
Therefore, the equation of the plane is
\[ 3x - 3y - 3z = c, \textnormal{where c is a constant} \]
Then we can put the point (0, 0, 0) into the equation:
\[ c = 0 \]
Hence the equation of the plane is 
\[ 3x - 3y - 3z = 0 \]

c) According to the problem description, two vectors on the plane are
\[ <1, -1, 1>, <-2, 3, 1> \]
Therefore, a normal vector to the plane can be calculated as
\begin{gather*}
  \begin{split}
    <1, -1, 1> \times <-2, 3, 1> 
    &= \begin{vmatrix}
         \vec{i} & \vec{j} & \vec{k} \\
         1 & -1 & 1 \\
         2 & 3 & 1 \\
       \end{vmatrix} \\
    &= <-4, 1, 5> \\
  \end{split}
\end{gather*}
Therefore, the equation of the plane is
\[ -4x + y + 5z = c, \textnormal{where c is a constant} \]
Then we can put the point (1, 0, 1) into the equation:
\[ c = -4 \times 1 + 0 + 5 \times 1 = 1 \]
Hence the equation of the plane is 
\[ -4x + y + 5z = 1 \]

d) According to the problem description, two vectors on the plane are
\[ <-a, b, 0>, <-a, 0, c> \]
Therefore, a normal vector to the plane can be calculated as
\begin{gather*}
  \begin{split}
    <-a, b, 0> \times <-a, 0, c> 
    &= \begin{vmatrix}
         \vec{i} & \vec{j} & \vec{k} \\
         -a & b & 0 \\
         -a & 0 & c \\
       \end{vmatrix} \\
    &= <bc, ac, ab> \\
  \end{split}
\end{gather*}
Therefore, the equation of the plane is
\[ bcx + acy + abz = k, \textnormal{where k is a constant} \]
Then we can put the point (a, 0, 0) into the equation:
\[ k = bc \cdot a + ac \cdot 0 + ab \cdot 0 = abc \]
Hence the equation of the plane is 
\begin{gather*}
   bcx + acy + abz = abc \\
   \frac{x}{a} + \frac{y}{b} + \frac{z}{c} = 1 \\
\end{gather*}

e) According to the problem description, one vector parallel to the plane can be 
derived from the two points (1, 0, 1) and (0, 1, 1), which is $<-1, 1, 0>$. 
Therefore, a normal vector to the plane can be calculated as
\begin{gather*}
  \begin{split}
    <-1, 1, 0> \times <1, -1, 2> 
    &= \begin{vmatrix}
         \vec{i} & \vec{j} & \vec{k} \\
         -1 & 1 & 0 \\
         1 & -1 & 2 \\
       \end{vmatrix} \\
    &= <2, 2, 0> \\
  \end{split}
\end{gather*}
Therefore, the equation of the plane is 
\[ 2x + 2y = c, \textnormal{where c is a constant} \]
Then we can put the point (1, 0, 1) into the equation:
\[ c = 2 \times 1 + 2 \times 0 = 2 \]
Hence the equation of the plane is
\begin{gather*}
  2x + 2y = 2 \\
  x + y = 1 \\
\end{gather*}

2. Find the dihedral angle between the planes $2x - y + z = 3$ and 
$x + y + 2z = 1$.

Solution:

Suppose that the intersection line between the planes $2x - y + z = 3$ and 
$x + y + 2z = 1$ is $l$. By the definition of dihedral angles, we can find a 
point $O$ on the line $l$, and in the plane $2x - y + z = 3$ find a line 
$OA \perp l$, and in the plane $x + y + 2z = 1$ find a line $OB \perp l$. Then 
the angle $\angle{AOB}$ is the dihedral angle.

Notice that any normal vectors to the plane $2x - y + z = 3$ and the plane 
$x + y + 2z = 1$ are also perpendicular to the line $l$ because $l$ is on both 
planes. Therefore, the vector $\vec{n_1} = <2, -1, 1>$, which is a normal vector 
to the plane $2x - y + z = 3$, and the vector $\vec{n_2} = <1, 1, 2>$, which is 
a normal vector to the plane $x + y + 2z = 1$, are perpendicular to the line $l$.

Therefore, all four vectors $\vec{OA}$, $\vec{OB}$, $\vec{n_1}$, and $\vec{n_2}$ 
are all parallel to the same plane, whose normal vectors are along the line $l$, 
and they form a quadrilateral. Since $\vec{OA} \perp \vec{n_1}$, and 
$\vec{OB} \perp \vec{n_2}$, the angle $\angle{AOB}$ is either equal to or 
supplementary to the angle $\theta$ between two normal vectors $\vec{n_1}$ and 
$\vec{n_2}$.

\begin{gather*}
  \begin{split}
    \cos\theta &= \frac{\vec{n_1} \cdot \vec{n_2}}{|\vec{n_1}||\vec{n_2}|} \\
               &= \frac{3}{\sqrt{6} \times \sqrt{6}} \\
               &= \frac{1}{2} \\
  \end{split} \\
  \theta = \frac{\pi}{3} \\
\end{gather*}
Therefore, the dihedral angle between the two planes are $\frac{\pi}{3}$.

3. Find in parametric form the equations for \\
a) the line through (1, 0, -1) and parallel to $2\vec{i} - \vec{j} + 3\vec{k}$ \\
b) the line through (2, -1, -1) and perpendicular to the plane $x - y + 2z = 3$ \\
c) all lines passing through (1, 1, 1) and lying in the plane $x + 2y - z = 2$

Solution:

a)
\begin{gather*}
  x(t) = 1 + 2t \\
  y(t) = -t \\
  z(t) = -1 + 3t \\
\end{gather*}

b) 
The line is perpendicular to the plane $x - y + 2z = 3$ 
$\iff$ The line is parallel to the normal vectors of the plane $x - y + 2z = 3$
$\iff$ The line is parallel to the vector $<1, -1, 2>$.

Therefore, the parametric equation of the line is
\begin{gather*}
  x(t) = 2 + t \\
  y(t) = -1 - t \\
  z(t) = -1 + 2t \\
\end{gather*}

c)
Since the line is lying in the plane $x + 2y - z = 2$, suppose the line vector 
is $<a, b, c>$, then it satisfies
\begin{gather*}
  <a, b, c> \cdot <1, 2, -1> = 0 \\
  a + 2b - c = 0 \\
  c = a + 2b \\
\end{gather*}
Therefore, the parametric equations of all lines described above are
\begin{gather*}
  x(t) = 1 + at \\
  y(t) = 1 + bt \\
  z(t) = 1 + (a + 2b)t \\
\end{gather*}
where $a$ and $b$ are any constants.

5. The line passing through (1, 1, -1) and perpendicular to the plane 
$x + 2y - z = 3$ intersects the plane $2x - y + z = 1$ at what point?

Solution:

The line is perpendicular to the plane $x + 2y - z = 3$ 
$\iff$ The line is parallel to the normal vectors of the plane $x + 2y - z = 3$
$\iff$ The line is parallel to the vector $<1, 2, -1>$.

Therefore, the parametric equation of the line is
\begin{gather*}
  x(t) = 1 + t \\
  y(t) = 1 + 2t \\
  z(t) = -1 - t \\
\end{gather*}

For the intersection point between the line and the plane, we have
\begin{gather*}
  2x(t) - y(t) + z(t) = 1 \\
  2(1 + t) - (1 + 2t) + (-1 - t) = 1 \\
  t = -1 \\
\end{gather*}
Therefore, the intersection point is $(x(-1), y(-1), z(-1))$, which is 
$(0, -1, 0)$.

6. Show that the distance $D$ from the origin to the plane $ax + by + cz = d$ is 
given by the formula $D = \frac{|d|}{\sqrt{a^2 + b^2 + c^2}}$.

Solution:

We need to find the parametric equation of the line through the origin and 
perpendicular to the plane $ax + by + cz = d$.

The line is perpendicular to the plane $ax + by + cz = d$ 
$\iff$ The line is parallel to the normal vectors of the plane $ax + by + cz = d$
$\iff$ The line is parallel to the vector $<a, b, c>$.

Therefore, the parametric equation of the line is
\begin{gather*}
  x(t) = 0 + at = at \\
  y(t) = 0 + bt = bt \\
  z(t) = 0 + ct = ct \\
\end{gather*}

For the intersection point between the line and the plane, we have
\begin{gather*}
  ax(t) + by(t) + cz(t) = d \\
  a^2t + b^2t + c^2t = d \\
  t = \frac{d}{a^2 + b^2 + c^2} \\
\end{gather*}
Therefore, the intersection point is 
$(\frac{ad}{a^2 + b^2 + c^2}, \frac{bd}{a^2 + b^2 + c^2}, \frac{cd}{a^2 + b^2 + c^2})$

Therefore, the distance $D$ from the origin to the plane is equivalent to the 
distance between these two points:
\begin{gather*}
  \begin{split}
    D &= \sqrt{(\frac{ad}{a^2 + b^2 + c^2})^2 + (\frac{bd}{a^2 + b^2 + c^2})^2 + (\frac{cd}{a^2 + b^2 + c^2})^2)} \\
      &= \sqrt{\frac{(a^2 + b^2 + c^2)d^2}{(a^2 + b^2 + c^2)^2}} \\
      &= \frac{|d|}{\sqrt{a^2 + b^2 + c^2}}
  \end{split}
\end{gather*}

\subsection*{Unit 1F Matrix Algebra}

5. a) Let $A = \begin{pmatrix}
                 0 & 1 \\
                 1 & 1 \\ 
               \end{pmatrix}$. Compute $A^2, A^3$. \\
b) Find $A^2, A^3, A^n$ if $A = \begin{pmatrix}
                                  1 & 1 \\
                                  0 & 1 \\ 
                                \end{pmatrix}$.

Solution:

a) 
\begin{gather*}
  \begin{split}
    A^2 &= \begin{pmatrix}
             0 & 1 \\
             1 & 1 \\ 
           \end{pmatrix} \cdot
           \begin{pmatrix}
             0 & 1 \\
             1 & 1 \\
           \end{pmatrix} \\
        &= \begin{pmatrix}
             1 & 1 \\
             1 & 2 \\ 
           \end{pmatrix} \\
  \end{split} \\
  \begin{split}
    A^3 &= A^2 \cdot A \\
        &= \begin{pmatrix}
             1 & 1 \\
             1 & 2 \\ 
           \end{pmatrix} \cdot
           \begin{pmatrix}
             0 & 1 \\
             1 & 1 \\
           \end{pmatrix} \\
        &= \begin{pmatrix}
             1 & 2 \\
             2 & 3 \\ 
           \end{pmatrix} \\
  \end{split} \\
\end{gather*}

b)
\begin{gather*}
  \begin{split}
    A^2 &= A \cdot A \\
        &= \begin{pmatrix}
             1 & 1 \\
             0 & 1 \\ 
           \end{pmatrix} \cdot
           \begin{pmatrix}
             1 & 1 \\
             0 & 1 \\
           \end{pmatrix} \\
        &= \begin{pmatrix}
             1 & 2 \\
             0 & 1 \\ 
           \end{pmatrix} \\
  \end{split} \\
  \begin{split}
    A^3 &= A^2 \cdot A \\
        &= \begin{pmatrix}
             1 & 2 \\
             0 & 1 \\ 
           \end{pmatrix} \cdot
           \begin{pmatrix}
             1 & 1 \\
             0 & 1 \\
           \end{pmatrix} \\
        &= \begin{pmatrix}
             1 & 3 \\
             0 & 1 \\ 
           \end{pmatrix}
  \end{split} \\
\end{gather*}
The first column in $A$ determines that the first column in $A^{k+1}$ will be 
the same as the first column in $A^k$ where $k$ can be any positive integers.
The second column in $A$ determines that the second column in $A^{k+1}$ will be 
the sum of the first and second columns in $A^k$ where $k$ can be any positive 
integers. Therefore, we can induce that
\begin{gather*}
  A^n = \begin{pmatrix}
          1 & n \\
          0 & 1 \\ 
        \end{pmatrix} \\
\end{gather*}

8. a) If $A \begin{pmatrix}
              1 \\
              0 \\
              0 \\
            \end{pmatrix} = 
            \begin{pmatrix}
              2 \\
              3 \\
              1 \\
            \end{pmatrix}$,
         $A \begin{pmatrix}
              0 \\
              1 \\
              0 \\
            \end{pmatrix} = 
            \begin{pmatrix}
              -1 \\
              0 \\
              4 \\
            \end{pmatrix}$,
         $A \begin{pmatrix}
              0 \\
              0 \\
              1 \\
            \end{pmatrix} = 
            \begin{pmatrix}
              1 \\
              1 \\
              -1 \\
            \end{pmatrix}$, what is the $3 \times 3$ matrix $A$? \\
b) If $A \begin{pmatrix}
           2 \\
           0 \\
           0 \\
         \end{pmatrix} = 
         \begin{pmatrix}
           -2 \\
           0 \\
           4 \\
         \end{pmatrix}$,
      $A \begin{pmatrix}
           1 \\
           1 \\
           1 \\
         \end{pmatrix} = 
         \begin{pmatrix}
           3 \\
           0 \\
           3 \\
         \end{pmatrix}$,
      $A \begin{pmatrix}
           0 \\
           2 \\
           1 \\
         \end{pmatrix} = 
         \begin{pmatrix}
           7 \\
           1 \\
           1 \\
         \end{pmatrix}$, what is the $3 \times 3$ matrix $A$?

Solution:

a) 
According to the definition of matrix product, the result of 
$A \begin{pmatrix}
     1 \\
     0 \\
     0 \\
   \end{pmatrix}$ is the first column of $A$, the result of 
$A \begin{pmatrix}
     0 \\
     1 \\
     0 \\
   \end{pmatrix}$ is the second column of $A$, and the result of 
$A \begin{pmatrix}
     0 \\
     0 \\
     1 \\
   \end{pmatrix}$ is the third column of $A$. Therefore,
\begin{gather*}
  A = \begin{pmatrix}
        2 & -1 & 1 \\
        3 & 0 & 1 \\
        1 & 4 & -1 \\ 
      \end{pmatrix} \\
\end{gather*}

a) (Method 2) 
If we combine the three matrix equations, we can get
\begin{gather*}
  A \cdot \begin{pmatrix}
            1 & 0 & 0 \\
            0 & 1 & 0 \\
            0 & 0 & 1 \\
          \end{pmatrix} =
  \begin{pmatrix}
    2 & -1 & 1 \\
    3 & 0 & 1 \\
    1 & 4 & -1 \\
  \end{pmatrix} \\
  A = \begin{pmatrix}
        2 & -1 & 1 \\
        3 & 0 & 1 \\
        1 & 4 & -1 \\ 
      \end{pmatrix} \cdot
      \begin{pmatrix}
        1 & 0 & 0 \\
        0 & 1 & 0 \\
        0 & 0 & 1 \\
      \end{pmatrix}^{-1} \\
  A = \begin{pmatrix}
        2 & -1 & 1 \\
        3 & 0 & 1 \\
        1 & 4 & -1 \\ 
      \end{pmatrix} \cdot
      \begin{pmatrix}
        1 & 0 & 0 \\
        0 & 1 & 0 \\
        0 & 0 & 1 \\
      \end{pmatrix} \\
  A = \begin{pmatrix}
        2 & -1 & 1 \\
        3 & 0 & 1 \\
        1 & 4 & -1 \\ 
      \end{pmatrix} \\
\end{gather*}

b) (Method 1) 
Suppose $A = \begin{pmatrix}
                  a_1 & a_2 & a_3 \\
                  b_1 & b_2 & b_3 \\
                  c_1 & c_2 & c_3 \\  
                \end{pmatrix}$. Then
\begin{gather*}
  A \begin{pmatrix}
      2 \\
      0 \\
      0 \\
    \end{pmatrix} = 
    \begin{pmatrix}
      -2 \\
      0 \\
      4 \\
    \end{pmatrix} \\
  \begin{pmatrix}
    a_1 & a_2 & a_3 \\
    b_1 & b_2 & b_3 \\
    c_1 & c_2 & c_3 \\
  \end{pmatrix} \cdot 
  \begin{pmatrix}
    2 \\
    0 \\
    0 \\
  \end{pmatrix} = 
  \begin{pmatrix}
    -2 \\
    0 \\
    4 \\
  \end{pmatrix} \\
  \begin{pmatrix}
    2a_1 \\
    2b_1 \\
    2c_1 \\
  \end{pmatrix} = 
  \begin{pmatrix}
    -2 \\
    0 \\
    4 \\
  \end{pmatrix} \\
  a_1 = -1, b_1 = 0, c_1 = 2 \\
\end{gather*}
Also
\begin{gather*}
  A \begin{pmatrix}
      1 \\
      1 \\
      1 \\
    \end{pmatrix} = 
    \begin{pmatrix}
      3 \\
      0 \\
      3 \\
    \end{pmatrix} \\
  \begin{pmatrix}
    a_1 & a_2 & a_3 \\
    b_1 & b_2 & b_3 \\
    c_1 & c_2 & c_3 \\
  \end{pmatrix} \cdot 
  \begin{pmatrix}
    1 \\
    1 \\
    1 \\
  \end{pmatrix} = 
  \begin{pmatrix}
    3 \\
    0 \\
    3 \\
  \end{pmatrix} \\
  \begin{pmatrix}
    a_1 + a_2 + a_3 \\
    b_1 + b_2 + b_3 \\
    c_1 + c_2 + c_3 \\
  \end{pmatrix} = 
  \begin{pmatrix}
    3 \\
    0 \\
    3 \\
  \end{pmatrix} \\
  a_1 + a_2 + a_3 = 3 \\
  b_1 + b_2 + b_3 = 0 \\
  c_1 + c_2 + c_3 = 3 \\
\end{gather*}
And
\begin{gather*}
  A \begin{pmatrix}
      0 \\
      2 \\
      1 \\
    \end{pmatrix} = 
    \begin{pmatrix}
      7 \\
      1 \\
      1 \\
    \end{pmatrix} \\
  \begin{pmatrix}
    a_1 & a_2 & a_3 \\
    b_1 & b_2 & b_3 \\
    c_1 & c_2 & c_3 \\
  \end{pmatrix} \cdot 
  \begin{pmatrix}
    0 \\
    2 \\
    1 \\
  \end{pmatrix} = 
  \begin{pmatrix}
    7 \\
    1 \\
    1 \\
  \end{pmatrix} \\
  \begin{pmatrix}
    2a_2 + a_3 \\
    2b_2 + b_3 \\
    2c_2 + c_3 \\
  \end{pmatrix} = 
  \begin{pmatrix}
    7 \\
    1 \\
    1 \\
  \end{pmatrix} \\
  2a_2 + a_3 = 7 \\
  2b_2 + b_3 = 1 \\
  2c_2 + c_3 = 1 \\
\end{gather*}
Hence we derive the following system of equations:
\begin{gather*}
  \begin{cases}
     a_1 = -1 \\
     b_1 = 0 \\
     c_1 = 2 \\
     a_1 + a_2 + a_3 = 3 \\
     b_1 + b_2 + b_3 = 0 \\
     c_1 + c_2 + c_3 = 3 \\
     2a_2 + a_3 = 7 \\
     2b_2 + b_3 = 1 \\
     2c_2 + c_3 = 1 \\
  \end{cases} \\
\end{gather*}
We can solve this system of equations and get
\begin{gather*}
  A = \begin{pmatrix}
        -1 & 3 & 1 \\
        0 & 1 & -1 \\
        2 & 0 & 1 \\ 
      \end{pmatrix} \\
\end{gather*}

b) (Method 2) 
If we combine the three matrix equations, we can get
\begin{gather*}
  A \cdot \begin{pmatrix}
            2 & 1 & 0 \\
            0 & 1 & 2 \\
            0 & 1 & 1 \\
          \end{pmatrix} =
  \begin{pmatrix}
    -2 & 3 & 7 \\
    0 & 0 & 1 \\
    4 & 3 & 1 \\
  \end{pmatrix} \\
  A = \begin{pmatrix}
        -2 & 3 & 7 \\
        0 & 0 & 1 \\
        4 & 3 & 1 \\ 
      \end{pmatrix} \cdot
      \begin{pmatrix}
        2 & 1 & 0 \\
        0 & 1 & 2 \\
        0 & 1 & 1 \\
      \end{pmatrix}^{-1}
\end{gather*}

9. A square $n \times n$ matrix is called \textbf{orthogonal} if 
$A \cdot A^T = I_n$. Show that this condition is equivalent to saying that \\
a) each row of $A$ is a row vector of length 1. \\
b) two different rows are orthogonal vectors.

Solution:

Suppose $A = \begin{pmatrix}
               a_1 & a_2 & a_3 \\
               b_1 & b_2 & b_3 \\
               c_1 & c_2 & c_3 \\ 
             \end{pmatrix}$, then
$A^T = \begin{pmatrix}
         a_1 & b_1 & c_1 \\
         a_2 & b_2 & c_2 \\
         a_3 & b_3 & c_3 \\ 
       \end{pmatrix}$.
\begin{gather*}
  A \cdot A^T = I_n \\
  \begin{pmatrix}
    a_1 & a_2 & a_3 \\
    b_1 & b_2 & b_3 \\
    c_1 & c_2 & c_3 \\
  \end{pmatrix} \cdot
  \begin{pmatrix}
    a_1 & b_1 & c_1 \\
    a_2 & b_2 & c_2 \\
    a_3 & b_3 & c_3 \\
  \end{pmatrix} = 
  \begin{pmatrix}
    1 & 0 & 0 \\
    0 & 1 & 0 \\
    0 & 0 & 1 \\
  \end{pmatrix} \\
\end{gather*}
Expand the matrix equation, we can derive the following equations:
\begin{gather*}
  a_1^2 + a_2^2 + a_3^3 = 1 \\
  b_1^2 + b_2^2 + b_3^3 = 1 \\
  c_1^2 + c_2^2 + c_3^3 = 1 \\
  a_1b_1 + a_2b_2 + a_3b_3 = 0 \\
  b_1c_1 + b_2c_2 + b_3c_3 = 0 \\
  a_1c_1 + a_2c_2 + a_3c_3 = 0 \\
\end{gather*}
which is equivalent to that each row of $A$ is a row vector of length 1, and two 
different rows in $A$ are orthogonal.

Therefore, the statement in the problem description is proved.

\subsection*{Unit 1G Solving Square Systems; Inverse Matrices}

3. $A = \begin{pmatrix}
          1 & -1 & 1 \\
          0 & 1 & 1 \\
          -1 & -1 & 2 \\
        \end{pmatrix}$, $b = \begin{pmatrix}
                               2 \\
                               0 \\
                               3 \\ 
                             \end{pmatrix}$. Solve $Ax = b$ by finding $A^{-1}$.
                        
Solution:

\begin{gather*}
  Ax = b \\
  x = A^{-1}b \\
  x = \frac{1}{\det(A)}adj(A) \cdot b \\
  x = \begin{pmatrix}
        1 & -1 & 1 \\
        0 & 1 & 1 \\
        -1 & -1 & 2 \\ 
      \end{pmatrix}^{-1} \cdot
      \begin{pmatrix}
        2 \\
        0 \\
        3 \\
      \end{pmatrix} \\
  x = \begin{pmatrix}
        \frac{3}{5} & \frac{1}{5} & -\frac{2}{5} \\
        -\frac{1}{5} & \frac{3}{5} & -\frac{1}{5} \\
        \frac{1}{5} & \frac{2}{5} & \frac{1}{5} \\ 
      \end{pmatrix} \cdot
      \begin{pmatrix}
        2 \\
        0 \\
        3 \\
      \end{pmatrix} \\
  x = \begin{pmatrix}
        0 \\
        -1 \\
        1 \\ 
      \end{pmatrix}
\end{gather*}

4. Referring to Exercise 3 above, solve the system 
\[ x_1 - x_2 + x_3 = y_1, x_2 + x_3 = y_2, -x_1 - x_2 + 2x_3 = y_3 \]
for the $x_i$ as functions of the $y_i$.

Solution:

\begin{gather*}
  \begin{cases}
    x_1 - x_2 + x_3 = y_1 \\
    x_2 + x_3 = y_2 \\
    -x_1 - x_2 + 2x_3 = y_3 \\
  \end{cases} \\
  \begin{pmatrix}
    1 & -1 & 1 \\
    0 & 1 & 1 \\
    -1 & -1 & 2 \\
  \end{pmatrix} \cdot
  \begin{pmatrix}
    x_1 \\
    x_2 \\
    x_3 \\
  \end{pmatrix} = 
  \begin{pmatrix}
    y_1 \\
    y_2 \\
    y_3 \\
  \end{pmatrix} \\
  \begin{pmatrix}
    x_1 \\
    x_2 \\
    x_3 \\
  \end{pmatrix} = 
  \begin{pmatrix}
    1 & -1 & 1 \\
    0 & 1 & 1 \\
    -1 & -1 & 2 \\
  \end{pmatrix}^{-1} \cdot
  \begin{pmatrix}
    y_1 \\
    y_2 \\
    y_3 \\
  \end{pmatrix} \\
  \begin{pmatrix}
    x_1 \\
    x_2 \\
    x_3 \\
  \end{pmatrix} = 
  \begin{pmatrix}
    \frac{3}{5} & \frac{1}{5} & -\frac{2}{5} \\
    -\frac{1}{5} & \frac{3}{5} & -\frac{1}{5} \\
    \frac{1}{5} & \frac{2}{5} & \frac{1}{5} \\ 
  \end{pmatrix} \cdot
  \begin{pmatrix}
    y_1 \\
    y_2 \\
    y_3 \\
  \end{pmatrix} \\
  \begin{cases}
    x_1 = \frac{3}{5}y_1 + \frac{1}{5}y_2 - \frac{2}{5}y_3 \\
    x_2 = -\frac{1}{5}y_1 + \frac{3}{5}y_2 - \frac{1}{5}y_3 \\
    x_3 = \frac{1}{5}y_1 + \frac{2}{5}y_2 + \frac{1}{5}y_3 \\ 
  \end{cases} \\
\end{gather*}

5. Show that $(AB)^{-1} = B^{-1}A^{-1}$, by using the definition of inverse 
matrix.

Solution:

By the definition of inverse matrix,
\begin{gather*}
  AB \cdot (AB)^{-1} = I_n \\
  A^{-1}AB \cdot (AB)^{-1} = A^{-1}I_n \\
  B \cdot (AB)^{-1} = A^{-1}I_n \\
  B^{-1}B \cdot (AB)^{-1} = B^{-1}A^{-1}I_n \\
  (AB)^{-1} = B^{-1}A^{-1}I_n \\
  (AB)^{-1} = B^{-1}A^{-1} \\
\end{gather*}

\subsection*{Unit 1H Theorems about Square Systems}

3. a) For what $c$-value(s) will
\begin{equation*}
  \begin{cases}
    x_1 - x_2 + x_3 = 0 \\
    2x_1 + x_2 + x_3 = 0 \\
    -x_1 + cx_2 + 2x_3 = 0 \\
  \end{cases}
\end{equation*}
have a non-trivial solution? \\
b) For what $c$-value(s) will 
\begin{equation*}
  \begin{pmatrix}
    2 & 1 \\
    0 & -1 \\
  \end{pmatrix} \cdot
  \begin{pmatrix}
    x \\
    y \\
  \end{pmatrix} = c
  \begin{pmatrix}
    x \\
    y \\
  \end{pmatrix}
\end{equation*}
have a non-trivial solution? (Write it as a system of homogeneous equations.) \\
c) For each value of $c$ in part (a), find a non-trivial solution to the 
corresponding system. \\
d) For each value of $c$ in part (b), find a non-trivial solution to the 
corresponding system. \\

Solution:

a) 
\begin{gather*}
  \begin{cases}
    x_1 - x_2 + x_3 = 0 \\
    2x_1 + x_2 + x_3 = 0 \\
    -x_1 + cx_2 + 2x_3 = 0 \\
  \end{cases} \\
  \begin{pmatrix}
    1 & -1 & 1 \\
    2 & 1 & 1 \\
    -1 & c & 2 \\
  \end{pmatrix} \cdot
  \begin{pmatrix}
    x_1 \\
    x_2 \\
    x_3 \\
  \end{pmatrix} = 0
\end{gather*}
This is a homogeneous linear system. It always has a trivial solution 
\begin{equation*}
  \begin{pmatrix}
    x_1 \\
    x_2 \\
    x_3 \\
  \end{pmatrix} = 
  \begin{pmatrix}
    0 \\
    0 \\
    0 \\
  \end{pmatrix}
\end{equation*}

The linear system has a non-trivial solution \\
$\iff$ The linear system has more than one solution \\
$\iff$ The transformation matrix is not inversible \\
$\iff$ The determinant of the transformation matrix is 0

\begin{gather*}
  \det(A) = 0 \\
  \begin{vmatrix}
    1 & -1 & 1 \\
    2 & 1 & 1 \\
    -1 & c & 2 \\
  \end{vmatrix} = 0 \\
  1 \times (2 - c) + 1 \times (4 + 1) + 1 \times (2c + 1) = 0 \\
  c + 8 = 0 \\
  c = -8 \\
\end{gather*}

Therefore when $c = -8$ the linear system have a non-trivial solution.

b)
\begin{gather*}
  \begin{pmatrix}
    2 & 1 \\
    0 & -1 \\
  \end{pmatrix} \cdot
  \begin{pmatrix}
    x \\
    y \\
  \end{pmatrix} = c
  \begin{pmatrix}
    x \\
    y \\
  \end{pmatrix} \\
  \begin{cases}
    2x + y = cx \\
    -y = cy \\
  \end{cases} \\
  \begin{cases}
    (2 - c)x + y = 0 \\
    (c + 1)y = 0 \\
  \end{cases} \\
  \begin{pmatrix}
    2 - c & 1 \\
    0 & c + 1 \\
  \end{pmatrix} \cdot
  \begin{pmatrix}
    x \\
    y \\
  \end{pmatrix} = 
  \begin{pmatrix}
    0 \\
    0 \\
  \end{pmatrix} \\
\end{gather*}

Similar to part (a), it is a homogeneous linear system and it has a trivial 
solution 0. The system has a non-trivial solution $\iff$ the determinant of the 
transformation matrix is 0.

\begin{gather*}
  \det(A) = 0 \\
  \begin{vmatrix}
    2 - c & 1 \\
    0 & c + 1 \\
  \end{vmatrix} = 0 \\
  (c + 1)(2 - c) = 0 \\
  c = 2, c = -1 \\
\end{gather*}

Therefore, when $c = 2$ or $c = -1$, the linear system has a non-trivial 
solution.

c) 
When $c = -8$, the system of equations becomes
\begin{equation*}
  \begin{cases}
    x_1 - x_2 + x_3 = 0 \\
    2x_1 + x_2 + x_3 = 0 \\
    -x_1 - 8x_2 + 2x_3 = 0 \\
  \end{cases}
\end{equation*}

We can try to directly solve it. Though we cannot get a unique solution by 
solving it, we can get some relations between $x_1$, $x_2$, and $x_3$ helping 
us find a valid solution.

If we add up the first and second equations, and add up the first and third 
equations, we can get
\begin{gather*}
  3x_1 + 2x_3 = 0 \\
  -3x_2 + x_3 = 0 \\
\end{gather*}

Therefore, we can construct a solution as $(x_1, x_2, x_3) = (-2, 1, 3)$ and we 
can verify that it is a valid solution.

d)
When $c = 2$, the system of equations becomes
\begin{equation*}
  \begin{cases}
    y = 0 \\
    3y = 0 \\
  \end{cases}
\end{equation*}

Therefore, it is obvious that a non-trivial solution can be $(x, y) = (1, 0)$.

When $c = -1$, the system of equations becomes
\begin{equation*}
  3x + y = 0 \\
  0 = 0 \\
\end{equation*}

Therefore, it is obvious that a non-trivial solution can be $(x, y) = (-1, 3)$.

7. Suppose we want to find a pure oscillation (sine wave) of frequency 1 passing 
through two given points. In other words, we want to choose constants $a$ and 
$b$ so that the function 
\begin{equation*}
  f(x) = a\cos x + b\sin x
\end{equation*}
has prescribed values at two given $x$-values: $f(x_1) = y_1$, $f(x_2) = y_2$: \\
a) Show this is possible in one and only one way, if we assume that 
$x_2 \neq x_1 + n\pi$, for every integer $n$. \\
b) If $x_2 = x_1 + n\pi$ for some integer $n$, when can $a$ and $b$ be found?

Solution:

a)
According to the problem description, the values of $a$ and $b$ satisfy the 
following system of equations:
\begin{gather*}
  \begin{cases}
    f(x_1) = y_1 \\
    f(x_2) = y_2 \\
  \end{cases} \\
  \begin{cases}
    \cos x_1 a + \sin x_1 b = y_1 \\
    \cos x_2 a + \sin x_2 b = y_2 \\
  \end{cases} \\
  \begin{pmatrix}
    \cos x_1 & \sin x_1 \\
    \cos x_2 & \sin x_2 \\
  \end{pmatrix}
  \begin{pmatrix}
    a \\
    b \\
  \end{pmatrix} = 
  \begin{pmatrix}
    y_1 \\
    y_2 \\
  \end{pmatrix}
\end{gather*}

Therefore, the values of $a$ and $b$ satisfy a linear system. According to the 
property of linear systems, we know that: \\
The linear system has a unique solution \\
$\iff$ the determinant of the transformation matrix doesn't equal to 0.

\begin{gather*}
  \begin{split}
    \det(A) 
    &= \begin{vmatrix}
         \cos x_1 & \sin x_1 \\
         \cos x_2 & \sin x_2 \\
       \end{vmatrix} \\
    &= \sin x_2 \cos x_1 - \sin x_1 \cos x_2 \\
    &= \sin(x_2 - x_1) \\
  \end{split}
\end{gather*}

We also know that
\begin{gather*}
  x_2 \neq x_1 + n\pi \\
  x_2 - x_1 \neq n\pi \\
  \sin (x_2 - x_1) \neq 0 \\
  \det(A) \neq 0 \\
\end{gather*}

Therefore, the linear system of $a$ and $b$ has a unique solution.

b)
First of all, if $(y_1, y_2) = (0, 0)$, the linear system of $a$ and $b$ is 
homogeneous, hence it always has a trivial solution which is $(a, b) = (0, 0)$.

For non-homogeneous cases, if $x_2 = x_1 + n\pi$ holds for an odd integer $n$, 
then the linear system becomes
\begin{equation*}
  \begin{cases}
    \cos x_1 a + \sin x_1 b = y_1 \\
    -\cos x_1 a - \sin x_1 b = y_2 \\
  \end{cases}
\end{equation*}
It is obvious that when $y_2 = -y_1$, we can find solutions for $a$ and $b$, 
otherwise the linear system has no solution.

For non-homogeneous cases where $x_2 = x_1 + n\pi$ holds for an even integer 
$n$, the linear system becomes
\begin{equation*}
  \begin{cases}
    \cos x_1 a + \sin x_1 b = y_1 \\
    \cos x_1 a + \sin x_1 b = y_2 \\
  \end{cases}
\end{equation*}
It is obvious that when $y_1 = y_2$, we can find solutions for $a$ and $b$, 
otherwise the linear system has no solution.

\begin{center}
\section*{Part II}
\end{center}

\bigskip

1. Suppose we know that when the three planes $P_1$, $P_2$ and $P_3$ in 
$\mathbb{R}^3$ intersect in pairs, we get three lines $L_1$, $L_2$, and $L_3$ 
which are distinct and parallel. \\
a) Sketch a picture of this situation. \\
b) Show that the three normals to $P_1$, $P_2$, and $P_3$ all lie in one plane, 
using a geometric argument. \\
c) Show that the three normals to $P_1$, $P_2$, and $P_3$ all lie in one plane,
using an algebraic argument.

Solution:

a) TODO(jinxinwang)

b) Suppose that the intersection of $P_1$ and $P_2$ is $L_1$, the intersection 
of $P_2$ and $P_3$ is $L_2$, and the intersection of $P_3$ and $P_1$ is $L_3$. 
Let $\vec{n_1}$ be the normal to $P_1$, $\vec{n_2}$ be the normal to $P_2$, and 
$\vec{n_3}$ be the normal to $P_3$.

According to the intersection relationship, we can derive
\begin{gather*}
  \vec{n_1} \perp L_1 \\
  \vec{n_2} \perp L_1 \\
  \vec{n_2} \perp L_2 \\
  \vec{n_3} \perp L_2 \\
  \vec{n_3} \perp L_3 \\
  \vec{n_1} \perp L_3 \\
\end{gather*}
Since $L_1$, $L_2$, $L_3$ are parallel to each other, and if $l_1 \perp l_2$, 
then $l_1$ is also perpendicular to all lines parallel to $l_2$, we can also 
derive that
\begin{gather*}
  \vec{n_1} \perp L_2 \\
  \vec{n_2} \perp L_3 \\
  \vec{n_3} \perp L_1 \\
\end{gather*}
Most importantly, we derive that
\begin{gather*}
  \vec{n_1} \perp L_1 \\
  \vec{n_2} \perp L_1 \\
  \vec{n_3} \perp L_1 \\
\end{gather*}

Then I claim that if a vector is perpendicular to another vector $\vec{n}$, then 
it lies in the plane whose normal vector is $\vec{n}$. From the geometric 
perspective, this claim can be proved by contradiction:

Suppose that a vector $\vec{a}$ is perpendicular to another vector $\vec{n}$, 
and it doesn't lie in the plane $P$ whose normal vector is $\vec{n}$. From a 
random point $O$ in the plane $P$, we construct a line $l_a$ in the direction of 
$\vec{a}$. From a random point $A$ on the line $l_a$ other than the point $O$, 
we construct a line towards the plane in the same or reverse direction of 
$\vec{n}$, and the intersection point between the line and the plane is $A'$. 
$AOA'$ forms a triangle. Since $\vec{a} \perp \vec{n}$, $AA' \perp OA$ and 
$\angle{OAA'} = \frac{\pi}{2}$. Since $\vec{n}$ is the normal vector of the 
plane $P$ and $OA'$ is on the plane, $AA' \perp OA'$ and 
$\angle{AA'O} = \frac{\pi}{2}$. Since $l_a$ is not on the plane $P$, 
$\angle{AOA'} > 0$. Therefore, the sum of the three angles of the triangle 
$AOA'$ is greater than $\pi$, which is a contradiction. Therefore, any vector 
which is perpendicular to a vector $\vec{n}$ must lie in the plane whose normal 
vector is $\vec{n}$.

Therefore, $\vec{n_1}$, $\vec{n_2}$, and $\vec{n_3}$ all lie in the plane whose 
normal vector is along the line $L_1$.

c) Using the same notation and conclusion as part (b), we derive that 
\begin{gather*}
  \vec{n_1} \perp L_1 \\
  \vec{n_2} \perp L_1 \\
  \vec{n_3} \perp L_1 \\
\end{gather*}

Then I claim that if a vector is perpendicular to another vector $\vec{n}$, then 
it lies in the plane whose normal vector is $\vec{n}$, and I will prove it from 
the algebraic perspective:

Suppose that the equation of the plane is $n_1x + n_2y + n_3z = c$, and one of 
its normal vectors is $\vec{n} = <n_1, n_2, n_3>$. Also suppose that the vector 
that is perpendicular to $\vec{n}$ is $\vec{a}$. Let $P_0$ denote a random point 
on the plane $n_1x + n_2y + n_3z = c$ and $P_0 = (x_0, y_0, z_0)$. Then we have
\begin{equation*}
  n_1x_0 + n_2y_0 + n_3z_0 = c
\end{equation*}
For any points $P = (x, y, z)$ that $\vec{P_0P} \parallel \vec{a}$, since 
$\vec{a} \perp \vec{n}$, then
\begin{gather*}
  <x - x_0, y - y_0, z - z_0> \cdot <n_1, n_2, n_3> = 0 \\
  n_1(x - x_0) + n_2(y - y_0) + n_3(z - z_0) = 0 \\
  n_1x + n_2y + n_3z = n_1x_0 + n_2y_0 + n_3z_0 \\
  n_1x + n_2y + n_3z = c \\
\end{gather*}
Therefore, $P$ is also on the plane, which means the vector $\vec{a}$ lies in 
the plane. Hence it is proved that a vector is perpendicular to another vector 
$\vec{n}$, then it lies in the plane whose normal vector is $\vec{n}$.

Therefore, $\vec{n_1}$, $\vec{n_2}$, and $\vec{n_3}$ all lie in the plane whose 
normal vector is along the line $L_1$.

\bigskip

2. A manufacturing process mixes three raw materials $M_1$, $M_2$, and $M_3$ to 
produce three products $P_1$, $P_2$, and $P_3$. The ratios of the amounts of the 
raw materials (in the order $M_1$, $M_2$, $M_3$) which are used to make up each 
of the three products are as follows: For $P_1$ the ratio is $1:2:3$; for $P_2$ 
the ratio is $1:3:5$; and for $P_3$ the ratio is $3:5:8$. In a certain 
production run, 137 units of $M_1$, 279 units of $M_2$, and 448 units of $M_3$ 
were used. The problem is to determine how many units of each of the products 
$P_1$, $P_2$, and $P_3$ were produced in that run. \\
a) Set this problem up in matrix form. Use the letter $A$ for the matrix, and 
write down the (one-line) formula for the solution in matrix form. \\
b) Compute the inverse matrix of $A$ and use it to solve for the production 
vector $P$. \\
c) Find a choice for the ratios for the third product (in lowest form), 
different from the other ratios, and for which the resulting system has 
non-unique solutions.

Solution:

a) 
Suppose that in the mentioned run, $x_1$ units of $P_1$, $x_2$ units of $P_2$, 
and $x_3$ units of $P_3$ were produced. According to the problem description, we 
can set the following linear system:
\begin{gather*}
  AX = B \\
  \begin{pmatrix}
    1 & 1 & 3 \\
    2 & 3 & 5 \\
    3 & 5 & 8 \\
  \end{pmatrix}
  \begin{pmatrix}
    x_1 \\
    x_2 \\
    x_3 \\
  \end{pmatrix} = 
  \begin{pmatrix}
    137 \\
    279 \\
    448 \\
  \end{pmatrix} \\
  X = A^{-1}B \\
  \begin{pmatrix}
    x_1 \\
    x_2 \\
    x_3 \\
  \end{pmatrix} = 
  \begin{pmatrix}
    1 & 1 & 3 \\
    2 & 3 & 5 \\
    3 & 5 & 8 \\
  \end{pmatrix}^{-1}
  \begin{pmatrix}
    137 \\
    279 \\
    448 \\
  \end{pmatrix} \\
\end{gather*}

b)
\begin{gather*}
  \begin{split}
    A^{-1} &= \frac{1}{\det(A)} adj(A) \\
           &= \frac{1}{1} \begin{pmatrix}
                            -1 & -1 & 1 \\
                            7 & -1 & -2 \\
                            -4 & 1 & 1 \\ 
                          \end{pmatrix}^T \\
           &= \begin{pmatrix}
                -1 & 7 & -4 \\
                -1 & -1 & 1 \\
                1 & -2 & 1 \\ 
              \end{pmatrix} \\
  \end{split} \\
  X = A^{-1}B \\
  \begin{split}
    \begin{pmatrix}
      x_1 \\
      x_2 \\
      x_3 \\
    \end{pmatrix} 
    &= \begin{pmatrix}
         -1 & 7 & -4 \\
         -1 & -1 & 1 \\
         1 & -2 & 1 \\
       \end{pmatrix}
       \begin{pmatrix}
         137 \\
         279 \\
         448 \\
       \end{pmatrix} \\
    &= \begin{pmatrix}
         24 \\
         32 \\
         27 \\
       \end{pmatrix} \\
  \end{split} \\
\end{gather*}

c)
Suppose that the consumption ratio of the product $P_3$ is $a_1 : a_2 : a_3$. 
Then the transformation matrix $A$ in the linear system becomes
\begin{equation*}
  A = \begin{pmatrix}
        1 & 1 & a_1 \\
        2 & 3 & a_2 \\
        3 & 5 & a_3 \\ 
      \end{pmatrix}
\end{equation*}

For a non-homogeneous linear system like the one in this problem, \\
The linear system has non-unique system \\
$\iff$ The transformation matrix $A$ is not inversible \\
$\iff$ The corresponding determinant of the transformation matrix $A$ equals 0.

\begin{gather*}
  \det(A) = 0 \\
  1 \cdot (3a_3 - 5a_2) - 1 \cdot (2a_3 - 3a_2) + a_1 \cdot 1 = 0 \\
  a_1 - 2a_2 + a_3 = 0 \\
\end{gather*}

Hence, a possible choice of the consumption ratio of the product $P_3$ which 
makes the linear system has non-unique solutions is $1 : 1 : 1$.

\bigskip

3. For any plane $P$ which is not parallel to the x-y plane, define the steepest 
direction on $P$ to be the direction of any vector which lies in $P$ and which 
makes the largest (acute) angle with the x-y plane. \\
a) Let $P$ be the plane through the origin with the normal vector $\vec{n}$. 
Derive a formula, in terms of $\vec{n}$, for a vector $\vec{w}$ which points in 
the steepest direction on $P$. \\
b) Now Let $P$ be the plane through the origin which contains two non-parallel 
vectors $\vec{u}$ and $\vec{v}$, where $\vec{u}$ and $\vec{v}$ do not both lie 
in the x-y plane. Derive a formula, in terms of $\vec{u}$ and $\vec{v}$, for a 
vector $\vec{w}$ which points in the steepest direction on $P$.

Solution: (TODO)

a)
Suppose $\vec{n} = <n_1, n_2, n_3>$, and $\vec{w} = <w_1, w_2, w_3>$. Since 
$\vec{n}$ is a normal vector to $P$, and $\vec{w}$ lies in $P$,
\begin{gather*}
  \vec{n} \cdot \vec{w} = 0 \\
  n_1w_1 + n_2w_2 + n_3w_3 = 0 \\
\end{gather*}

Also since $\vec{w}$ makes the largest (acute) angle with the $x$-$y$ plane, 
from the geometric perspective, it means the angle between $\vec{w}$ and the 
unit vector $\vec{k}$ is the smallest, and hence $\vec{w} \cdot \vec{k}$ has the 
largest result.

b)
\begin{equation*}
  \vec{u} \times \vec{v} = \vec{n}
\end{equation*}


\end{document}