\documentclass{article}
\usepackage[utf8]{inputenc}
\usepackage{amsmath}
\usepackage{amssymb}
\usepackage{amsthm}
\usepackage{tikz}
\setlength{\parindent}{0pt}

\newtheorem*{theorem}{Theorem}
\newtheorem*{definition}{Definition}
\newtheorem*{lemma}{Lemma}
\newtheorem*{corollary}{Corollary}
\newtheorem{example}{Example}
\newtheorem{trick}{Trick}
\newtheorem{question}{Question}

\title{Lecture 9: Max-Min and Least Squares}
\author{}
\date{}

\begin{document}
    
\maketitle

\section{Application of Partial Derivatives: Optimization Problem}

Optimization problems refer to the tasks of finding the maximum or minimum point 
of a function. Here we focus on finding the maximum or minimum point of 
multivariable functions, such as $f(x, y)$.

Another concept is local max/min point, which means the function value of this 
point is greater/less than the function values of adjacent points, but it is not 
necessarily the greatest or leaast function value in the function domain.

\begin{theorem}
  If a point in a function $f$ is a local max/min point, then all of the partial 
  derivatives at this point are equal to 0.

  Proof: \\
  We will prove it by contradiction. Suppose there is a local max/min point in a 
  function $f(x_1, x_2, ..., x_n)$, and one of its partial derivative 
  $\frac{\partial f}{\partial x_i}$ is not 0. According to the definition of 
  partial derivatives, in that direction the function values of the two adjacent 
  point are 
  \begin{gather*}
    f(x_1, x_2, ..., x_i + \Delta x_i, ..., x_n) = f(x_1, x_2, ..., x_n) + \frac{\partial f}{\partial x_i} \Delta x_i \\
    f(x_1, x_2, ..., x_i - \Delta x_i, ..., x_n) = f(x_1, x_2, ..., x_n) - \frac{\partial f}{\partial x_i} \Delta x_i \\
  \end{gather*}
  Therefore, one of them is less than the local max/min point and the other is 
  greater than the local max/min point. Hence, it is not a local max/min point.
  There is a contradiction.

  Therefore, it is proved that if a point in a function $f$ is a local max/min 
  point, then all of the partial derivatives at this point are equal to 0.
\end{theorem}

Therefore, A point is a local max/min point \\
$\Rightarrow$ All partial derivatives at that point are 0 \\
$\iff$ The tangent plane at that point is horizontal \\
$\iff$ The point is a critical point.

\section{Example of Optimization Problem: Least Squares}

\end{document}