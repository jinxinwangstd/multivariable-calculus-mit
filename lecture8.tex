\documentclass{article}
\usepackage[utf8]{inputenc}
\usepackage{amsmath}
\usepackage{amssymb}
\usepackage{amsthm}
\usepackage{tikz}
\setlength{\parindent}{0pt}

\newtheorem*{theorem}{Theorem}
\newtheorem*{definition}{Definition}
\newtheorem*{lemma}{Lemma}
\newtheorem*{corollary}{Corollary}
\newtheorem{example}{Example}

\title{Lecture 8: Partial Derivatives}
\author{}
\date{}

\begin{document}
    
\maketitle

\section{Multivariable Functions}

Multivariable functions are functions whose values depend on more than one 
independent variables.

Examples of Multivariable functions:
\begin{itemize}
  \item $f(x, y) = x^2 + y^2$
  \item $f(x, y) = \frac{1}{x + y},\, x + y \neq 0$
  \item $f(x, y) = \frac{1}{\sqrt{y}},\, y > 0$
  \item The temperature of a certain point on earth at a given time is a 
  multivariable function of latitude, longitude, and height.
\end{itemize}

For simplicity, in this course we will mainly study multivariable functions with 
two or three variables, but the concepts apply to any multivariable functions 
with any number of variables.

\section{Visualization of Multivariable Functions with Two Variables}

\subsection{Function Graph}

For a single-variable function $f$, we plot the function graph as $(x, f(x))$, 
which is very straightforward.

For a multivariable function with two variables $f$, we can similarly plot the 
function graph as $(x, y, f(x, y))$, which is a surface in 3D space.

\begin{example}
  Plot the graph of the function $f(x, y) = -y$.

  The equation of the function graph is $z = -y$, which is a plane.
\end{example}

\begin{example}
  Plot the graph of the function $f(x, y) = 1 - x^2 - y^2$.

  The equation of the function graph is $z = 1 - x^2 - y^2$, whose shape is not 
  so easy to be identified. \\
  We can first take a look at the intersection between the function graph and 
  the $x-z$ plane, which means let $y = 0$. Hence, the equation of the 
  intersection is $z = 1 - x^2$, which is a parabola. \\
  We can then take a look at the intersection between the function graph and the 
  $y-z$ plane, which means let $x = 0$. Hence, the equation of the intersection
  is $z = 1 - y^2$, which is a parabola. \\
  Finally, we can take a look at the intersection between the function graph and 
  the $x-y$ plane, which means let $z = 0$. Hence the equation of the 
  intersection is $x^2 + y^2 = 1$, which is a unit circle. \\
  Based on the above information, we can guess the shape of the function graph 
  is as follows:
\end{example}

From the above examples, we can see that the graph of multivariable functions 
with two variables are not only hard to plot, but also often hard to read.

\subsection{Contour Plot}

Contour Plot is another way to visualize multivariable functions with two 
variables. It basically shows all the points where $f(x, y) =$ some fixed 
constants, which are chosen at regular intervals. Examples include contour map,
isotherm map, etc.

From geometric point of view, contour plot is equivalent to using horizontal 
planes, $z =$ some fixed constants, to slice the function graph and combining 
the intersection points.

\section{Partial Derivatives}

\end{document}